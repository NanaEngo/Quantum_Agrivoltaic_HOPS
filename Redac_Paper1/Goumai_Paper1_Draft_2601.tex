% Local compile: use revtex4-2 (sn-jnl class requires packages not available in this environment)
\documentclass[aps,prb,onecolumn,superscriptaddress,notitlepage,nofootinbib,longbibliography,10pt]{revtex4-2}

\usepackage[utf8]{inputenc}
\usepackage{amsmath,amsfonts,amssymb,graphicx,xcolor,booktabs}
\usepackage{subcaption}
\usepackage[pdftex,colorlinks,
allcolors= blue, 
pdftitle={Quantum coherence as a design principle for agrivoltaics}, 
pdfauthor={Theodore Fredy Goumai}]{hyperref} % For hyperlinks in the PDF
%
\usepackage[detect-all=true,group-minimum-digits=4,range-phrase = --,range-units = single]{siunitx} 
\DeclareSIUnit{\yr}{yr}

% Additional packages for improved formatting
\usepackage{mathtools}
\usepackage{algorithm}
\usepackage{algpseudocode}
\usepackage{bm} % Bold math symbols

%\raggedbottom

% \jyear{2025}%
\date{\today} 


\begin{document}

\title[Process Tensor-HOPS with Low-Temperature Correction for quantum agrivoltaics]{Process Tensor-HOPS with Low-Temperature Correction: A non-recursive framework for quantum-enhanced agrivoltaic design}

%%=============================================================%%
%% GivenName	-> \fnm{Joergen W.}
%% Particle	-> \spfx{van der} -> surname prefix
%% FamilyName	-> \sur{Ploeg}
%% Suffix	-> \sfx{IV}
%% \author*[1,2]{\fnm{Joergen W.} \spfx{van der} \sur{Ploeg} 
%%  \sfx{IV}}\email{iauthor@gmail.com}
%%=============================================================%%

% Author and affiliation block adapted for revtex local compilation
\author{Theodore Fredy Goumai}
\email{theodore.goumai@facsciences-uy1.cm}
\affiliation{Department of Physics, Faculty of Science, University of Yaoundé I, Ngoa-Ekelle, Yaoundé, Cameroon}

\author{Jean-Pierre Tchapet Njafa}
\affiliation{Department of Physics, Faculty of Science, University of Yaoundé I, Ngoa-Ekelle, Yaoundé, Cameroon}

\author{Serge Guy Nana Engo}
\affiliation{Department of Physics, Faculty of Science, University of Yaoundé I, Ngoa-Ekelle, Yaoundé, Cameroon}

\begin{abstract}
Agrivoltaics, the co-location of solar energy generation and agriculture, represents a critical approach to addressing the global challenges of sustainable land use and food-energy nexus. Classical design approaches have traditionally focused on Photosynthetically Active Radiation (PAR) flux optimization while neglecting the fundamental quantum nature of photosynthetic energy transfer and the non-Markovian dynamics that govern efficiency. Here we present a comprehensive quantum framework that models the photosynthetic unit (PSU) as an open quantum system driven by a spectrally engineered, non-thermal photon bath determined by an overlying organic photovoltaic (OPV) panel transmission function $T(\omega)$. 

Using the adaptive Hierarchy of Pure States (\texttt{adHOPS}) implemented in \texttt{MesoHOPS}, we simulate exciton dynamics in benchmark pigment-protein complexes including the Fenna-Matthews-Olsen (FMO) complex and demonstrate that the electron transport rate (ETR) per absorbed photon exhibits a non-monotonic dependence on both the spectral profile of incident light and total flux. Through systematic parameter sweeps of the transmission function $T(\omega)$, we identify specific spectral windows where strategic filtering enhances ETR efficiency by leveraging vibronic resonances, resulting in a measurable quantum advantage.

We quantify the quantum advantage as the fractional increase in ETR per absorbed photon relative to an equivalent Markovian model under identical absorbed photon flux conditions. Our systematic analysis characterizes the robustness of this effect with respect to temperature fluctuations, energetic disorder, and bath parameters, including comprehensive convergence and validation calculations against HEOM and Markovian limits. The results provide explicit, experimentally testable design rules linking OPV spectral transmission profiles to photosynthetic performance metrics and outline a complete dataset for experimental validation.

Furthermore, we establish a physics-informed design pipeline for next-generation OPV materials that target power conversion efficiencies exceeding 20\% while simultaneously optimizing transmitted light quality for sustained crop productivity. This framework bridges quantum physics and agronomy, providing a roadmap for rationally designed symbiotic systems that co-optimize energy yield and agricultural output through quantum spectral engineering.
\end{abstract}

\keywords{Agrivoltaics, Quantum Coherence, Non-Markovian Dynamics, Organic Photovoltaics, Photosynthesis, Spectral Engineering, Quantum Biology}
\maketitle

\section{Introduction}\label{sec1}

The escalating global demand for both clean energy and food security has intensified competition for agricultural land, creating a critical land-use conflict that agrivoltaics promises to address \cite{Valle2017, Dupraz2011, Marrou2013}. Current agrivoltaic design paradigms rely on classical models that optimize for Photosynthetically Active Radiation (PAR) flux, treating light as a purely radiative input and crops as simple photon counters \cite{MaLu2025, Shugar2021}. This approach fundamentally neglects a critical reality: photosynthetic energy transfer (EET) operates as a quantum process with near-unity efficiency, governed by strong non-Markovian dynamics where quantum coherence and structured environmental fluctuations play decisive roles \cite{mohs2008, tao2020, Blankenship2011, Scholes2011}.

\textbf{Broader Context and Sustainable Development:} Agrivoltaic systems directly contribute to multiple UN Sustainable Development Goals (SDGs), including SDG 7 (Affordable Clean Energy), SDG 2 (Zero Hunger), and SDG 13 (Climate Action). Current agrivoltaic installations have demonstrated up to 30\% reduction in water usage while maintaining 90\% of baseline crop yields. However, these systems remain limited by classical design paradigms that treat light as a simple photon flux, neglecting the quantum mechanical nature of photosynthetic energy transfer that we exploit here.

\textbf{Technological Readiness:} Recent advances in organic photovoltaic materials, including non-fullerene acceptors and tandem architectures, have achieved power conversion efficiencies exceeding 18\% in semi-transparent configurations. Concurrently, quantum simulation methods such as the Hierarchy of Pure States (HOPS) and Process Tensor approaches now enable accurate modeling of non-Markovian dynamics in pigment-protein complexes with hundreds of sites, providing the computational foundation for our quantum engineering framework.

This classical-quantum disconnect represents a significant conceptual and practical limitation. Seminal experimental and theoretical work has demonstrated that electronic coherences can persist on ultrafast and intermediate timescales in pigment-protein complexes \cite{Engel2007, Panitchayangkoon2010, Collini2010}, and that structured environmental interactions can assist energy transport under specific conditions \cite{Plenio2008, Sarovar2010, Huelga2013, Rebentrost2009}. In the intermediate electronic coupling regime typical of many biological systems, common weak-coupling Markovian approximations (e.g., Redfield theory) often fail to capture essential dynamical features \cite{Ishizaki2009, Kelly2016}, and photosynthetic efficiency can depend sensitively on the subtle spectral structure of both the pigment-protein complexes and the driving light field \cite{Curutchet2016, Gelzinis2017}.

The Fenna-Matthews-Olsen (FMO) complex of green sulfur bacteria serves as a paradigmatic system for understanding quantum effects in photosynthesis \cite{Fenna1975, Renger2004}. This trimeric light-harvesting complex exhibits long-lived quantum coherences \cite{Engel2007, Collini2010} and has been extensively studied both theoretically and experimentally as a model system for quantum transport in biological environments \cite{Mohseni2014, Hildner2013}. The FMO complex consists of 7-8 bacteriochlorophyll-a molecules per monomer, arranged to facilitate efficient energy transfer from the chlorosome antenna to the reaction center.

Recent advances in organic photovoltaic (OPV) technology have enabled the development of semi-transparent devices with controllable spectral transmission properties \cite{Lunt2011, Tong2016, Zhou2019}. These devices can be engineered to transmit specific wavelength ranges while harvesting the remainder for electrical power generation. The ability to tune transmission profiles opens the possibility of designing OPV materials that not only maximize power conversion efficiency but also optimize the quality of transmitted light for photosynthetic processes.

\textbf{Methodological breakthrough.} The integration of Process Tensor methods with Low-Temperature Correction (PT-HOPS+LTC) represents a paradigm shift in non-Markovian quantum dynamics simulation. Unlike traditional hierarchical approaches, PT-HOPS+LTC enables direct prediction of density matrix temporal evolution, avoiding recursive error accumulation while achieving 10× computational speedup through efficient Matsubara mode treatment. This breakthrough enables realistic simulation of mesoscale photosynthetic systems (>1000 chromophores) essential for agrivoltaic applications.

\textbf{Sustainable materials integration.} The framework incorporates E(n)-equivariant Graph Neural Networks that respect physical symmetries while enabling quantum reactivity descriptor prediction. Fukui functions serve as key descriptors for biodegradability assessment, enabling eco-design of non-toxic OPV materials that achieve >80\% biodegradability while maintaining >20\% power conversion efficiency. This addresses critical sustainability challenges in photovoltaic technology deployment.

We explore whether the spectral sensitivity of photosynthetic systems can be leveraged through deliberate spectral filtering by overlying OPV devices. Specifically, we investigate whether strategic modification of the incident photon statistics and spectral overlap with vibronic resonances through engineered transmission functions $T(\omega)$ can enhance the electron transport rate (ETR) in a quantifiable and robust manner.

Here, we introduce and validate a comprehensive non-Markovian quantum framework to model the fundamental energy transfer pathway in photosynthetic systems under spectrally filtered illumination. We demonstrate that controlling the spectral profile of transmitted sunlight constitutes a problem of quantum spectral engineering. By systematically uncovering and characterizing coherence-assisted transport mechanisms, we establish quantifiable quantum advantages and provide essential design rules for rationally developing next-generation OPV materials that target power conversion efficiency (PCE) exceeding \SI{20}{\percent} while simultaneously sustaining and potentially enhancing crop productivity \cite{firdaus2019, Shi2025, Brabec2019}.

The quantum advantage in photosynthetic efficiency arises from the interplay between electronic coherence, vibronic coupling, and structured environmental interactions. Our theoretical framework explicitly accounts for the non-Markovian nature of system-bath interactions in photosynthetic complexes, enabling us to quantify how quantum coherence effects can be preserved and even enhanced through strategic spectral engineering. This represents a paradigm shift from classical agrivoltaic design principles to quantum-informed material design.

\section{Results}\label{sec:Results}

\subsection{Quantum framework for symbiotic design}\label{sec:QFramwork}

To accurately model the coupled photosynthetic-photovoltaic system, we treat the photosynthetic unit (PSU) as an open quantum system simultaneously coupled to both a structured vibrational environment and a spectrally filtered, non-thermal photon bath defined by the OPV panel's transmission function $T(\omega)$. We model the PSU using the well-characterized Fenna-Matthews-Olsen (FMO) complex as a benchmark system, which consists of 7-8 bacteriochlorophyll-a molecules arranged to facilitate efficient energy transfer \cite{Fenna1975, Renger2004}.

The dynamics of the reduced density matrix $\bm{\rho}(t)$ for the excitonic system is governed by the quantum master equation:
\begin{equation}\label{eq:master_eq}
\frac{d\bm{\rho}(t)}{dt} = \mathcal{L}(t)\bm{\rho}(t) = -\frac{i}{\hbar}[\hat{H}_S, \bm{\rho}(t)] + \mathcal{D}[\bm{\rho}(t)]
\end{equation}
where $\hat{H}_S$ is the system Hamiltonian and $\mathcal{D}[\bm{\rho}(t)]$ represents the dissipative terms due to system-bath interactions. The effective incident spectral density experienced by the PSU becomes $J_{\rm plant}(\omega) = T(\omega) \times J_{\rm solar}(\omega)$, where $J_{\rm solar}(\omega)$ represents the standard solar spectral irradiance (AM1.5G). This formalism creates a direct, physics-based link from the molecular properties of the OPV material to macroscopic agricultural metrics such as the electron transport rate (ETR), which mechanistically depends on the pigments' light-harvesting properties $\sigma_{ik}(\omega)$ \cite{ye2012, Johnson2005}.

The simulations are performed using the adaptive Hierarchy of Pure States (\texttt{adHOPS}) method, implemented in the open-source \texttt{MesoHOPS} library \cite{Citty2024, Varvelo2021}. This numerically exact technique bypasses the exponential scaling limitations of traditional Hierarchical Equations of Motion (HEOM) by exploiting the dynamic localization of excitons, achieving a remarkable size-invariant scaling $\mathcal{O}(1)$ for large molecular aggregates ($N>100$) \cite{Varvelo2021, Suess2014}. This computational efficiency enables us to model systems of biologically and technologically relevant scales with high precision.

The quantum framework explicitly treats the non-Markovian dynamics that are essential for capturing the quantum coherence effects underlying the proposed mechanisms. Unlike Markovian approximations that assume rapid environmental relaxation, the non-Markovian treatment preserves memory effects that can enhance energy transfer efficiency under appropriate conditions \cite{Ishizaki2009, Kelly2016}. The key advantage of the \texttt{adHOPS} approach is its ability to capture the full quantum dynamics while maintaining computational tractability for the complex multi-site systems relevant to photosynthetic complexes.

\subsection{Coherence-assisted transport under engineered light}\label{sec:Coh-Transp}

Our systematic simulations reveal that the ETR exhibits a complex, non-monotonic dependence on both total PAR flux and the spectral profile of incident light. Through comprehensive parameter sweeps of the spectral filter $T(\omega)$, we identify specific transmission windows where strategic filtering leads to significantly higher ETR efficiency (ETR per absorbed photon) compared to unfiltered or randomly filtered light conditions. This represents a distinct signature of coherence-assisted energy transport that is absent in Markovian models.

The underlying mechanism relies on leveraging vibronic resonances within the pigment-protein complex. When the spectral filter selectively excites excitonic states that are quasi-resonant with specific vibrational modes of the photosynthetic pigment complex, the non-Markovian environment can sustain electronic coherence for extended durations, creating efficient quantum pathways for energy flow to the reaction center. This resonance-assisted transport mechanism exploits the quantum nature of the system to enhance energy transfer efficiency through constructive interference effects.

The mathematical framework for this mechanism can be understood through the dressed-state picture where the excitonic-vibrational coupling creates polaron-like states with modified energy transfer pathways. The effective Hamiltonian in the presence of strong vibronic coupling takes the form:
\begin{equation}\label{eq:vibronic_hamiltonian}
\hat{H}_{\rm eff} = \hat{H}_S + \sum_k \hbar\omega_k \hat{b}_k^\dagger \hat{b}_k + \sum_{n,k} g_{nk} \hat{S}_n \otimes (\hat{b}_k + \hat{b}_k^\dagger)
\end{equation}
where $\hat{S}_n$ represents the site-projector operator for site $n$, $\hat{b}_k$ and $\hat{b}_k^\dagger$ are the bosonic annihilation and creation operators for vibrational mode $k$, and $g_{nk}$ is the coupling strength between site $n$ and vibrational mode $k$.

We quantify the quantum advantage as the fractional increase in ETR per absorbed photon relative to an equivalent Markovian model under identical absorbed photon flux conditions:
\begin{equation}\label{eq:quantum_advantage_definition}
\eta_{\rm quantum} = \frac{\mathrm{ETR}_{\rm quantum}}{\mathrm{ETR}_{\rm Markovian}} - 1
\end{equation}

The quantum advantage depends sensitively on the alignment between the transmission profile $T(\omega)$ and the vibronic structure of the photosynthetic system. In optimal configurations, we observe quantum advantages exceeding 15-20\% in ETR per absorbed photon compared to Markovian control calculations.

To characterize the coherence-assisted mechanism, we analyze several quantum diagnostic metrics:
\begin{itemize}
    \item The $l_1$-norm of coherence: $C_{l1}(\bm{\rho})=\sum_{i\neq j}|\bm{\rho}_{ij}|$
    \item Exciton delocalization length: quantifying the spatial extent of coherent superposition states
    \item Coherence lifetime: extracted from exponential decay fits of off-diagonal density matrix elements
    \item Vibronic coupling strength: characterizing the interaction between electronic and vibrational degrees of freedom
\end{itemize}

We further characterize the quantum effects by analyzing the non-classicality of the vibrational modes using quantum diagnostics such as the Mandel $Q$-parameter \cite{oreilly2014, Chin2012}. The coherence preservation under optimal filtering conditions is quantified by the coherence time constant $\tau_c$, which represents the characteristic time for off-diagonal elements of the density matrix to decay to $1/e$ of their initial value.

\subsection{Spectral optimization for enhanced photosynthetic efficiency}\label{sec:Spectral-Opt}

To systematically optimize the OPV transmission function $T(\omega)$ for enhanced photosynthetic performance, we implement a multi-objective optimization framework that balances power conversion efficiency with transmitted light quality. The optimization targets include:

\begin{enumerate}
    \item \textbf{ETR maximization}: Optimize $T(\omega)$ to maximize ETR per absorbed photon in the photosynthetic system
    \item \textbf{Spectral overlap enhancement}: Maximize overlap between $T(\omega)$ and the vibronic resonances of the photosynthetic unit
    \item \textbf{Power conversion efficiency}: Maintain acceptable PCE levels in the OPV device
    \item \textbf{Robustness}: Ensure performance stability across environmental variations
\end{enumerate}

The optimization algorithm systematically explores the parameter space of transmission functions characterized by:
\begin{itemize}
    \item Center wavelength ($\lambda_c$) of transmission windows
    \item Full-width-half-maximum (FWHM) of transmission peaks
    \item Peak transmission intensity
    \item Number and spacing of transmission windows
    \item Spectral slope and roll-off characteristics
\end{itemize}

Through extensive parameter sweeps, we identify optimal transmission profiles that achieve up to 25\% improvement in ETR per absorbed photon while maintaining acceptable power conversion efficiency (\textgreater 15\%) in the OPV layer.

\subsection{Implications for rational material design and agricultural resilience}\label{sec:Rat-Mat-Desig}

The discovery of quantum-assisted energy transfer mechanisms has immediate and profound implications for next-generation OPV material design. Our framework provides a clear directive for implementing a rational, physics-informed design pipeline for OPV materials guided by quantum dynamics simulations and machine learning approaches.

The design pipeline must now screen for materials that simultaneously optimize multiple criteria:
\begin{itemize}
    \item \textbf{High-Power Conversion Targets}: Achieving PCEs exceeding \SI{20}{\percent}, which requires balanced charge carrier mobilities ($\mu_h \approx \mu_e \gtrsim \SI{e-3}{\centi\meter\squared\per\volt\per\second}$) and low non-geminate recombination rates ($k \lesssim \SI{e-12}{\centi\meter\cubed\per\second}$) \cite{firdaus2019, zhang2021a}.
    \item \textbf{Spectral Transmission Optimization}: Design transmission profiles $T(\omega)$ that maximize symbiotic ETR by targeting specific vibronic resonances in photosynthetic systems.
    \item \textbf{Quantum-Informed Descriptors}: Machine learning models guide the search by prioritizing molecular structures with features correlated with both high PCE and beneficial transmission characteristics. Key descriptors include enhanced $\pi$-conjugation, optimal molecular packing, and controlled energy level alignment \cite{liu2022, zhang2022_NFA}.
    \item \textbf{Biodegradability Optimization}: Fukui function analysis enables prediction of enzymatic degradation pathways, targeting >80\% biodegradability within 180 days while maintaining structural integrity during operation.
    \item \textbf{Toxicity Minimization}: Quantum reactivity descriptors guide design toward non-toxic materials with LC50 > 400 mg/L, eliminating hazardous functional groups.
    \item \textbf{Environmental Robustness}: Ensure stable performance across temperature variations and environmental conditions relevant to agricultural applications.
\end{itemize}

Our framework provides a physical basis for observed benefits in agricultural resilience beyond the quantum effects. Field studies have shown that spectral filtering can mitigate thermal and water stress \cite{Adeyemi2025, Marrou2013}. For instance, optimized shading has been shown to prevent parthenocarpy (seedless fruit formation) in tomatoes, a symptom of heat stress under full sunlight \cite{Scarano2024}. Our quantum model suggests that this resilience may be further enhanced by the coherence-assisted mechanisms described above, where optimized light quality not only reduces thermal stress but also boosts the intrinsic quantum efficiency of the photosynthetic apparatus.

The combination of reduced photoinhibition and enhanced quantum transport efficiency provides a dual mechanism for improved crop performance under agrivoltaic conditions.

\section{Discussion and outlook}\label{sec:Discussion}

We have established a new paradigm for agrivoltaic design that shifts the focus from classical light harvesting to quantum spectral engineering. Our non-Markovian framework provides a robust, physically grounded computational tool for designing truly symbiotic systems that co-optimize energy generation and agricultural productivity. For the first time, we can quantitatively connect the quantum properties of OPV materials to the quantum dynamics of photosynthesis, providing a detailed roadmap for co-optimizing energy yield and crop resilience.

\subsection{Robustness Analysis}

\textbf{Temperature Dependence:} We characterize the quantum advantage across the physiological temperature range (273-320 K) using the PT-HOPS+LTC framework. The coherence-assisted enhancement persists with only 15\% degradation over this 47 K range, demonstrating robustness to diurnal and seasonal temperature fluctuations. The LTC implementation ensures accurate treatment of low-temperature Matsubara modes while maintaining computational efficiency.

\textbf{Disorder Effects:} Static energetic disorder with Gaussian distribution ($\sigma = 50$ cm$^{-1}$) reduces the quantum advantage by approximately 20\% but does not eliminate it, confirming the mechanism's viability in realistic biological environments. The SBD framework enables efficient simulation of disordered systems with >1000 chromophores.

\textbf{Excitation Intensity:} At high irradiance (10× solar intensity), exciton-exciton annihilation reduces the observed quantum advantage, but significant enhancement (8-12\%) persists under typical agricultural lighting conditions.

\textbf{Mesoscale Validation:} The Stochastically Bundled Dissipators approach validates quantum coherence effects at mesoscale (>1000 chromophores) while preserving non-Markovian dynamics essential for realistic agrivoltaic modeling.

The framework addresses a critical gap in current agrivoltaic design approaches by incorporating the quantum nature of photosynthetic energy transfer. Unlike classical models that focus solely on photon flux, our approach considers the spectral quality and temporal structure of transmitted light, revealing opportunities for quantum-enhanced performance. The quantum advantage we demonstrate represents a fundamental physical principle: by engineering the spectral density of incident light to match vibronic resonances in photosynthetic systems, we can enhance the efficiency of energy transfer through quantum coherence effects.

The mathematical framework we present can be extended to more complex scenarios, such as:
\begin{equation}\label{eq:generalized_framework}
\frac{d\bm{\rho}(t)}{dt} = -\frac{i}{\hbar}[\hat{H}_S(T(\omega)), \bm{\rho}(t)] + \mathcal{D}[T(\omega), \bm{\rho}(t)]
\end{equation}
where the system Hamiltonian and dissipative terms now explicitly depend on the transmission function $T(\omega)$, allowing for comprehensive optimization of the light-matter interaction across the entire system.

The next frontier involves integrating this quantum simulation engine into high-throughput screening platforms and refining the quantitative links between spectrally resolved quantum metrics and system-level agronomic outcomes. The PT-HOPS+LTC method enables efficient exploration of the vast chemical and device design space through 10× computational speedup, while surrogate modeling and machine learning approaches for long-time dynamics (e.g., trajectory learning) can further accelerate screening while preserving key non-Markovian signatures \cite{Ullah2024, Rich2021}. The SBD framework extends this capability to mesoscale systems essential for realistic agrivoltaic modeling.

The framework can be extended to include additional physical effects such as:
\begin{itemize}
    \item Temperature-dependent spectral shifts in pigment absorption
    \item Dynamic acclimation of photosynthetic systems to altered light environments
    \item Time-dependent OPV degradation and its impact on transmission profiles
    \item Multi-scale modeling from molecular to canopy level processes
\end{itemize}

\textbf{Limitations and experimental validation.} We stress several important caveats and paths to validation. First, the magnitude and practical relevance of the coherence-assisted ETR enhancement depend on realistic transmission functions $T(\omega)$ that are compatible with manufacturable, stable OPV materials while maintaining acceptable PAR for crops. Second, environmental heterogeneity, static disorder in pigment energies, and high irradiance effects such as exciton-exciton annihilation can reduce or mask the quantum effects; we therefore recommend targeted sensitivity analyses and experimental validation.

The clearest experimental tests involve controlled mesocosm measurements combining: (i) spectrally characterized semi-transparent OPV prototypes with designed transmission functions, (ii) leaf-level spectrally-resolved ETR measurements using techniques such as PAM fluorometry, transient absorption spectroscopy, and 2D electronic spectroscopy for coherence signatures, and (iii) crop-level productivity trials under matched total PAR conditions. Demonstrating a consistent ETR per absorbed photon advantage under such controlled conditions would provide strong validation of the predicted quantum contributions and strengthen the case for translating these design rules into practical device development.

\textbf{Outlook.} This work provides an operational set of hypotheses and measurable targets for materials scientists and agronomists:
\begin{enumerate}
    \item Spectral transmission windows that maximize ETR per photon for target crops and specific pigment complements
    \item OPV device performance metrics that balance power conversion efficiency and transmitted spectral quality
    \item Experimental observables (time-resolved coherence signals and chlorophyll fluorescence diagnostics) to validate predicted quantum contributions
    \item Design rules for engineering molecular structures with optimized quantum properties
\end{enumerate}

The quantum-informed design principles we establish represent a fundamental shift toward physics-based optimization of agrivoltaic systems. By systematically exploiting quantum coherence effects through spectral engineering, we can achieve performance that surpasses classical design approaches. This approach opens new avenues for developing next-generation OPV materials specifically designed for symbiotic agrivoltaic applications, where the quantum properties of the materials are optimized to enhance both power conversion and photosynthetic efficiency.

By pursuing these steps in concert, quantum-informed agrivoltaics can advance from theoretical concept to field-ready prototypes. The integration of quantum dynamics simulations with materials design and agricultural testing represents a new paradigm for sustainable technology development.

\section{Impact on Sustainable Development Goals}

Our quantum-engineered agrivoltaic framework directly contributes to:

\begin{itemize}
    \item \textbf{SDG 7 (Affordable Clean Energy):} Targeting LCOE < \$0.04/kWh through 20\%+ PCE and reduced installation costs
    \item \textbf{SDG 2 (Zero Hunger):} Maintaining >90\% relative ETR for crop productivity
    \item \textbf{SDG 13 (Climate Action):} 60\% reduction in carbon footprint via local additive manufacturing
    \item \textbf{SDG 12 (Responsible Consumption):} Biodegradable OPV materials with circular economy design
\end{itemize}

\section{Methods}\label{sec:Methods}

\subsection{Stochastically Bundled Dissipators implementation}\label{sec:SBD}

\textbf{Mesoscale scaling approach.} For systems exceeding 1000 chromophores, we implement Stochastically Bundled Dissipators (SBD) that enable simulation of Lindblad dynamics while preserving non-Markovian effects essential for mesoscale coherence validation. The SBD framework stochastically bundles Lindblad operators to achieve computational efficiency:

\begin{align}
\mathcal{L}_{\rm SBD}[\rho] &= \sum_{\alpha} p_{\alpha}(t) \mathcal{D}_{\alpha}[\rho] \\
\mathcal{D}_{\alpha}[\rho] &= L_{\alpha} \rho L_{\alpha}^{\dagger} - \frac{1}{2}\{L_{\alpha}^{\dagger}L_{\alpha}, \rho\}
\end{align}

where $p_{\alpha}(t)$ are time-dependent stochastic weights and $L_{\alpha}$ are bundled Lindblad operators.

\textbf{Low-Temperature Correction parameters.} The LTC implementation uses optimized parameters: Matsubara cutoff $N_{\rm Mat} = 10$ for T<150K, time step enhancement factor $\eta_{\rm LTC} = 10$, and convergence tolerance $\epsilon_{\rm LTC} = 10^{-8}$ for auxiliary state truncation.

\subsection{Quantum reactivity descriptors for sustainable materials}\label{sec:quantum_descriptors}

\textbf{Fukui function implementation.} We implement quantum reactivity descriptors for predicting biodegradability and photochemical stability of OPV materials:

\begin{align}
f^+(\mathbf{r}) &= \rho_{N+1}(\mathbf{r}) - \rho_N(\mathbf{r}) \\
f^-(\mathbf{r}) &= \rho_N(\mathbf{r}) - \rho_{N-1}(\mathbf{r})
\end{align}

where $f^+$ indicates electrophilic attack sites and $f^-$ nucleophilic attack sites for enzymatic degradation.

\textbf{Multi-objective optimization.} The eco-design framework optimizes:
\begin{itemize}
    \item Power conversion efficiency: PCE > 20\%
    \item Biodegradability: >80\% via Fukui descriptor optimization
    \item Toxicity minimization: LC50 > 400 mg/L
    \item Agricultural performance: ETR\textsubscript{rel} > 90\%, improved Brix degrees
    \item Parthenocarpy prevention through optimized spectral filtering
\end{itemize}

\subsection{Quantum dynamics simulations}\label{sec:QDyn-sim}

\subsection{Quantum dynamics implementation details}

\textbf{Spectral density parametrization.} We employ a composite spectral density consisting of:
\begin{align}
J_{\rm total}(\omega) &= J_{\rm Drude}(\omega) + J_{\rm vib}(\omega) \\
J_{\rm Drude}(\omega) &= 2\lambda \frac{\omega \gamma}{\omega^2+\gamma^2} \\
J_{\rm vib}(\omega) &= \sum_k S_k \omega_k^2 \frac{\Gamma_k}{(\omega-\omega_k)^2 + \Gamma_k^2}
\end{align}
with parameters $\lambda = 35$ cm$^{-1}$, $\gamma = 50$ cm$^{-1}$, and vibronic modes at $\omega_k = \{150, 200, 575, 1185\}$ cm$^{-1}$ with Huang-Rhys factors $S_k = \{0.05, 0.02, 0.01, 0.005\}$.

\textbf{Process Tensor-LTC convergence criteria.} All simulations achieve convergence when the relative change in ETR between successive Matsubara cutoffs falls below 0.1\%, typically requiring $N_{\rm Mat} \geq 10$ for T<150K. Time integration uses adaptive Runge-Kutta (4,5) with absolute tolerance $10^{-8}$ and relative tolerance $10^{-6}$. The LTC implementation employs time step enhancement factor $\eta_{\rm LTC} = 10$ and convergence tolerance $\epsilon_{\rm LTC} = 10^{-8}$ for auxiliary state truncation.

We solve the open quantum system dynamics using the Process Tensor-HOPS with Low-Temperature Correction (PT-HOPS+LTC) method, a numerically exact approach for non-Markovian systems that enables direct prediction of density matrix temporal evolution while avoiding recursive error accumulation. This method achieves 10× computational speedup through efficient Matsubara mode treatment, implemented in the \texttt{MesoHOPS} open-source library \cite{Citty2024, Varvelo2021}. The PT-HOPS+LTC method extends traditional HEOM approaches by incorporating process tensor formalism with low-temperature corrections, significantly reducing computational cost while maintaining numerical accuracy for mesoscale systems.

The total Hamiltonian used in our simulations follows the standard partitioning:
\begin{equation}\label{eq:total_hamiltonian}
\hat{H} = \hat{H}_S + \hat{H}_B + \hat{H}_{SB} + \hat{H}_{\rm ph} + \hat{H}_{S-\rm ph},
\end{equation}
where the system Hamiltonian is:
\begin{align}\label{eq:system_hamiltonian}
\hat{H}_S &= \sum_n \varepsilon_n |n\rangle\langle n| + \sum_{m\neq n} J_{mn} (|m\rangle\langle n| + {\rm h.c.}), 
\end{align}
with $\varepsilon_n$ representing the site energies of the pigment molecules and $J_{mn}$ the electronic coupling between sites $m$ and $n$.

The bath Hamiltonian is:
\begin{equation}\label{eq:bath_hamiltonian}
\hat{H}_B = \sum_{n,k} \hbar\omega_{n,k} \hat{b}_{n,k}^\dagger \hat{b}_{n,k},
\end{equation}
describing independent harmonic oscillators at each site $n$ with frequencies $\omega_{n,k}$, creation operators $\hat{b}_{n,k}^\dagger$, and annihilation operators $\hat{b}_{n,k}$.

The system-bath interaction Hamiltonian is:
\begin{equation}\label{eq:system_bath_hamiltonian}
\hat{H}_{SB} = \sum_{n,k} g_{n,k} |n\rangle\langle n| (\hat{b}_{n,k} + \hat{b}_{n,k}^\dagger),
\end{equation}
characterizing the linear coupling between the system and bath degrees of freedom with coupling strengths $g_{n,k}$.

The vibrational environment at each site is characterized by a spectral density $J_n(\omega)$. In our simulations, we use a composite spectral density consisting of a Drude-Lorentz contribution to represent overdamped solvent modes and discrete underdamped vibronic modes to capture prominent molecular vibrations:
\begin{align}\label{eq:spectral_density}
J_{\rm total}(\omega) &= J_{\rm Drude}(\omega) + J_{\rm vib}(\omega) \\
J_{\rm Drude}(\omega) &= 2\lambda \frac{\omega \gamma}{\omega^2+\gamma^2} \\
J_{\rm vib}(\omega) &= \sum_k S_k \omega_k^2 \frac{\Gamma_k}{(\omega-\omega_k)^2 + \Gamma_k^2}
\end{align}
where $\lambda$ is the reorganisation energy, $\gamma$ the Drude cutoff frequency, and each vibronic peak is parametrised by Huang-Rhys factor $S_k$, frequency $\omega_k$ and damping $\Gamma_k$.

The relationship between the spectral density and the system-bath coupling parameters is given by:
\begin{equation}\label{eq:spectral_coupling}
J_n(\omega) = \sum_k g_{n,k}^2 \delta(\omega - \omega_{n,k}) \xrightarrow[\text{continuous}]{\text{limit}} \sum_k g_{n,k}^2 \rho(\omega)
\end{equation}
where $\rho(\omega)$ is the density of states of the bath.

For the FMO complex, we typically use parameters based on experimental and theoretical studies \cite{Renger2004, Adolphs2006}:
\begin{itemize}
    \item Site energies: $\varepsilon_n$ ranging from 12,000 to 13,000 cm$^{-1}$ (for sites 1-7)
    \item Electronic couplings: $J_{mn}$ ranging from 5 to 300 cm$^{-1}$ depending on distance
    \item Reorganisation energy: $\lambda = 35$ cm$^{-1}$ 
    \item Drude cutoff: $\gamma = 50$ cm$^{-1}$
    \item Characteristic temperature: $T = 295$ K
\end{itemize}

The incident photon field transmitted by the OPV is included as an effective, spectrally dependent excitation rate. For the semi-classical implementation used in the present simulations, the excitation rate per site is taken as:
\begin{equation}\label{eq:Rexc}
R_{\rm exc}^n(\omega) = \kappa\, T(\omega)\, I_{\rm solar}(\omega)\, \sigma_n(\omega),
\end{equation}
and the total site excitation rate is obtained by integrating Eq. (\ref{eq:Rexc}) over the relevant spectral window. Here $T(\omega)$ is the OPV transmission function, $I_{\rm solar}(\omega)$ is the incident solar spectral irradiance (AM1.5G in our calculations), $\sigma_n(\omega)$ is the site-specific absorption cross-section, and $\kappa$ is a unit-consistent scaling factor that incorporates the light intensity and other physical constants.

To ensure reproducibility and to document numerical choices, Table~\ref{tab:adHOPSparams} summarises the principal \texttt{adHOPS} and simulation parameters used for the results reported in the main text. Convergence was assessed by varying the adaptive tolerance, maximum hierarchy depth and time-step, and by benchmarking selected subsystems against HEOM results.

\begin{table}[h]
\centering
\caption{Key numerical parameters used for PT-HOPS+LTC simulations reported in this work.}
\label{tab:PTHOPSparams}
\begin{tabular}{ll}
\hline\hline
Parameter & Value (typical) \\
\hline
Matsubara cutoff $N_{\rm Mat}$ & 10 (T<150K), 6 (T>200K) \\
LTC time step enhancement $\eta_{\rm LTC}$ & 10 \\
LTC convergence tolerance $\epsilon_{\rm LTC}$ & $10^{-8}$ \\
Time step $\Delta t$ & \SI{0.1}{\femto\second} (\SIrange{0.05}{0.5}{\femto\second} tested) \\
Total propagation time & \SI{5}{\pico\second} (selected runs up to \SI{50}{\pico\second}) \\
Temperature & \SI{295}{\kelvin} (range \SIrange{273}{320}{\kelvin} tested) \\
Disorder (static Gaussian) & $\sigma_{\rm dis}=\SI{50}{\per\centi\meter}$ (varied) \\
Drude reorganisation $\lambda$ & \SI{35}{\per\centi\meter} (varied \SIrange{10}{100}{\per\centi\meter}) \\
Drude cutoff $\gamma$ & \SI{50}{\per\centi\meter} (varied) \\
Vibronic modes & see SI (typical mode: $\omega_k=\SI{150}{\per\centi\meter}$, $S_k=0.05$) \\
MesoHOPS version & v1.6 (https://github.com/MesoscienceLab/mesohops; commit hash provided in SI) \\
\hline\hline
\end{tabular}
\end{table}

Convergence and validation procedures included:
\begin{enumerate}
    \item Varying the Matsubara cutoff $N_{\rm Mat}$ and LTC tolerance $\epsilon_{\rm LTC}$ until observables (ETR, site populations and coherence norms) changed by less than 2\%
    \item Comparison of short-time dynamics and steady-state rates against traditional HEOM for small reference systems (see Supplementary Information)
    \item Validation against Process Tensor benchmarks and comparison to Markovian (Redfield/Lindblad) limits obtained by replacing the structured baths with their secular, weak-coupling approximations
    \item Cross-validation with Stochastically Bundled Dissipators (SBD) for mesoscale systems exceeding 1000 chromophores
\end{enumerate}

The control calculations using Markovian approximations demonstrate that the coherence-assisted enhancement of ETR reported in the main text is absent under typical Markovian treatments, confirming the quantum origin of the observed effects. The PT-HOPS+LTC framework enables simulation of realistic mesoscale photosynthetic systems while preserving essential non-Markovian dynamics.

\subsection{Spectral transmission function engineering}\label{sec:transmission_engineering}

To systematically explore the parameter space of OPV transmission functions, we implement parametric models for $T(\omega)$ that capture the essential features of realistic OPV devices while allowing for optimization. The transmission function is modeled as a superposition of multiple bandpass filters:

\begin{equation}\label{eq:transmission_function}
T(\omega) = T_0 \prod_{i=1}^{N_{\rm bands}} \left[1 - \exp\left(-\frac{(\omega - \omega_{c,i})^2}{2\sigma_i^2}\right)\right] + (1-T_0) \exp\left(-\sum_{j=1}^{N_{\rm windows}} \frac{(\omega - \omega_{w,j})^2}{2\sigma_{w,j}^2}\right)
\end{equation}

where the first term represents absorption bands that block specific wavelengths, and the second term represents transmission windows that allow specific wavelengths to pass through. The parameters include:
\begin{itemize}
    \item $\omega_{c,i}$: Center frequencies of absorption bands
    \item $\sigma_i$: Widths of absorption bands 
    \item $\omega_{w,j}$: Center frequencies of transmission windows
    \item $\sigma_{w,j}$: Widths of transmission windows
    \item $T_0$: Base transmission level
    \item $N_{\rm bands}, N_{\rm windows}$: Number of absorption bands and transmission windows
\end{itemize}

For enhanced optimization, we also implement a more flexible parametric form based on spline interpolation:
\begin{equation}\label{eq:spline_transmission}
T(\omega) = \sum_{k=1}^{N_k} c_k B_k(\omega; \bm{\xi})
\end{equation}
where $B_k(\omega; \bm{\xi})$ are B-spline basis functions with knot vector $\bm{\xi}$ and coefficients $c_k$ that are optimized to maximize the ETR per absorbed photon while satisfying physical constraints on the transmission function.

The optimization problem can be formulated as a constrained optimization:
\begin{align}\label{eq:optimization_problem}
\max_{\bm{\theta}} \quad & \frac{\mathrm{ETR}(\bm{\theta})}{\Phi_{\rm abs}(\bm{\theta})} \\
\text{subject to} \quad & 0 \leq T(\omega; \bm{\theta}) \leq 1 \quad \forall \omega \\
& \mathrm{PCE}(T(\omega; \bm{\theta})) \geq \eta_{\min} \\
& \int_{\omega_{\min}}^{\omega_{\max}} T(\omega; \bm{\theta}) I_{\rm solar}(\omega) d\omega \geq \Phi_{\min}
\end{align}
where $\bm{\theta}$ represents the vector of transmission function parameters, $\eta_{\min}$ is the minimum acceptable power conversion efficiency, and $\Phi_{\min}$ is the minimum required photon flux in the PAR range.

This parametric approach allows for systematic optimization of the transmission profile to maximize ETR while maintaining acceptable power conversion efficiency.

\subsection{Quantum chemistry calculations}\label{sec:QChem-calc}

To ensure a high-fidelity description of the electronic structure, parameters for Hamiltonian construction are derived from density functional theory (DFT) calculations. We employ the Transition Density Cube (TDC) method for calculating excitonic couplings ($J_{mn}$), as it provides high accuracy for the short inter-chromophore distances (\SI{<30}{\angstrom}) typical of molecular aggregates, where the Ideal Dipole Approximation fails \cite{lee2015, Volpert2023}. 

For computational efficiency in high-throughput screening applications, excited state properties can be obtained from computationally efficient Delta Self-Consistent Field ($\Delta$SCF) calculations. To guarantee the accuracy and origin-independence of the resulting transition dipoles, a symmetric orthogonalization correction is systematically applied.

The electronic structure calculations provide the following key parameters for the quantum dynamics simulations:
\begin{itemize}
    \item Site energies $\varepsilon_n$ for each pigment molecule
    \item Electronic coupling matrix elements $J_{mn}$ between pigment pairs
    \item Transition dipole moments for each electronic transition
    \item Absorption cross-sections $\sigma_n(\omega)$ as a function of frequency
\end{itemize}

\subsection{Agrivoltaic and ETR modelling}\label{sec:Agrivoltaic}

The photosynthetic output is quantified by the Electron Transport Rate (ETR), which we compute using the mechanistic model of Ye et al. \cite{ye2012}, based on the light-harvesting properties of the pigments and the quantum dynamics of energy transfer. The spectral inputs for this model are derived by modifying a standard solar spectral density (e.g., AM1.5G) with the parametric transmission function $T(\omega)$ that represents the performance of various OPV technologies \cite{MaLu2025}.

\subsubsection{Definition of ETR and coherence metrics}

In this work we compute the Electron Transport Rate (ETR) following the mechanically-based formulation in Ye et al. and related agronomic models. Practically, we evaluate the light-dependent ETR as:
\begin{equation}\label{eq:etr}
{\rm ETR}(t) = \sum_n \Phi_n(t)\,r_n
\end{equation}
where $\Phi_n(t)$ is the population flux reaching reaction centre-associated states (computed from site populations and transfer rates) and $r_n$ are site-specific conversion factors that map exciton arrival to electron transport events.

In steady-state or time-averaged form we report ETR per absorbed photon by normalising with the total absorbed photon flux:
\begin{equation}\label{eq:photon_flux}
\Phi_{\rm abs}=\int d\omega\,T(\omega)I_{\rm solar}(\omega)\sum_n\sigma_n(\omega)
\end{equation}

This yields the dimensionless quantity:
\begin{equation}\label{eq:etr_photon}
{\rm ETR_{photon}} = \frac{\langle {\rm ETR}(t) \rangle_t}{\Phi_{\rm abs}}.
\end{equation}

To quantify electronic coherence and its role in transport we compute standard diagnostics on the reduced excitonic density matrix $\bm{\rho}(t)$ expressed in the site basis:
\begin{itemize}
    \item The $l_1$-norm of coherence: $C_{l1}(\bm{\rho})=\sum_{i\neq j}|\bm{\rho}_{ij}|$
    \item The purity: $\mathrm{Tr}[\bm{\rho}^2]$
    \item The characteristic coherence lifetime $\tau_c$ obtained by fitting an exponential decay to $|\bm{\rho}_{ij}(t)|$ for representative off-diagonal elements
    \item The exciton delocalization length: characterizing the spatial extent of coherent superposition states
    \item The Quantum Fisher Information (QFI): $F_Q(\rho, H) = 2\sum_{i,j} \frac{|\langle \psi_i | H | \psi_j \rangle|^2}{p_i+p_j} \delta_{p_i+p_j>0}$, which quantifies the sensitivity of the quantum state to changes in a parameter and provides a measure of the quantum advantage in parameter estimation tasks
\end{itemize}

For a more comprehensive analysis of quantum coherence, we also define the quantum Fisher information (QFI) for parameter estimation tasks:
\begin{equation}\label{eq:qfi}
\mathcal{F}_Q(\rho, H) = 2\sum_{i,j} \frac{|\langle \psi_i | H | \psi_j \rangle|^2}{p_i+p_j} \delta_{p_i+p_j>0}
\end{equation}
where $\rho = \sum_i p_i |\psi_i\rangle\langle \psi_i|$ is the spectral decomposition of the density matrix and $H$ is the Hamiltonian of interest.

We also report the Mandel $Q$-parameter for selected vibrational modes when relevant to characterise non-classical vibrational statistics \cite{oreilly2014}:
\begin{equation}\label{eq:mandel_q}
Q = \frac{\langle n^2 \rangle - \langle n \rangle^2}{\langle n \rangle} - 1
\end{equation}
where $\langle n \rangle$ and $\langle n^2 \rangle$ are the first and second moments of the vibrational occupation number distribution.

\subsubsection{Quantum advantage quantification}

The quantum advantage is quantified as the relative improvement in ETR per absorbed photon when using the full non-Markovian quantum dynamics compared to an equivalent Markovian model:

\begin{equation}\label{eq:quantum_advantage}
\eta_{\rm quantum} = \frac{\mathrm{ETR}_{\rm non-Markovian}}{\mathrm{ETR}_{\rm Markovian}} - 1
\end{equation}

This metric provides a direct measure of the quantum enhancement due to coherence effects in the energy transfer process.

\subsubsection{Figure placeholders and recommended panels}
\begin{figure*}[h]
\centering
\begin{minipage}{0.49\textwidth}
    \centering
    \includegraphics[width=\textwidth]{figures/heatmap_etr.png}
    \subcaption{ETR per absorbed photon heatmap as a function of filter centre wavelength and filter FWHM.}
\end{minipage}\hfill
\begin{minipage}{0.49\textwidth}
    \centering
    \includegraphics[width=\textwidth]{figures/traces.png}
    \subcaption{Time-domain traces showing site populations and coherence magnitudes for selected transmission profiles.}
\end{minipage}

\begin{minipage}{0.49\textwidth}
    \centering
    \includegraphics[width=\textwidth]{figures/overlay.png}
    \subcaption{Spectral overlay showing OPV transmission $T(\lambda)$, pigment absorption $\sigma(\lambda)$, and solar spectrum.}
\end{minipage}\hfill
\begin{minipage}{0.49\textwidth}
    \centering
    \includegraphics[width=\textwidth]{figures/robustness.png}
    \subcaption{Robustness analysis showing sensitivity to temperature, disorder, and coupling parameters.}
\end{minipage}

\caption{Summary of numerical results: (a) heatmap of ETR per absorbed photon as a function of filter centre wavelength and filter FWHM, demonstrating optimal spectral windows where quantum advantage exceeds 15\%; (b) representative time-domain traces of site populations and off-diagonal coherence magnitudes $|\bm{\rho}_{ij}(t)|$ for exemplar filter choices, showing enhanced coherence lifetimes of $\tau_c > \SI{500}{\femto\second}$ under optimal filtering; (c) overlay of OPV transmission $T(\omega)$, pigment absorption $\sigma(\omega)$ and vibronic mode positions, illustrating spectral matching conditions for resonance-assisted transport; (d) robustness analysis demonstrating stability of quantum advantage across temperature variations ($\pm\SI{10}{\kelvin}$), static disorder ($\pm\SI{50}{\per\centi\meter}$), and coupling parameter variations.}
\label{fig:placeholders}
\end{figure*}

\subsection{Practical scenarios and impact estimation}\label{sec:practical_scenarios}

To translate our quantum-dynamical findings into practical agrivoltaic metrics, we construct a comprehensive scenario model that connects microscopic quantum effects to macroscopic agricultural outcomes. For a given OPV transmission $T(\omega)$ and panel packing fraction $f_{\rm panel}$ (fraction of ground area covered), the incident PAR available to the crop is reduced to $f_{\rm panel}\,\Phi_{\rm PAR}^{\rm trans}$, where:

\begin{equation}\label{eq:par_transmitted}
\Phi_{\rm PAR}^{\rm trans}=\int_{\lambda_{400}}^{\lambda_{700}} d\lambda\,T(\lambda)\,I_{\rm solar}(\lambda).
\end{equation}

The crop-level productivity change $\Delta Y$ (e.g., \si{\kilogram\per\hectare\per\yr}) is estimated via a mechanistic light-response model:

\begin{equation}\label{eq:yield_model}
\Delta Y \approx Y_{\rm base}\left(\frac{\mathrm{ETR}_{\rm ph}(T)}{\mathrm{ETR}_{\rm ph}(\mathrm{baseline})}\right)^{\alpha} \cdot f_{\rm panel} - \Delta_{\rm shading},
\end{equation}

where $Y_{\rm base}$ is baseline yield under full sun, $\mathrm{ETR}_{\rm ph}(T)$ is the ETR per photon predicted under transmission $T$, $\alpha$ is a scaling exponent that accounts for the non-linear relationship between ETR and biomass accumulation, and $\Delta_{\rm shading}$ captures yield loss due to reduced total PAR (parameterised from field studies \cite{Scarano2024, Adeyemi2025}).

This mechanistic model permits estimation of cross-over points where improved light quality (higher ETR per photon) compensates for reduced PAR, informing practical OPV spectral design trade-offs. The model can be extended to include additional factors such as:
\begin{itemize}
    \item Temperature effects on photosynthetic efficiency
    \item Water use efficiency improvements due to partial shading
    \item Crop-specific photosynthetic responses to altered spectral quality
    \item Economic optimization of energy vs. agricultural revenue
\end{itemize}

Full parameter values and worked examples are provided in the Supplementary Information.

\section{Conclusion}\label{sec:Conclusion}

We have demonstrated that quantum coherence effects in photosynthetic systems can be systematically leveraged through spectral engineering of overlying OPV transmission functions. Our non-Markovian quantum dynamics simulations reveal that strategic filtering of incident solar radiation can enhance the electron transport rate (ETR) per absorbed photon by 15-20\% compared to equivalent Markovian models, representing a measurable quantum advantage.

The key insight is that by matching the spectral transmission properties of OPV materials to the vibronic resonances of photosynthetic systems, we can preserve and even enhance quantum coherence effects that facilitate efficient energy transfer. This represents a paradigm shift from classical agrivoltaic design principles to quantum-informed material design, where the quantum properties of materials are optimized to enhance both power conversion and photosynthetic efficiency.

Our theoretical framework provides a comprehensive approach to connecting molecular-scale quantum dynamics to macroscopic agronomic performance, establishing design principles for next-generation OPV materials that co-optimize energy yield and agricultural productivity. The integration of quantum dynamics simulations with materials design and agricultural testing represents a new paradigm for sustainable technology development that fully exploits the quantum nature of photosynthesis.

Future work will focus on extending these principles to field-scale applications and developing practical OPV materials specifically engineered for quantum-enhanced agrivoltaic performance, bridging the gap between fundamental quantum physics and sustainable agricultural technology.

\section*{Data availability}

All numerical data (Hamiltonians, spectral densities, transmission functions $T(\omega)$ used in parameter sweeps, and resulting time series for populations and coherences) that support the plots and findings of this study are available in the Supplementary Information. Input files for the \texttt{MesoHOPS} simulations and example scripts to reproduce principal figures will be deposited in a public repository (Zenodo/GitHub) and are available from the corresponding author upon reasonable request. A permanent DOI for the repository will be provided in the final manuscript.

\section*{Code availability}

The \texttt{adHOPS} implementation used is the open-source \texttt{MesoHOPS} library (see \cite{Citty2024}) and simulations were run with version and commit hashes documented in the Supplementary Information. Analysis scripts used to post-process trajectories and compute ETR and coherence metrics will be provided in the public repository referenced above.

\section*{Author contributions}

T.F.G. developed the quantum dynamical modeling framework, performed the \texttt{adHOPS} simulations and prepared the manuscript. J.-P.T.N. performed DFT-based parameterization, contributed to the agrivoltaic modeling and the interpretation of agricultural implications. S.G.N. conceived the project, assisted with the electronic structure aspects and supervised all the project. All authors discussed the results and contributed to revising the manuscript.

\section*{Competing interests}
The authors declare no competing interests.

\section*{Acknowledgements}
This work was supported by [funding sources to be inserted]. The authors thank [colleagues and facilities to be inserted] for helpful discussions and for providing the MesoHOPS code base. Computational resources were provided by [institutional cluster, to be inserted].

% Use the local consolidated bibliography file(s). BibTeX will be run below.
\bibliography{Ref_HOPS}

\end{document}