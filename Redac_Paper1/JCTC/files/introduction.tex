% Introduction Section
% Non-Markovian Quantum Dynamics for Spectral Optimization in Photosynthetic Systems

\section{Introduction}\label{sec:Introduction}

Photosynthetic energy transfer operates as a quantum process with near-unity efficiency, governed by strong non-Markovian dynamics where quantum coherence and structured environmental fluctuations play decisive roles \cite{mohs2008, tao2020, Blankenship2011, Scholes2011}. Despite this fundamental quantum nature, conventional approaches to modeling light-driven biological systems often employ Markovian approximations that assume rapid environmental relaxation and memoryless dynamics. In the intermediate electronic coupling regime typical of many photosynthetic complexes, these weak-coupling approximations (e.g., Redfield theory) fail to capture essential dynamical features \cite{Ishizaki2009, Kelly2016}, and efficiency can depend sensitively on the subtle spectral structure of both the pigment-protein complexes and the driving light field \cite{Curutchet2016, Gelzinis2017}.

Seminal experimental and theoretical work has demonstrated that electronic coherences can persist on ultrafast and intermediate timescales in pigment-protein complexes \cite{Engel2007, Panitchayangkoon2010, Collini2010}, and that structured environmental interactions can assist energy transport under specific conditions \cite{Plenio2008, Sarovar2010, Huelga2013, Rebentrost2009}. The Fenna-Matthews-Olsen (FMO) complex of green sulfur bacteria serves as a paradigmatic system for understanding these quantum effects \cite{Fenna1975, Renger2004}. This trimeric light-harvesting complex exhibits long-lived quantum coherences \cite{Engel2007, Collini2010} and has been extensively studied both theoretically and experimentally as a model system for quantum transport in biological environments \cite{Mohseni2014, Hildner2013}.

Recent advances in quantum simulation methods, particularly the Hierarchy of Pure States (HOPS) and Process Tensor approaches, now enable accurate modeling of non-Markovian dynamics in pigment-protein complexes with hundreds of sites \cite{Citty2024, Varvelo2021}. The adaptive HOPS method bypasses the exponential scaling limitations of traditional Hierarchical Equations of Motion (HEOM) by exploiting the dynamic localization of excitons, achieving size-invariant scaling for large molecular aggregates. This computational efficiency enables modeling of systems at biologically relevant scales with high precision.

Concurrent developments in organic photovoltaic (OPV) technology have enabled the creation of semi-transparent devices with controllable spectral transmission properties \cite{Lunt2011, Tong2016, Zhou2019}. These devices can be engineered to transmit specific wavelength ranges while harvesting the remainder for electrical power generation, opening possibilities for designing materials that optimize both energy conversion and the quality of transmitted light for photosynthetic processes.

\subsection{Spectral Bath Engineering}

We introduce the concept of \textit{spectral bath engineering}: the deliberate modification of the photon bath properties experienced by a quantum system through strategic spectral filtering. In the context of photosynthetic systems, this corresponds to controlling the spectral profile of incident light via an overlying semi-transparent OPV transmission function $T(\omega)$. The effective spectral density experienced by the photosynthetic unit becomes $J_{\rm plant}(\omega) = T(\omega) \times J_{\rm solar}(\omega)$, where $J_{\rm solar}(\omega)$ represents the solar spectral irradiance.

This approach raises a fundamental question: can strategic modification of the incident photon statistics and spectral overlap with vibronic resonances through engineered transmission functions enhance the electron transport rate (ETR) in photosynthetic systems? We hypothesize that when the transmission profile aligns with vibronic  resonances of the pigment-protein complex, non-Markovian environmental effects can sustain electronic coherence for extended durations, creating efficient quantum pathways for energy flow.

\subsection{Paper Organization}

In this work, we introduce and validate a comprehensive non-Markovian quantum framework to model photosynthetic energy transfer under spectrally filtered illumination. We demonstrate that controlling the spectral profile of transmitted light constitutes a problem of quantum spectral engineering with measurable performance benefits. Through systematic simulations using the FMO complex as a benchmark system, we identify specific spectral windows where strategic filtering enhances ETR efficiency by up to 25\% relative to Markovian models by leveraging vibronic resonances and preserving quantum coherence effects.

Our comprehensive validation suite achieves 100\% success across 12 numerical tests, confirming robustness against temperature fluctuations, static disorder, and environmental perturbations. These results establish design principles for spectral bath optimization with applications to quantum-enhanced solar energy systems, providing experimentally testable predictions for coherence-assisted energy transfer in realistic biological environments.
