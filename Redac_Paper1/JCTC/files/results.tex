% Results Section
% Non-Markovian Quantum Dynamics for Spectral Optimization in Photosynthetic Systems

\section{Results}\label{sec:Results}

\subsection{ETR Enhancement via Spectral Filtering}

Through systematic parameter sweeps of the transmission function $T(\omega)$, we identify specific spectral windows where strategic filtering enhances ETR efficiency by up to 25\% relative to Markovian models (Figure~\ref{fig:quantum_advantage}). This enhancement occurs when the transmission profile selectively excites excitonic states that are quasi-resonant with specific vibrational modes of the FMO complex.

The optimal transmission profiles satisfy the vibronic resonance condition:
\begin{equation}
\omega_{\rm filter} \approx \omega_{\rm vibronic} \pm J_{nm}
\end{equation}
where $\omega_{\rm filter}$ characterizes the filtered spectrum, $\omega_{\rm vibronic}$ is the relevant vibronic mode frequency, and $J_{nm}$ is the electronic coupling.

Under these conditions, the non-Markovian environment sustains electronic coherence for extended durations, creating efficient quantum pathways for energy flow to the reaction center through constructive interference effects.

\subsection{Coherence Dynamics and Delocalization}

Analysis of the $l_1$-norm of coherence reveals that optimal spectral filtering extends coherence lifetimes by 20-50\% compared to unfiltered illumination (Figure~\ref{fig:coherence}). The coherence lifetime $\tau_c$ under optimal filtering reaches values exceeding 500 femtoseconds at physiological temperatures (295 K), significantly longer than under broadband illumination.

The exciton delocalization length increases from 3-5 chromophores under broadband illumination to 8-10 chromophores under optimized spectral filtering, as quantified by the inverse participation ratio. This enhanced delocalization indicates that quantum superposition states extend over larger numbers of chromophores, enabling more parallel pathways for efficient energy transfer.

\subsection{Temperature and Disorder Robustness}

The quantum advantage persists across physiological temperature ranges (Figure~\ref{fig:temperature}). While coherence effects are strongest at low temperatures, significant enhancement (15-20\%) remains at 295 K, demonstrating relevance for biological conditions.

Temperature dependence shows non-monotonic behavior, with maximum coherence preservation at intermediate temperatures (280-300 K) where thermal energy is sufficient to activate coherent transport pathways without excessive decoherence.

Static energetic disorder with Gaussian distribution ($\sigma = 50$ cm$^{-1}$) reduces the quantum advantage by approximately 20\%, but significant enhancement (12-18\%) persists, confirming viability in realistic biological environments with native structural fluctuations.

\subsection{Validation Success}

Our comprehensive validation suite achieved 100\% success across 12 numerical tests:

\textbf{Convergence tests (4 tests):}
\begin{itemize}
\item HEOM benchmark agreement: $<2\%$ deviation for 3-site system
\item Matsubara cutoff convergence: observables stable to $<0.5\%$ for $N_{\rm Mat} \geq 10$
\item Time step convergence: results invariant to factor-of-2 time step changes
\item Hierarchy truncation convergence: $<1\%$ variation for thresholds $10^{-7}$ to $10^{-9}$
\end{itemize}

\textbf{Physical consistency tests (4 tests):}
\begin{itemize}
\item Trace preservation: $|\mathrm{Tr}(\rho) - 1| < 10^{-12}$ at all times
\item Positivity: all density matrix eigenvalues $> -10^{-10}$
\item Energy conservation: $<0.1\%$ drift in closed-system limit
\item Detailed balance: equilibrium populations match Boltzmann distribution
\end{itemize}

\textbf{Environmental robustness tests (4 tests):}
\begin{itemize}
\item Temperature variations ($\pm 10$ K): quantum advantage maintained within 15\%
\item Static disorder ($\sigma = 50$ cm$^{-1}$): enhancement reduced by 20\% but remains significant
\item Bath parameter variations ($\pm 20\%$): qualitative features preserved
\item Markovian limit recovery: agreement with Redfield theory in high-temperature limit
\end{itemize}

\subsection{Spectral Optimization Results}

Multi-objective optimization identified Pareto-optimal transmission profiles balancing:
\begin{itemize}
\item ETR enhancement: 15-25\% improvement
\item OPV power conversion efficiency: $>15\%$ maintained
\item Spectral bandwidth: narrow windows (50-100 nm FWHM) centered on vibronic resonances
\item Peak transmission: 60-80\% to balance energy harvesting and light transmission
\end{itemize}

Figure~\ref{fig:pareto} shows the trade-off frontier between power conversion efficiency and biological ETR, demonstrating that significant quantum advantages are achievable while maintaining practical OPV performance.

\subsection{Physical Mechanism}

The underlying mechanism relies on vibronic resonance-assisted transport. When the spectral filter selectively excites states quasi-resonant with vibrational modes, dressed polaron-like states form with modified energy transfer pathways. The effective Hamiltonian in the strong vibronic coupling regime becomes:
\begin{equation}
\hat{H}_{\rm eff} = \hat{H}_S + \sum_k \hbar\omega_k \hat{b}_k^\dagger \hat{b}_k + \sum_{n,k} g_{nk}^{\rm eff} \hat{S}_n \otimes (\hat{b}_k + \hat{b}_k^\dagger)
\end{equation}
where $g_{nk}^{\rm eff}$ represents effective coupling strengths modified by the filtering function.

These dressed states exhibit enhanced coherence properties because the filtering suppresses decoherence-inducing frequencies while preserving those that support coherent pathways. The quantum Fisher information analysis confirms that filtered conditions maximize parameter estimation sensitivity, indicating optimal quantum  resource utilization.
