% Supporting Information - EES Version
% Quantum Spectral Engineering for Enhanced Agrivoltaic Efficiency

\documentclass[11pt]{article}
\usepackage[utf8]{inputenc}
\usepackage{amsmath,amssymb}
\usepackage{graphicx}
\usepackage{hyperref}
\usepackage{natbib}
\usepackage{booktabs}  % Professional quality tables
\usepackage{multirow}  % Multirow cells in tables
\usepackage{cleveref}  % Smart cross-references
\usepackage{siunitx}   % SI units
\usepackage{physics}   % Physics notation
\DeclareSIUnit{\yr}{yr}
\DeclareSIUnit{\tonne}{t}
\DeclareSIUnit{\hectare}{ha}
\DeclareSIUnit{\kWh}{kWh}

\title{Supporting Information:\\
Quantum Spectral Engineering for Enhanced Agrivoltaic Efficiency:\\
Non-Markovian Dynamics in Photosynthetic Energy Transfer}

\author{Steve Cabrel Teguia Kouam$^{2,*}$, Theodore Goumai Vodekoi$^{1}$, Jean-Pierre Tchapet Njafa$^{1}$,\\
Jean-Pierre Nguenang$^{2}$, Serge Guy Nana Engo$^{1}$\\
\\
$^{1}$Department of Physics, Faculty of Science, University of Yaoundé I, Cameroon\\
$^{2}$Department of Physics, Faculty of Science, University of Douala, Cameroon\\
\\
*Corresponding author: \texttt{steve.teguia@univ-douala.cm}}

\date{\today}

\begin{document}

\maketitle

\tableofcontents
\newpage

%=============================================================================
\section{Environmental factor models}\label{si:environmental}
%=============================================================================

This section details environmental factor models used to assess real-world applicability of quantum-optimized agrivoltaic systems across diverse geographic and climatic conditions.

\subsection{Solar spectral modeling}

\subsubsection{Reference spectrum (AM1.5G)}

The baseline solar spectral irradiance follows the ASTM G173-03 reference standard (Air Mass 1.5 Global tilted):
\begin{equation}
J_{\rm solar}^{\rm ref}(\lambda) = J_{\rm AM1.5G}(\lambda) \quad \text{for } \lambda \in \numrange{280}{4000}\,\text{nm}
\end{equation}
with integrated power density $P_{\rm total} = \int J_{\rm solar}^{\rm ref}(\lambda) \dd{\lambda} = \SI{1000}{W/m^2}$.

Photosynthetically active radiation (PAR) range: \numrange{400}{700}\,\text{nm} represents approximately \SI{45}{\percent} of total solar energy (\SI{450}{W/m^2}).

\subsubsection{Geographic variations}

Solar spectra vary by latitude due to atmospheric path length differences. We model this using Beer-Lambert attenuation, $J(\lambda, \theta_z) = J_0(\lambda) \exp[-\tau(\lambda) \cdot \mathrm{AM}(\theta_z)]$, where $J_0(\lambda)$ is the extraterrestrial spectrum, $\tau(\lambda)$ is the wavelength-dependent atmospheric optical depth, and $\mathrm{AM}(\theta_z) = 1/\cos(\theta_z)$ is the air mass for zenith angle $\theta_z$.

Representative locations include temperate (\SI{50}{\degree}N, Germany; average $\mathrm{AM} \approx \numrange{1.3}{2.9}$), subtropical (\SI{20}{\degree}N, India; $\mathrm{AM} \approx \numrange{1.1}{1.5}$), tropical (\SI{0}{\degree}, Kenya; $\mathrm{AM} \approx \numrange{1.0}{1.1}$), and desert regions (\SI{32}{\degree}N, Arizona; $\mathrm{AM} \approx \numrange{1.2}{2.2}$). These cover the primary climatic regimes where agrivoltaics are deployed.

\subsubsection{Seasonal and diurnal variations}

The time-dependent solar zenith angle is calculated as $\cos(\theta_z) = \sin(\phi)\sin(\delta) + \cos(\phi)\cos(\delta)\cos(h)$, where $\phi$ is latitude, $\delta$ is the solar declination (varying by $\pm \SI{23.45}{\degree}$ annually), and $h$ is the hour angle. The seasonal declination follows $\delta(d) = -\SI{23.45}{\degree} \times \cos[\frac{360}{365}(d + 10)]$ for day $d$.

\subsection{Atmospheric effects}

\subsubsection{Aerosol optical depth (AOD)}

Wavelength-dependent aerosol scattering is modeled using the Ångström formula $\tau_{\rm aer}(\lambda) = \beta \lambda^{-\alpha}$, where $\beta$ is the turbidity coefficient (\numrange{0.05}{0.2} for clear to hazy conditions) and $\alpha$ is the Ångström exponent (\numrange{1.0}{1.5} for continental aerosols).

\subsubsection{Water vapor absorption}

Integrated water vapor column depth $w$ affects near-infrared transmission via $T_{\rm H_2O}(\lambda) = \exp[-k_{\rm H_2O}(\lambda) \cdot w \cdot \mathrm{AM}]$, with absorption coefficient $k_{\rm H_2O}(\lambda)$ peaking at \SIlist{940;1100;1400}{nm}. Standard values of $w$ range from \SIrange{0.5}{1}{cm} in desert zones to \SIrange{3}{5}{cm} in tropical zones.

\subsubsection{Cloud cover and diffuse radiation}

Cloud effects are modeled using the clearness index $K_t$, defined as the ratio of measured to extraterrestrial irradiance. Sky conditions are categorized as clear ($K_t > 0.65$), partly cloudy (\numrange{0.35}{0.65}), or overcast ($K_t < 0.35$). The diffuse fraction $k_d$ is then determined using standard empirical correlations based on the clearness index.

\subsection{Dust and soiling effects}

\subsubsection{Particle accumulation model}

Dust accumulation on the OPV surface reduces transmission according to $T_{\rm dust}(t) = T_0 \exp[-\gamma_{\rm dust} \cdot m(t)]$, where $m(t)$ is the accumulated mass per area (\si{mg\per\centi\meter\squared}), $T_0$ is the clean surface transmission, and $\gamma_{\rm dust}$ is the extinction coefficient (\SIrange{0.05}{0.15}{m^2/g}). The accumulation rate $dm/dt = r_{\rm dep} (1 - r_{\rm clean})$ varies by region, from \SIrange{0.1}{0.5}{mg\per\centi\meter\squared\per\day} in grasslands to \SIrange{3}{5}{mg\per\centi\meter\squared\per\day} in desert regions.

\subsubsection{Spectral selectivity of soiling}

Dust preferentially scatters shorter wavelengths, shifting the transmitted spectrum toward the red. Optimal transmission windows (\SIlist{750;820}{nm}) lie in spectral regions less affected by soiling than the blue-green range.

\subsection{Weather modeling}

Daily and seasonal temperature variations are modeled using sinusoidal cycles. Our simulations confirm that the photosynthetic quantum advantage remains significant (\SIrange{18}{26}{\percent}) across the physiological temperature range (\SIrange{280}{310}{K}), supporting year-round viability across diverse climates.

\subsection{Integration with quantum simulations}

Environmental factors modify the effective incident spectrum sampling over time (hourly resolution), geography (four representative sites), and weather/soiling states. Results confirm that quantum advantages persist under realistic variability.

%=============================================================================
\section{Biodegradability assessment}\label{si:biodegradability}
%=============================================================================

Sustainable deployment of agrivoltaic OPV materials requires biodegradability to minimize environmental impact. We employ computational quantum chemistry to assess enzymatic degradation susceptibility.

\subsection{Fukui function analysis}

The Fukui function quantifies local reactivity of molecular sites toward nucleophilic or electrophilic attack, predicting enzymatic degradation pathways.

\subsubsection{Theoretical framework}

Fukui functions are defined as functional derivatives of electron density:
\begin{align}
f^+(\vec{r}) &= \left(\frac{\delta \rho(\vec{r})}{\delta N}\right)_{v(r)}^+ \approx \rho_{N+1}(\vec{r}) - \rho_N(\vec{r}) \quad \text{(electrophilic)}\\
f^-(\vec{r}) &= \left(\frac{\delta \rho(\vec{r})}{\delta N}\right)_{v(r)}^- \approx \rho_N(\vec{r}) - \rho_{N-1}(\vec{r}) \quad \text{(nucleophilic)}\\
f^0(\vec{r}) &= \frac{1}{2}[f^+(\vec{r}) + f^-(\vec{r})] \quad \text{(radical)}
\end{align}

where:
\begin{itemize}
\item $\rho_N(\vec{r})$ = electron density of neutral molecule
\item $\rho_{N+1}(\vec{r})$ = electron density of anionic state
\item $\rho_{N-1}(\vec{r})$ = electron density of cationic state
\end{itemize}

Higher Fukui values indicate more reactive sites, susceptible to enzymatic attack.

\subsubsection{Computational details}

Density Functional Theory (DFT) calculations performed using:
\begin{itemize}
\item Functional: B3LYP (hybrid exchange-correlation)
\item Basis set: 6-31G(d,p) (double-zeta with polarization)
\item Software: Gaussian 16 or ORCA 5.0
\item Convergence: SCF \SI{e-8}{Ha}, geometry optimization \SI{e-5}{Ha\per Bohr}
\end{itemize}

For each candidate OPV molecule:
1. Optimize ground-state geometry (N electrons)
2. Single-point calculation for N+1 electrons (anion)
3. Single-point calculation for N-1 electrons (cation)
4. Compute Fukui functions on molecular grid

\subsection{Global reactivity descriptors}

\subsubsection{Chemical hardness and softness}

Chemical hardness $\eta$ (resistance to electron density change):
\begin{equation}
\eta = \frac{1}{2}(I - A) = \frac{1}{2}(\varepsilon_{\rm LUMO} - \varepsilon_{\rm HOMO})
\end{equation}

Chemical softness $S = 1/\eta$. Softer molecules are more reactive, thus more biodegradable.

\subsubsection{Electrophilicity index}

Global electrophilicity $\omega$:
\begin{equation}
\omega = \frac{\mu^2}{2\eta} = \frac{(I + A)^2}{8(I - A)}
\end{equation}

where $\mu = -(I + A)/2$ is chemical potential, $I$ = ionization energy, $A$ = electron affinity.

\subsubsection{Nucleophilicity index}

Using Koopmans' theorem:
\begin{equation}
N = \varepsilon_{\rm HOMO} - \varepsilon_{\rm HOMO}^{\rm ref}
\end{equation}
referenced to tetracyanoethylene (TCNE, strong electrophile).

\subsection{Enzymatic degradation pathways}

\subsubsection{Hydrolase attack (ester linkages)}

Ester bonds (common in biodegradable polymers) are cleaved by hydrolases. Fukui nucleophilic index $f^-$ at carbonyl carbon predicts susceptibility:
\begin{equation}
k_{\rm hydrolysis} \propto f^-(\text{C}_{\rm carbonyl}) \times S
\end{equation}

Target: $f^- > 0.05$ for rapid biodegradation ($< 1$~year).

\subsubsection{Oxidase attack (aromatic rings)}

Cytochrome P450 enzymes oxidize aromatic systems. High $f^+$ at aromatic carbons indicates vulnerability:
\begin{equation}
k_{\rm oxidation} \propto \max[f^+(\text{C}_{\rm aromatic})] \times \omega
\end{equation}

\subsubsection{Bond dissociation energies}

Weakest bonds are preferential degradation sites:
\begin{equation}
\text{BDE}(\text{A-B}) = E(\text{A}\cdot) + E(\text{B}\cdot) - E(\text{A-B})
\end{equation}

Bonds with BDE $< \SI{300}{kJ\per\mol}$ are readily cleaved by enzymatic radicals.

\subsection{Biodegradability index}

We define composite biodegradability score:
\begin{equation}
B_{\rm index} = w_1 S + w_2 \langle f^- \rangle + w_3 N_{\rm ester} + w_4 (400 - \text{BDE}_{\rm min})
\end{equation}

where:
\begin{itemize}
\item $S$ = global softness
\item $\langle f^- \rangle$ = average nucleophilic Fukui function
\item $N_{\rm ester}$ = number of hydrolyzable ester linkages
\item $\text{BDE}_{\rm min}$ = weakest bond dissociation energy (\si{kJ\per\mol})
\item Weights: $w_1 = 0.3, w_2 = 0.3, w_3 = 0.2, w_4 = 0.2$
\end{itemize}

\textbf{Classification:}
\begin{itemize}
\item $B_{\rm index} > 70$: Highly biodegradable ($<$ 6 months)
\item $50 < B_{\rm index} < 70$: Moderately biodegradable (6--18 months)
\item $30 < B_{\rm index} < 50$: Slowly biodegradable (1.5--5 years)
\item $B_{\rm index} < 30$: Recalcitrant ($>$ 5 years)
\end{itemize}

We evaluated candidate non-fullerene acceptor molecules for quantum-optimized agrivoltaic systems. \textbf{Molecule A (PM6 derivative)} exhibit high biodegradability ($B_{\rm index} = 72$) due to four hydrolyzable ester linkages ($f^-_{\rm max} = 0.08$ at the carbonyl carbon) and a low minimum bond dissociation energy (BDE) of \SI{285}{kJ\per\mol} at the thiophene-ester bond. \textbf{Molecule B (Y6-BO derivative)} is moderately biodegradable ($B_{\rm index} = 58$) with two ester linkages and a minimum BDE of \SI{310}{kJ\per\mol}. Both candidates achieve $>\SI{15}{\percent}$ PCE in semi-transparent configurations while ensuring environmental compatibility.

%=============================================================================
\section{Extended validation data}\label{si:validation}
%=============================================================================

This section provides documentation of all 12 validation tests referenced in the main text.

\subsection{FMO complex Hamiltonian}

The FMO complex consists of 7 bacteriochlorophyll-a (BChl-a) chromophores arranged in a specific geometry. The system Hamiltonian is given by:
\begin{equation}
H_{\rm sys} = \sum_{n=1}^{7} \epsilon_n \dyad{n} + \sum_{n \neq m} J_{nm} \dyad{n}{m}
\end{equation}

\cref{tab:fmo_hamiltonian} provides the complete parameterization based on X-ray crystallographic data and spectroscopic measurements.

\begin{table}[ht]
\centering
\caption{\textbf{FMO complex Hamiltonian parameters.} Site energies ($\epsilon_n$, diagonal) and electronic couplings ($J_{nm}$, off-diagonal) in \si{\per\cm}. Parameters determined from structure-based calculations validated against spectroscopic data. Site energies from optical absorption, couplings from point-dipole approximation corrected with quantum chemistry. This standard parameterization reproduces experimentally observed spectral features and energy transfer dynamics.}
\label{tab:fmo_hamiltonian}
\begin{tabular}{lccccccc}
\toprule
& \textbf{Site 1} & \textbf{Site 2} & \textbf{Site 3} & \textbf{Site 4} & \textbf{Site 5} & \textbf{Site 6} & \textbf{Site 7} \\
\midrule
Site energies & 12410 & 12530 & 12210 & 12320 & 12480 & 12630 & 12440 \\
\midrule
Site 1 & --- & -87.7 & 5.5 & -5.9 & 6.7 & -13.7 & -9.9 \\
Site 2 & -87.7 & --- & 30.8 & 8.2 & 0.7 & 11.4 & 4.7 \\
Site 3 & 5.5 & 30.8 & --- & -53.5 & -2.2 & -9.6 & 6.0 \\
Site 4 & -5.9 & 8.2 & -53.5 & --- & -70.7 & -17.0 & -63.3 \\
Site 5 & 6.7 & 0.7 & -2.2 & -70.7 & --- & 81.1 & -1.3 \\
Site 6 & -13.7 & 11.4 & -9.6 & -17.0 & 81.1 & --- & 39.7 \\
Site 7 & -9.9 & 4.7 & 6.0 & -63.3 & -1.3 & 39.7 & --- \\
\bottomrule
\multicolumn{8}{l}{\scriptsize Source: Adolphs \& Renger (2006). Site 1 is the reaction center-proximal BChl.} \\
\end{tabular}
\end{table}

\textbf{Key features}:
\begin{itemize}
\item Site energies span \SI{420}{\per\cm} (\SI{295}{K} $\approx$ \SI{205}{\per\cm}), ensuring mixed quantum-classical regime
\item Strongest coupling: Site 5--6 (\SI{81.1}{\per\cm})
\item Funneling network: Sites 1, 2, 3 $\rightarrow$ 4, 7 $\rightarrow$ 5, 6 $\rightarrow$ reaction center
\end{itemize}

\subsection{Convergence tests (4 tests)}

We validated adHOPS against numerically exact HEOM for a 3-site model system (site energies \SIlist{12000;12100;12200}{\per\cm}; Drude bath $\lambda = \SI{35}{\per\cm}$ at \SI{295}{K}). The results show a maximum deviation of \SI{1.8}{\percent} at early times ($t < \SI{50}{fs}$) and an average deviation of \SI{0.6}{\percent} over a \SI{1000}{fs} window, passing the \SI{2}{\percent} threshold.

\subsubsection{Test 2: Matsubara cutoff convergence}

Varying $N_{\rm Mat}$ for the FMO system at \SI{295}{K} confirms that $N_{\rm Mat} = 10$ achieves convergence, with observables stable to within \SI{0.3}{\percent} for $N_{\rm Mat} \geq 10$. Production runs utilize $N_{\rm Mat} = 12$ to ensure a sufficient safety margin.

Numerical integration accuracy was verified by comparing results for $\Delta t \in \SIlist{0.5;1.0;2.0}{fs}$. Differences were negligible (\SI{0.08}{\percent} between 0.5 and \SI{1.0}{fs}), and production runs use $\Delta t = \SI{1.0}{fs}$.

\subsubsection{Test 4: Hierarchy truncation convergence}

Parsimonious hierarchy truncation thresholds were tested for sensitivity. Observables vary by less than \SI{0.8}{\percent} for $\epsilon_{\rm trunc} \in \{10^{-7}, 10^{-8}, 10^{-9}\}$. Production runs use $\epsilon_{\rm trunc} = 10^{-8}$ to balance computational cost and precision.

\subsection{Physical consistency tests (4 tests)}

\subsubsection{Test 5: Trace preservation}

Density matrix normalization was maintained with a maximum deviation of $5 \times 10^{-13}$ (machine precision limit) throughout \SI{100}{ps} simulations, with no systematic drift.

\subsubsection{Test 6: Positivity}

The density matrix remained positive semidefinite within numerical precision, with no large negative eigenvalues recorded. The minimum eigenvalue was observed at $-2.1 \times 10^{-11}$, consistent with numerical noise.

In the zero-coupling limit ($\lambda = 0$), system energy drift was limited to \SI{0.08}{\percent} over \SI{100}{ps} with no systematic trend, confirming numerical conservation.

\subsubsection{Test 8: Detailed balance}

Long-time population limits successfully match the Boltzmann distribution, with a maximum deviation of \SI{0.6}{\percent} across the physiological temperature range (\SIrange{280}{310}{K}), confirming consistency with thermodynamic equilibrium.

\subsection{Environmental robustness tests (4 tests)}

Simulations across $T \in \SIlist{285;295;305}{K}$ demonstrate that the quantum advantage remains robust within \SI{15}{\percent} of the \SI{295}{K} reference value despite thermal fluctuations.

\subsubsection{Test 10: Static disorder}

Gaussian disorder added to site energies ($\sigma = \SI{50}{\per\cm}$) resulted in a \SI{20}{\percent} mean reduction in quantum advantage relative to the disorder-free case, but a significant enhancement persisted across the ensemble.

\subsubsection{Test 11: Bath parameter variations}

Varying spectral density parameters ($\lambda, \gamma, \omega_k$) by $\pm \SI{20}{\percent}$ preserved the qualitative features of vibronic resonance, with peak shifts restricted to $< \SI{5}{nm}$, confirming the robustness of engineering predictions.

\subsubsection{Test 12: Markovian limit recovery}

At high temperature ($T = \SI{500}{K}$), adHOPS converges to the Redfield theory result (within \SI{1.8}{\percent} deviation) as the quantum advantage nearly vanishes, confirming correct asymptotic behavior in the Markovian regime.

\subsection{Summary table: validation results}

\begin{table}[h]
\centering
\caption{Complete validation suite results}
\label{tab:validation}
\begin{tabular}{|l|l|c|c|}
\hline
\textbf{Category} & \textbf{Test} & \textbf{Criterion} & \textbf{Result} \\
\hline
\hline
\multirow{4}{*}{Convergence} 
& HEOM Benchmark & $< \SI{2}{\percent}$ deviation & \SI{1.8}{\percent} \checkmark \\
& Matsubara Cutoff & $< \SI{0.5}{\percent}$ change & \SI{0.3}{\percent} \checkmark \\
& Time Step & Invariance & $< \SI{0.1}{\percent}$ \checkmark \\
& Hierarchy Trunc. & $< \SI{1}{\percent}$ variation & \SI{0.8}{\percent} \checkmark \\
\hline
\multirow{4}{*}{Physical} 
& Trace Preservation & $< 10^{-12}$ & $5 \times 10^{-13}$ \checkmark \\
& Positivity & $\lambda_i > -10^{-10}$ & $-2 \times 10^{-11}$ \checkmark \\
& Energy Conservation & $< 0.1\%$ drift & 0.08\% \checkmark \\
& Detailed Balance & Match Boltzmann & 0.6\% dev. \checkmark \\
\hline
\multirow{4}{*}{Robustness} 
& Temperature ($\pm \SI{10}{K}$) & Within \SI{15}{\percent} & \SIrange{12}{16}{\percent} \checkmark \\
& Static Disorder & Persists & \SI{20}{\percent} reduction \checkmark \\
& Bath Parameters & Qualitative & Features preserved \checkmark \\
& Markovian Limit & Redfield agreement & \SI{1.8}{\percent} dev. \checkmark \\
\hline
\multicolumn{3}{|c|}{\textbf{Overall Success Rate}} & \textbf{12/12 (100\%)} \\
\hline
\end{tabular}
\end{table}

%=============================================================================
\section{Complete FMO parameter sets}\label{si:parameters}
%=============================================================================

\subsection{Site energies (Adolphs \& Renger, 2006)}

Room temperature (\SI{295}{K}) site energies for FMO monomer:

\begin{table}[h]
\centering
\begin{tabular}{|c|c|c|}
\hline
\textbf{Site} & \textbf{Energy (\si{\per\cm})} & \textbf{Wavelength (nm)} \\
\hline
1 & 12410 & 806 \\
2 & 12530 & 798 \\
3 & 12210 & 819 \\
4 & 12320 & 812 \\
5 & 12480 & 801 \\
6 & 12630 & 792 \\
7 & 12440 & 804 \\
\hline
\end{tabular}
\caption{FMO site energies}
\end{table}

\subsection{Electronic couplings}

Coupling matrix $J_{nm}$ (\si{\per\cm}, symmetric):

\begin{equation}
\mathbf{J} = \begin{pmatrix}
0 & -104 & 8 & -5 & 6 & -13 & -2 \\
-104 & 0 & 30 & 8 & 2 & 7 & 11 \\
8 & 30 & 0 & -53 & -2 & -9 & -3 \\
-5 & 8 & -53 & 0 & -70 & -17 & -8 \\
6 & 2 & -2 & -70 & 0 & 81 & 3 \\
-13 & 7 & -9 & -17 & 81 & 0 & 39 \\
-2 & 11 & -3 & -8 & 3 & 39 & 0
\end{pmatrix}
\end{equation}

\subsection{Spectral density parameters}

\textbf{Overdamped (Drude--Lorentz) component:}
\begin{itemize}
\item Reorganization energy: $\lambda_D = \SI{35}{\per\cm}$
\item Cutoff frequency: $\gamma_D = \SI{50}{\per\cm}$ (\SI{200}{fs} correlation time)
\end{itemize}

\textbf{Underdamped (vibronic) modes:}

\begin{table}[h]
\centering
\begin{tabular}{|c|c|c|c|}
\hline
\textbf{Mode} & \textbf{Frequency $\omega_k$ (\si{\per\cm})} & \textbf{Huang-Rhys $S_k$} & \textbf{Damping $\gamma_k$ (\si{\per\cm})} \\
\hline
1 & 150 & 0.05 & 10 \\
2 & 200 & 0.02 & 10 \\
3 & 575 & 0.01 & 20 \\
4 & 1185 & 0.005 & 30 \\
\hline
\end{tabular}
\caption{Vibronic mode parameters. Reorganization energies: $\lambda_k = S_k \hbar \omega_k$.}
\end{table}

Total reorganization energy: $\lambda_{\rm total} = \lambda_D + \sum_k \lambda_k \approx \SI{50}{\per\cm}$

%=============================================================================
\section{Process Tensor-HOPS with low-temperature correction}\label{si:pthops}
%=============================================================================

The adHOPS simulations employ Process Tensor decomposition with Low-Temperature Correction (PT-HOPS+LTC) for enhanced computational efficiency at physiological temperatures. The PT-HOPS+LTC method achieves $10\times$ computational speedup compared to traditional HEOM while maintaining $<$\SI{2}{\percent} accuracy for the 7-site FMO complex. \cref{tab:computational_benchmarks} provides detailed performance comparison across methods and system sizes.

\begin{table}[ht]
\centering
\caption{\textbf{Computational performance: PT-HOPS+LTC vs traditional methods.} All simulations performed for \SI{1}{ps} dynamics at \SI{295}{K} with Drude+vibronic bath on AMD EPYC 7542 (32 cores @ \SI{2.9}{GHz}). HEOM provides exact reference for 7-site FMO. Redfield (Markovian) approximation shown for comparison but fails to capture coherence effects. PT-HOPS+LTC achieves near-HEOM accuracy with $10\times$ speedup, enabling large-scale simulations ($N>20$ sites) intractable for HEOM.}
\label{tab:computational_benchmarks}
\begin{tabular}{lcccc}
\toprule
\textbf{Method} & \textbf{System Size} & \textbf{Wall Time} & \textbf{Memory} & \textbf{Accuracy} \\
 & \textbf{(sites)} & \textbf{(hours)} & \textbf{(GB)} & \textbf{(vs HEOM)} \\
\midrule
HEOM (reference) & 7 & 38.2 & 12.4 & Exact \\
PT-HOPS+LTC & 7 & 3.8 & 2.1 & $<$\SI{2}{\percent} deviation \\
Redfield (Markov) & 7 & 0.3 & 0.5 & \SI{18}{\percent} deviation$^*$ \\
\midrule
PT-HOPS+LTC & 24 & 12.5 & 6.3 & N/A$^\dagger$ \\
PT-HOPS+LTC & 100 & 48.7 & 22.1 & N/A$^\dagger$ \\
\bottomrule
\multicolumn{5}{l}{\scriptsize $^*$Markovian methods fail to capture non-Markovian coherence effects.} \\
\multicolumn{5}{l}{\scriptsize $^\dagger$HEOM computationally intractable for $N>10$ sites.} \\
\end{tabular}
\end{table}

The PT-HOPS+LTC method exhibits near-linear scaling with system size for localized excitons, enabling simulations of complete photosynthetic antenna complexes (100+ chromophores) with non-Markovian accuracy.
This method achieves approximately $10\times$ speedup compared to traditional HEOM while maintaining $<$\SI{2}{\percent} accuracy.

\subsection{Padé decomposition of bath correlation function}

The bath correlation function $C(t)$ is decomposed via Padé approximation into exponentially decaying terms plus a residual non-exponential component:
\begin{equation}
K_{\rm PT}(t,s) = \sum_{k} g_k(t) f_k(s) e^{-\lambda_k |t-s|} + K_{\rm non-exp}(t,s)
\end{equation}

where $g_k(t)$ and $f_k(s)$ are effective coupling functions, $\lambda_k$ are decay rates, and $K_{\rm non-exp}(t,s)$ captures residual memory effects beyond the exponential approximation.

\subsection{Low-temperature correction parameters}

For simulations at physiological temperatures and below, the Low-Temperature Correction (LTC) integrate quantum noise while maintaining computational efficiency. Optimized parameters include a Matsubara cutoff of $N_{\rm Mat} = 12$ for $T = \SI{295}{K}$, a time step enhancement factor $\eta_{\rm LTC} = 10$, a convergence tolerance $\epsilon_{\rm LTC} = 10^{-8}$ for auxiliary state truncation, and a memory kernel truncation of \SI{20}{ps} (sufficiently beyond system decoherence timescales).

Benchmark comparisons for the 7-site FMO system at \SI{295}{K} (\SI{100}{ps} simulation) show that PT-HOPS+LTC achieves a \num{10}-fold speedup relative to traditional HEOM (3.8 vs 38 hours) on a single core. The maximum observable deviation remains within \SI{1.4}{\percent}, confirming accuracy within convergence tolerances. This efficiency stems from optimized Matsubara mode treatment, enhanced time stepping stability, and adaptive truncation.

This computational efficiency enables high-throughput screening of OPV transmission functions and disorder ensembles essential for realistic agrivoltaic design optimization.

%=============================================================================
\section{Computational performance metrics}\label{si:performance}
%=============================================================================

\subsection{Hardware specifications}

Simulations performed on:
\begin{itemize}
\item CPU: AMD EPYC 7542 (32 cores @ \SI{2.9}{GHz})
\item RAM: \SI{256}{GB} DDR4--3200
\item OS: Ubuntu 20.04 LTS
\item Compiler: GCC 9.4.0 with -O3 optimization
\item MPI: OpenMPI 4.0.5 (for parallel ensemble runs)
\end{itemize}

\subsection{Scaling analysis}

Typical production runs for the 7-site FMO system (\SI{100}{ps}, $\Delta t = \SI{1.0}{fs}$) utilize 12 Matsubara modes and an adaptive hierarchy (averaging \numrange{800}{1200} states), requiring approximately 3.8 hours and \SI{4.2}{GB} of peak memory on a single core. Scaling analysis demonstrates that adHOPS remains size-invariant for localized excitons ($\mathcal{O}(1)$), whereas HEOM exhibits $\mathcal{O}(N^3)$ scaling, making adHOPS the preferred method for $N > 5$.

\subsection{Parallelization efficiency}

Ensemble averaging over 100 independent disorder realizations demonstrates near-perfect parallel scalability. A total wall time of 4 hours was achieved on 100 cores, compared to 16.7 days on a single core.

\section{Additional figures}\label{si:figures}

\subsection{Figure S1: spectral density components}

\begin{figure}[ht]
\centering
\includegraphics[width=0.85\textwidth]{Graphics/Spectral_Density_Components_for_FMO_Environment.pdf}
\caption{Spectral density components for FMO environmental bath. Overdamped Drude contribution (blue, $\lambda = \SI{35}{\per\cm}$, $\gamma = \SI{50}{\per\cm}$) and underdamped vibronic modes (orange peaks at \SIlist{150;200;575;1185}{\per\cm}). Total spectral density $J(\omega)$ shown in black. The \SI{575}{\per\cm} mode plays critical role in quantum-enhanced energy transfer.}
\label{fig:SI_spectral_density}
\end{figure}

\subsection{Figure S2: global reactivity indices}

\begin{figure}[ht]
\centering
\includegraphics[width=0.85\textwidth]{Graphics/Global_Reactivity_Indices.pdf}
\caption{Global reactivity indices for biodegradable OPV candidates. Fukui functions $f^+$ (electrophilic, red) and $f^-$ (nucleophilic, blue) identify reactive sites susceptible to enzymatic degradation. Chemical hardness $\eta$, softness $S$, and biodegradability index $B$ shown for Molecule A (highly biodegradable, $<$ 6 months) and Molecule B (moderately biodegradable, 6--18 months). Both candidates achieve $>\SI{15}{\percent}$ PCE while maintaining environmental compatibility.}
\label{fig:SI_biodegradability}
\end{figure}

\subsection{Figure S3: PAR transmission (clean vs dusty)}

\begin{figure}[ht]
\centering
\includegraphics[width=0.85\textwidth]{Graphics/PAR_Transmission__Clean_vs_Dusty_Conditions.pdf}
\caption{Photosynthetically active radiation (PAR) transmission spectra under varying dust accumulation on OPV surface. Clean surface (black solid), 30-day accumulation (blue dashed), 90-day accumulation (red dotted). Critical quantum resonance windows (\numlist{750;820}\,nm, shaded regions) maintain effectiveness despite \SIrange{10}{18}{\percent} transmission reduction from soiling. Regular cleaning (monthly) recommended for optimal performance.}
\label{fig:SI_par_transmission}
\end{figure}

\subsection{Figure S4: response functions}

\begin{figure}[ht]
\centering
\includegraphics[width=0.85\textwidth]{Graphics/Response_Functions__OPV_vs_PSU.pdf}
\caption{Spectral response functions for organic photovoltaic (OPV, blue) and photosynthetic unit (PSU, orange). Optimal dual-band design (\numlist{750;820}\,nm, shaded green) minimizes spectral overlap for efficient electrical energy harvesting while maximizing targeted excitation of vibronic-resonant transitions in photosynthesis. This strategic partitioning enables simultaneous optimization of both energy conversion pathways.}
\label{fig:SI_response_functions}
\end{figure}

\subsection{Figure S5: geographic climate maps}

\begin{figure}[ht]
\centering
\includegraphics[width=0.9\textwidth]{Graphics/fLatitude__lat__u00b0__Month__month.pdf}
\caption{Geographic and seasonal variation of quantum ETR enhancement as function of latitude and month. Contour map showing year-round viability across temperate (\SIrange{40}{70}{\degree}N), subtropical (\SIrange{15}{35}{\degree}N), tropical (\SIrange{0}{23.5}{\degree}), and desert regions (\SIrange{20}{47}{\degree}N/S). Color scale represents ETR enhancement percentage (\SIrange{18}{28}{\percent}). Peak performance occurs at mid-latitudes during spring/fall when temperatures align with optimal \SI{295}{K}. Global deployment potential confirmed.}
\label{fig:SI_climate_map}
\end{figure}

\subsection{Figure S6: ETR uncertainty distributions}

\begin{figure}[ht]
\centering
\includegraphics[width=0.85\textwidth]{Graphics/ETR_Uncertainty_Distribution.pdf}
\caption{Statistical distribution of ETR enhancement from disorder ensemble simulation ($N = 100$ independent realizations, static disorder $\sigma = \SI{50}{\per\cm}$). Histogram (blue bars) shows mean enhancement \SI{20}{\percent}, standard deviation \SI{4}{\percent}. Gaussian fit (red curve) demonstrates near-normal distribution. Inset: quantile-quantile plot confirms statistical robustness. Narrow distribution (coefficient of variation $< \SI{20}{\percent}$) indicates quantum advantage is robust feature, not sensitive to specific molecular configurations.}
\label{fig:SI_etr_uncertainty}
\end{figure}

%=============================================================================
\bibliographystyle{plain}
\bibliography{references}
%=============================================================================

\end{document}
