% Supporting Information - EES Version
% Quantum Spectral Engineering for Enhanced Agrivoltaic Efficiency

\documentclass[11pt,a4paper]{article}
\usepackage[utf8]{inputenc}
\usepackage[T1]{fontenc}
\usepackage{amsmath,amssymb,amsfonts,graphicx,booktabs,bm,multirow,siunitx,physics}
\usepackage{xcolor,hyperref,cleveref}
\usepackage{natbib}
\usepackage[margin=2cm]{geometry}

% Hyperref setup
\hypersetup{
    pdftitle={Quantum Spectral Engineering for Enhanced Agrivoltaic Efficiency - Supporting Information},
    pdfauthor={Teguia et al.},
    pdfsubject={Quantum Photosynthesis, Agrivoltaics, Non-Markovian Dynamics},
    pdfkeywords={agrivoltaics, quantum photosynthesis, spectral engineering, non-Markovian dynamics, renewable energy},
    colorlinks=true,
    linkcolor=blue,
    citecolor=blue,
    urlcolor=blue
}

\DeclareSIUnit{\yr}{yr}
\DeclareSIUnit{\tonne}{t}
\DeclareSIUnit{\hectare}{ha}
\DeclareSIUnit{\kWh}{kWh}
\DeclareSIUnit{\hartree}{E_h}
\DeclareSIUnit{\bohr}{a_0}

\title{Supporting Information:\\
Quantum Spectral Engineering for Enhanced Agrivoltaic Efficiency:\\
Non-Markovian Dynamics in Photosynthetic Energy Transfer}

\author{Steve Cabrel Teguia Kouam$^{2,*}$, Theodore Goumai Vedekoi$^{1}$, Jean-Pierre Tchapet Njafa$^{1}$,\\
Jean-Pierre Nguenang$^{2}$, Serge Guy Nana Engo$^{1}$\\
\\
$^{1}$Department of Physics, Faculty of Science, University of Yaound\'e I, Cameroon\\
$^{2}$Department of Physics, Faculty of Science, University of Douala, Cameroon\\
\\
*Corresponding author: \texttt{steve.teguia@univ-douala.cm}}

\date{\today}

\begin{document}

\maketitle

\tableofcontents
\newpage

%=============================================================================
\section{Environmental factor models}\label{si:environmental}
%=============================================================================

This section details the environmental factor models used to assess the applicability of quantum-optimized agrivoltaic systems across diverse geographic and climatic conditions. These models underpin the geographic simulations reported in Section~3 of the main text.

\subsection{Solar spectral modeling}

\subsubsection{Reference spectrum (AM1.5G)}

The baseline solar spectral irradiance follows the ASTM G173-03 reference standard (Air Mass 1.5 Global tilted):
\begin{equation}
J_{\mathrm{solar}}^{\mathrm{ref}}(\lambda) = J_{\mathrm{AM1.5G}}(\lambda) \quad \text{for } \lambda \in \SIrange{280}{4000}{\nano\meter},
\end{equation}
with integrated power density $P_{\mathrm{total}} = \int J_{\mathrm{solar}}^{\mathrm{ref}}(\lambda) \dd{\lambda} = \SI{1000}{\watt\per\metre\squared}$.

Photosynthetically active radiation (PAR) spans \SIrange{400}{700}{\nano\meter}, representing approximately \SI{45}{\percent} of total solar energy (\SI{450}{\watt\per\metre\squared}).

\subsubsection{Geographic variations}

Solar spectra vary by latitude due to atmospheric path length differences. We model this using Beer--Lambert attenuation:
\begin{equation}
J(\lambda, \theta_z) = J_0(\lambda) \exp\!\bigl[-\tau(\lambda) \cdot \mathrm{AM}(\theta_z)\bigr],
\end{equation}
where $J_0(\lambda)$ is the extraterrestrial spectrum, $\tau(\lambda)$ is the wavelength-dependent atmospheric optical depth, and $\mathrm{AM}(\theta_z) = 1/\cos(\theta_z)$ is the air mass for zenith angle $\theta_z$.

Representative locations include temperate (\SI{50}{\degree}N, Germany; average $\mathrm{AM} \approx \numrange{1.3}{2.9}$), subtropical (\SI{20}{\degree}N, India; $\mathrm{AM} \approx \numrange{1.1}{1.5}$), tropical (\SI{0}{\degree}, Kenya; $\mathrm{AM} \approx \numrange{1.0}{1.1}$), and desert regions (\SI{32}{\degree}N, Arizona; $\mathrm{AM} \approx \numrange{1.2}{2.2}$). We extend geographic coverage to sub-Saharan Africa with five additional sites:
\begin{itemize}
    \item Yaound\'e, Cameroon (\SI{3.87}{\degree}N; equatorial humid; GHI $\approx$ \SI{1600}{\kWh\per\metre\squared\per\yr}; AOD $\approx$ \numrange{0.3}{0.5}),
    \item N'Djamena, Chad (\SI{12.13}{\degree}N; Sahel/semi-arid; GHI $\approx$ \SI{2200}{\kWh\per\metre\squared\per\yr}; AOD $\approx$ \numrange{0.4}{0.8}),
    \item Abuja, Nigeria (\SI{9.06}{\degree}N; tropical savanna; GHI $\approx$ \SI{1900}{\kWh\per\metre\squared\per\yr}; AOD $\approx$ \numrange{0.3}{0.6}),
    \item Dakar, Senegal (\SI{14.69}{\degree}N; Sahel/coastal; GHI $\approx$ \SI{2100}{\kWh\per\metre\squared\per\yr}; AOD $\approx$ \numrange{0.3}{0.7}),
    \item Abidjan, Ivory Coast (\SI{5.36}{\degree}N; tropical humid; GHI $\approx$ \SI{1650}{\kWh\per\metre\squared\per\yr}; AOD $\approx$ \numrange{0.3}{0.5}).
\end{itemize}
These nine sites cover the primary climatic regimes where agrivoltaics are deployed or offer high development potential.

\subsubsection{Seasonal and diurnal variations}

The time-dependent solar zenith angle is calculated as:
\begin{equation}
\cos(\theta_z) = \sin(\phi)\sin(\delta) + \cos(\phi)\cos(\delta)\cos(h),
\end{equation}
where $\phi$ is latitude, $\delta$ is the solar declination (varying by $\pm \SI{23.45}{\degree}$ annually), and $h$ is the hour angle. The seasonal declination follows $\delta(d) = -\SI{23.45}{\degree} \times \cos\!\bigl[\tfrac{360}{365}(d + 10)\bigr]$ for day $d$.

\subsection{Atmospheric effects}

\subsubsection{Aerosol optical depth (AOD)}

Wavelength-dependent aerosol scattering is modeled using the \AA ngstr\"om formula:
\begin{equation}
\tau_{\mathrm{aer}}(\lambda) = \beta \lambda^{-\alpha},
\end{equation}
where $\beta$ is the turbidity coefficient (\numrange{0.05}{0.2} for clear to hazy conditions) and $\alpha$ is the \AA ngstr\"om exponent (\numrange{1.0}{1.5} for continental aerosols).

\subsubsection{Water vapor absorption}

Integrated water vapor column depth $w$ affects near-infrared transmission via:
\begin{equation}
T_{\mathrm{H_2O}}(\lambda) = \exp\!\bigl[-k_{\mathrm{H_2O}}(\lambda) \cdot w \cdot \mathrm{AM}\bigr],
\end{equation}
with absorption coefficient $k_{\mathrm{H_2O}}(\lambda)$ peaking at \SIlist{940;1100;1400}{\nano\meter}. Standard values of $w$ range from \SIrange{0.5}{1}{\centi\meter} in desert zones to \SIrange{3}{5}{\centi\meter} in tropical zones.

\subsubsection{Cloud cover and diffuse radiation}

Cloud effects are modeled using the clearness index $K_t$, defined as the ratio of measured to extraterrestrial irradiance. Sky conditions are categorized as clear ($K_t > 0.65$), partly cloudy ($K_t \in \numrange{0.35}{0.65}$), or overcast ($K_t < 0.35$). The diffuse fraction $k_d$ is determined using standard empirical correlations based on the clearness index.

\subsection{Dust and soiling effects}

\subsubsection{Particle accumulation model}

Dust accumulation on the OPV surface reduces transmission according to:
\begin{equation}
T_{\mathrm{dust}}(t) = T_0 \exp\!\bigl[-\gamma_{\mathrm{dust}} \cdot m(t)\bigr],
\end{equation}
where $m(t)$ is the accumulated mass per area (\si{\milli\gram\per\centi\metre\squared}), $T_0$ is the clean surface transmission, and $\gamma_{\mathrm{dust}}$ is the extinction coefficient (\SIrange{0.05}{0.15}{\metre\squared\per\gram}). The accumulation rate $\dd{m}/\dd{t} = r_{\mathrm{dep}} (1 - r_{\mathrm{clean}})$ varies by region, from \SIrange{0.1}{0.5}{\milli\gram\per\centi\metre\squared\per\day} in grasslands to \SIrange{3}{5}{\milli\gram\per\centi\metre\squared\per\day} in desert regions.

\subsubsection{Spectral selectivity of soiling}

Dust preferentially scatters shorter wavelengths, shifting the transmitted spectrum toward the red. The optimal transmission windows (\SIlist{750;820}{\nano\meter}) lie in spectral regions less affected by soiling than the blue--green range.

\subsection{Weather modelling}

Daily and seasonal temperature variations are modelled using sinusoidal cycles. Simulations confirm that the photosynthetic quantum advantage remains significant (\SIrange{18}{26}{\percent}) across the physiological temperature range (\SIrange{280}{310}{\kelvin}), supporting year-round viability across diverse climates.

\subsection{Integration with quantum simulations}

Environmental factors modify the effective incident spectrum through sampling over time (hourly resolution), geography (nine representative sites across four continents), and weather/soiling states. Results confirm that quantum advantages persist under realistic variability, including elevated AOD conditions encountered at Sahel sites, with additional potential for carbon sequestration of \SIrange{0.5}{1.0}{\tonne\,\text{CO}_2\per\hectare\per\yr}.

%=============================================================================
\section{Biodegradability assessment}\label{si:biodegradability}
%=============================================================================

Sustainable deployment of agrivoltaic OPV materials requires low environmental impact at end of life. This section describes the computational quantum chemistry framework used to assess enzymatic degradation susceptibility of candidate molecules.

\subsection{Fukui function analysis}

The Fukui function quantifies local reactivity of molecular sites toward nucleophilic or electrophilic attack, predicting enzymatic degradation pathways.

\subsubsection{Theoretical framework}

Fukui functions are defined as functional derivatives of electron density with respect to electron number $N$ at constant external potential $v(\vec{r})$:
\begin{align}
f^+(\vec{r}) &= \left(\frac{\partial \rho(\vec{r})}{\partial N}\right)_{v(\vec{r})}^+ \approx \rho_{N+1}(\vec{r}) - \rho_N(\vec{r}), \quad &\text{(electrophilic attack),}\label{eq:fukui_plus}\\
f^-(\vec{r}) &= \left(\frac{\partial \rho(\vec{r})}{\partial N}\right)_{v(\vec{r})}^- \approx \rho_N(\vec{r}) - \rho_{N-1}(\vec{r}), \quad &\text{(nucleophilic attack),}\label{eq:fukui_minus}\\
f^0(\vec{r}) &= \tfrac{1}{2}\bigl[f^+(\vec{r}) + f^-(\vec{r})\bigr], \quad &\text{(radical attack),}\label{eq:fukui_zero}
\end{align}
where $\rho_N(\vec{r})$, $\rho_{N+1}(\vec{r})$, and $\rho_{N-1}(\vec{r})$ denote the electron densities of the neutral, anionic, and cationic species, respectively. Higher Fukui values indicate more reactive sites susceptible to enzymatic attack.

\subsubsection{Computational details}

Density functional theory (DFT) calculations employ the B3LYP hybrid exchange-correlation functional with a 6-31G(d,p) double-zeta polarised basis set, using Gaussian~16 or ORCA~5.0. Convergence criteria are set to \SI{e-8}{\hartree} for the SCF procedure and \SI{e-5}{\hartree\per\bohr} for geometry optimisation.

For each candidate OPV molecule, the protocol proceeds as follows:
\begin{enumerate}
    \item Optimize the ground-state geometry ($N$ electrons).
    \item Perform a single-point calculation for $N+1$ electrons (anion).
    \item Perform a single-point calculation for $N-1$ electrons (cation).
    \item Compute Fukui functions on the molecular grid.
\end{enumerate}

\subsection{Global reactivity descriptors}

\subsubsection{Chemical hardness and softness}

Chemical hardness $\eta$ quantifies resistance to electron density redistribution:
\begin{equation}
\eta = \frac{1}{2}(I - A) = \frac{1}{2}(\varepsilon_{\mathrm{LUMO}} - \varepsilon_{\mathrm{HOMO}}).
\end{equation}
Chemical softness $S = 1/\eta$; softer molecules are more reactive and hence more biodegradable.

\subsubsection{Electrophilicity index}

The global electrophilicity $\omega$ is defined as:
\begin{equation}
\omega = \frac{\mu^2}{2\eta} = \frac{(I + A)^2}{8(I - A)},
\end{equation}
where $\mu = -(I + A)/2$ is the chemical potential, $I$ is the ionization energy, and $A$ is the electron affinity.

\subsubsection{Nucleophilicity index}

Using Koopmans' theorem, the nucleophilicity index is:
\begin{equation}
\mathcal{N} = \varepsilon_{\mathrm{HOMO}} - \varepsilon_{\mathrm{HOMO}}^{\mathrm{ref}},
\end{equation}
referenced to tetracyanoethylene (TCNE) as a strong electrophile.

\subsection{Enzymatic degradation pathways}

\subsubsection{Hydrolase attack (ester linkages)}

Ester bonds, common in biodegradable polymers, are cleaved by hydrolases. The Fukui nucleophilic index $f^-$ at the carbonyl carbon predicts susceptibility:
\begin{equation}
k_{\mathrm{hydrolysis}} \propto f^-(\mathrm{C}_{\mathrm{carbonyl}}) \times S.
\end{equation}
A target of $f^- > 0.05$ corresponds to rapid biodegradation (${<}\,1$~year).

\subsubsection{Oxidase attack (aromatic rings)}

Cytochrome P450 enzymes oxidize aromatic systems. High $f^+$ at aromatic carbons indicates vulnerability:
\begin{equation}
k_{\mathrm{oxidation}} \propto \max\!\bigl[f^+(\mathrm{C}_{\mathrm{aromatic}})\bigr] \times \omega.
\end{equation}

\subsubsection{Bond dissociation energies}

Weakest bonds constitute preferential degradation sites:
\begin{equation}
\mathrm{BDE}(\mathrm{A{-}B}) = E(\mathrm{A}{\cdot}) + E(\mathrm{B}{\cdot}) - E(\mathrm{A{-}B}).
\end{equation}
Bonds with $\mathrm{BDE} < \SI{300}{\kilo\joule\per\mole}$ are readily cleaved by enzymatic radicals.

\subsection{Biodegradability index}

We define a composite biodegradability score:
\begin{equation}
B_{\mathrm{index}} = w_1 S + w_2 \langle f^- \rangle + w_3 N_{\mathrm{ester}} + w_4 (400 - \mathrm{BDE}_{\mathrm{min}}),
\end{equation}
where $S$ is the global softness, $\langle f^- \rangle$ is the average nucleophilic Fukui function, $N_{\mathrm{ester}}$ is the number of hydrolyzable ester linkages, $\mathrm{BDE}_{\mathrm{min}}$ is the weakest bond dissociation energy in \si{\kilo\joule\per\mole}, and the weights are $w_1 = 0.3$, $w_2 = 0.3$, $w_3 = 0.2$, $w_4 = 0.2$.

The resulting classification scheme is:
\begin{itemize}
    \item $B_{\mathrm{index}} > 70$: Highly biodegradable (${<}\,6$ months).
    \item $50 < B_{\mathrm{index}} < 70$: Moderately biodegradable (6--18 months).
    \item $30 < B_{\mathrm{index}} < 50$: Slowly biodegradable (1.5--5 years).
    \item $B_{\mathrm{index}} < 30$: Recalcitrant (${>}\,5$ years).
\end{itemize}
Two candidate non-fullerene acceptor molecules were evaluated for quantum-optimized agrivoltaic systems. \textbf{Molecule~A (PM6 derivative)} exhibits high biodegradability ($B_{\mathrm{index}} = 72$) due to four hydrolyzable ester linkages ($f^-_{\mathrm{max}} = 0.08$ at the carbonyl carbon) and a low minimum BDE of \SI{285}{\kilo\joule\per\mole} at the thiophene--ester bond. \textbf{Molecule~B (Y6-BO derivative)} is moderately biodegradable ($B_{\mathrm{index}} = 58$) with two ester linkages and a minimum BDE of \SI{310}{\kilo\joule\per\mole}. Both candidates achieve ${>}\,\SI{15}{\percent}$ PCE in semi-transparent configurations while ensuring environmental compatibility.

The eco-design assessment framework integrates multiple sustainability metrics:
\begin{enumerate}
    \item \textbf{Biodegradability Analysis.} Using quantum reactivity descriptors to predict enzymatic degradation pathways and timeframes
    \item \textbf{Life Cycle Assessment (LCA).} Comprehensive evaluation of environmental impacts from material synthesis through end-of-life disposal
    \item \textbf{Performance Metrics.} Power conversion efficiency, ETR enhancement, and operational lifetime
    \item \textbf{Environmental Compatibility.} Assessment of impacts on soil health, water resources, and local ecosystems
\end{enumerate}
The integrated eco-design score combines these metrics according to:
\begin{equation}
\eta_{\mathrm{eco}} = 0.4 \cdot \eta_{\mathrm{biodeg}} + 0.3 \cdot \eta_{\mathrm{PCE}} + 0.3 \cdot \eta_{\mathrm{LCA}},
\end{equation}
where $\eta_{\mathrm{biodeg}}$, $\eta_{\mathrm{PCE}}$, and $\eta_{\mathrm{LCA}}$ are normalized efficiency factors for biodegradability, power conversion efficiency, and life cycle impact respectively. Our analysis yields an eco-design score of $\eta_{\mathrm{eco}} = 0.78$ for the optimized materials, indicating good overall sustainability performance.

%=============================================================================
\section{Extended validation data}\label{si:validation}
%=============================================================================

This section documents the 12 validation tests referenced in Section~3 of the main text. Each test is assigned a pass/fail criterion; results are summarised in \Cref{tab:validation}.

\subsection{FMO complex Hamiltonian}

The FMO complex consists of seven bacteriochlorophyll-a (BChl-a) chromophores. The system Hamiltonian is:
\begin{equation}
H_{\mathrm{sys}} = \sum_{n=1}^{7} \epsilon_n \dyad{n} + \sum_{n \neq m} J_{nm} \dyad{n}{m}.
\end{equation}
\Cref{tab:fmo_hamiltonian} provides the complete parameterization based on X-ray crystallographic data and spectroscopic measurements.

\begin{table}[ht]
\centering
\caption{\textbf{FMO complex Hamiltonian parameters.} Site energies ($\epsilon_n$, diagonal) and electronic couplings ($J_{nm}$, off-diagonal) in \si{\per\cm} from the Adolphs \& Renger parameterization. These parameters reproduce experimental spectral features and provide the basis for non-Markovian transport simulations.}
\label{tab:fmo_hamiltonian}
\begin{tabular}{lccccccc}
\toprule
& \textbf{Site 1} & \textbf{Site 2} & \textbf{Site 3} & \textbf{Site 4} & \textbf{Site 5} & \textbf{Site 6} & \textbf{Site 7} \\
\midrule
$\epsilon_n$ (\si{\per\cm}) & 12410 & 12530 & 12210 & 12320 & 12480 & 12630 & 12440 \\
\midrule
Site 1 & --- & $-$87.7 & 5.5 & $-$5.9 & 6.7 & $-$13.7 & $-$9.9 \\
Site 2 & $-$87.7 & --- & 30.8 & 8.2 & 0.7 & 11.8 & 4.3 \\
Site 3 & 5.5 & 30.8 & --- & $-$53.5 & $-$2.2 & $-$9.6 & 6.0 \\
Site 4 & $-$5.9 & 8.2 & $-$53.5 & --- & $-$70.7 & $-$17.0 & $-$63.3 \\
Site 5 & 6.7 & 0.7 & $-$2.2 & $-$70.7 & --- & 81.1 & $-$1.3 \\
Site 6 & $-$13.7 & 11.8 & $-$9.6 & $-$17.0 & 81.1 & --- & 39.7 \\
Site 7 & $-$9.9 & 4.3 & 6.0 & $-$63.3 & $-$1.3 & 39.7 & --- \\
\bottomrule
\multicolumn{8}{l}{\scriptsize Source: Adolphs \& Renger (2006). Site 1 is the reaction-centre-proximal BChl.} \\
\end{tabular}
\end{table}

Key features of the Hamiltonian include: site energies spanning \SI{420}{\per\cm} ($k_{\mathrm{B}}T$ at \SI{295}{\kelvin} $\approx$ \SI{205}{\per\cm}), placing the system in the mixed quantum--classical regime; a strongest coupling of \SI{87.7}{\per\cm} between sites 1 and 2; and a funnelling network directing excitations along the pathway 1,\,2,\,3 $\to$ 4,\,7 $\to$ 5,\,6 $\to$ reaction centre.

\subsection{Convergence tests (tests 1--4)}

\subsubsection{Test 1: HEOM benchmark}

We validated adHOPS against numerically exact HEOM for a 3-site model system (site energies \SIlist{12000;12100;12200}{\per\cm}; Drude bath $\lambda = \SI{35}{\per\cm}$ at \SI{295}{\kelvin}). The maximum deviation is \SI{1.8}{\percent} at early times ($t < \SI{50}{\femto\second}$), with an average deviation of \SI{0.6}{\percent} over a \SI{1000}{\femto\second} window, passing the \SI{2}{\percent} acceptance threshold.

\subsubsection{Test 2: Matsubara cutoff convergence}

Varying $N_{\mathrm{Mat}}$ for the FMO system at \SI{295}{\kelvin} confirms that $N_{\mathrm{Mat}} = 10$ achieves convergence, with observables stable to within \SI{0.3}{\percent} for $N_{\mathrm{Mat}} \geq 10$. Production runs use $N_{\mathrm{Mat}} = 12$ to ensure a sufficient safety margin.

\subsubsection{Test 3: Time step convergence}

Numerical integration accuracy was verified by comparing results for $\Delta t \in \SIlist{0.5;1.0;2.0}{\femto\second}$. Differences are negligible (\SI{0.08}{\percent} between \num{0.5} and \SI{1.0}{\femto\second}); production runs use $\Delta t = \SI{1.0}{\femto\second}$.

\subsubsection{Test 4: Hierarchy truncation convergence}

Observables vary by less than \SI{0.8}{\percent} for truncation thresholds $\epsilon_{\mathrm{trunc}} \in \{10^{-7}, 10^{-8}, 10^{-9}\}$. Production runs use $\epsilon_{\mathrm{trunc}} = 10^{-8}$ to balance computational cost and precision.

\subsection{Physical consistency tests (tests 5--8)}

\subsubsection{Test 5: Trace preservation}

Density matrix normalization is maintained with a maximum deviation of $5 \times 10^{-13}$ (machine precision limit) throughout \SI{100}{\pico\second} simulations, with no systematic drift.

\subsubsection{Test 6: Positivity}

The density matrix remains positive semidefinite within numerical precision. The minimum eigenvalue observed is $-2.1 \times 10^{-11}$, consistent with floating-point noise.

\subsubsection{Test 7: Energy conservation}

In the zero-coupling limit ($\lambda = 0$), system energy drift is limited to \SI{0.08}{\percent} over \SI{100}{\pico\second} with no systematic trend, confirming numerical conservation of energy.

\subsubsection{Test 8: Detailed balance}

Long-time population limits match the Boltzmann distribution with a maximum deviation of \SI{0.6}{\percent} across the physiological temperature range (\SIrange{280}{310}{\kelvin}), confirming thermodynamic consistency.

\subsection{Environmental robustness tests (tests 9--12)}

\subsubsection{Test 9: Temperature sensitivity}

Simulations across $T \in \SIlist{285;295;305}{\kelvin}$ demonstrate that the quantum advantage remains within \SI{15}{\percent} of the \SI{295}{\kelvin} reference value despite thermal fluctuations.

\subsubsection{Test 10: Static disorder}

Gaussian disorder added to site energies ($\sigma = \SI{50}{\per\cm}$) results in a \SI{20}{\percent} mean reduction in quantum advantage relative to the disorder-free case, but a significant enhancement persists across the ensemble.

\subsubsection{Test 11: Bath parameter variations}

Varying spectral density parameters ($\lambda$, $\gamma$, $\omega_k$) by $\pm \SI{20}{\percent}$ preserves the qualitative features of vibronic resonance, with peak shifts restricted to ${<}\,\SI{5}{\nano\meter}$, confirming the robustness of engineering predictions.

\subsubsection{Test 12: Markovian limit recovery}

At high temperature ($T = \SI{500}{\kelvin}$), adHOPS converges to the Redfield theory result (within \SI{1.8}{\percent} deviation) as the quantum advantage nearly vanishes, confirming correct asymptotic behaviour in the Markovian regime.

\subsection{Summary of validation results}

\begin{table}[ht]
\centering
\caption{\textbf{Summary of validation results.} The 12-test suite covers convergence, physical consistency, and environmental robustness. All tests pass established acceptance thresholds.}
\label{tab:validation}
\begin{tabular}{llcc}
\toprule
\textbf{Category} & \textbf{Test} & \textbf{Criterion} & \textbf{Result} \\
\midrule
\multirow{4}{*}{Convergence}
& HEOM benchmark & ${<}\,\SI{2}{\percent}$ deviation & \SI{1.8}{\percent} \checkmark \\
& Matsubara cutoff & ${<}\,\SI{0.5}{\percent}$ change & \SI{0.3}{\percent} \checkmark \\
& Time step & Invariance & ${<}\,\SI{0.1}{\percent}$ \checkmark \\
& Hierarchy truncation & ${<}\,\SI{1}{\percent}$ variation & \SI{0.8}{\percent} \checkmark \\
\midrule
\multirow{4}{*}{Physical}
& Trace preservation & ${<}\,10^{-12}$ & $5 \times 10^{-13}$ \checkmark \\
& Positivity & $\lambda_i > -10^{-10}$ & $-2 \times 10^{-11}$ \checkmark \\
& Energy conservation & ${<}\,\SI{0.1}{\percent}$ drift & \SI{0.08}{\percent} \checkmark \\
& Detailed balance & Match Boltzmann & \SI{0.6}{\percent} dev.\ \checkmark \\
\midrule
\multirow{4}{*}{Robustness}
& Temperature ($\pm \SI{10}{\kelvin}$) & Within \SI{15}{\percent} & \SIrange{12}{16}{\percent} \checkmark \\
& Static disorder & Persists & \SI{20}{\percent} reduction \checkmark \\
& Bath parameters & Qualitative & Features preserved \checkmark \\
& Markovian limit & Redfield agreement & \SI{1.8}{\percent} dev.\ \checkmark \\
\midrule
\multicolumn{3}{l}{\textbf{Overall success rate}} & \textbf{12/12 (100\%)} \\
\bottomrule
\end{tabular}
\end{table}

%=============================================================================
\section{Complete FMO parameter sets}\label{si:parameters}
%=============================================================================

\subsection{Site energies (Adolphs \& Renger, 2006)}

Room-temperature (\SI{295}{\kelvin}) site energies for the FMO monomer are listed in \Cref{tab:site_energies}.

\begin{table}[ht]
\centering
\caption{\textbf{FMO site energies.} Transition energies and corresponding wavelengths for the seven BChl-a chromophores.}
\label{tab:site_energies}
\begin{tabular}{ccc}
\toprule
\textbf{Site} & \textbf{Energy (\si{\per\cm})} & \textbf{Wavelength (\si{\nano\meter})} \\
\midrule
1 & 12\,410 & 806 \\
2 & 12\,530 & 798 \\
3 & 12\,210 & 819 \\
4 & 12\,320 & 812 \\
5 & 12\,480 & 801 \\
6 & 12\,630 & 792 \\
7 & 12\,440 & 804 \\
\bottomrule
\end{tabular}
\end{table}

\subsection{Electronic couplings}

The coupling matrix $J_{nm}$ (\si{\per\cm}, symmetric) from the Adolphs \& Renger parameterisation is:
\begin{equation}
\bm{J} = \begin{pmatrix}
0 & -87.7 & 5.5 & -5.9 & 6.7 & -13.7 & -9.9 \\
-87.7 & 0 & 30.8 & 8.2 & 0.7 & 11.8 & 4.3 \\
5.5 & 30.8 & 0 & -53.5 & -2.2 & -9.6 & 6.0 \\
-5.9 & 8.2 & -53.5 & 0 & -70.7 & -17.0 & -63.3 \\
6.7 & 0.7 & -2.2 & -70.7 & 0 & 81.1 & -1.3 \\
-13.7 & 11.8 & -9.6 & -17.0 & 81.1 & 0 & 39.7 \\
-9.9 & 4.3 & 6.0 & -63.3 & -1.3 & 39.7 & 0
\end{pmatrix}.
\end{equation}

\subsection{Spectral density parameters}

The \textbf{overdamped (Drude--Lorentz) component} has reorganization energy $\lambda_D = \SI{35}{\per\cm}$ and cutoff frequency $\gamma_D = \SI{50}{\per\cm}$ (corresponding to a \SI{200}{\femto\second} correlation time).

The \textbf{underdamped (vibronic) modes} are parameterized in \Cref{tab:vibronic_modes}.

\begin{table}[ht]
\centering
\caption{\textbf{Vibronic mode parameters.} Reorganization energies are $\lambda_k = S_k \hbar \omega_k$.}
\label{tab:vibronic_modes}
\begin{tabular}{cccc}
\toprule
\textbf{Mode} & \textbf{Frequency $\omega_k$ (\si{\per\cm})} & \textbf{Huang--Rhys $S_k$} & \textbf{Damping $\gamma_k$ (\si{\per\cm})} \\
\midrule
1 & 150 & 0.05 & 10 \\
2 & 200 & 0.02 & 10 \\
3 & 575 & 0.01 & 20 \\
4 & 1185 & 0.005 & 30 \\
\bottomrule
\end{tabular}
\end{table}

The total reorganization energy is $\lambda_{\mathrm{total}} = \lambda_D + \sum_k \lambda_k \approx \SI{50}{\per\cm}$.

%=============================================================================
\section{Process Tensor-HOPS and spectrally bundled dissipators framework}\label{si:pthops}
%=============================================================================

Simulations employ the Process Tensor HOPS (PT-HOPS) and Spectrally Bundled Dissipators (SBD) methods to approximate the non-Markovian open quantum system dynamics. This section details the methods and benchmarks their performance against HEOM and Redfield theory.

\begin{table}[ht]
\centering
\caption{\textbf{Computational performance comparison.} All benchmarks are for \SI{1}{\pico\second} of FMO dynamics at \SI{295}{\kelvin} with a Drude + vibronic bath on an AMD EPYC 7542 processor (32 cores at \SI{2.9}{\giga\hertz}). Redfield (Markovian) results are shown for comparison but fail to capture coherence effects. PT-HOPS achieves near-HEOM accuracy with a \num{10}$\times$ speedup, enabling large-scale simulations ($N > 20$ sites) intractable for HEOM. SBD provides additional computational advantages for systems with $N > 100$ chromophores.}
\label{tab:computational_benchmarks}
\begin{tabular}{lcccc}
\toprule
\textbf{Method} & \textbf{System size} & \textbf{Wall time} & \textbf{Memory} & \textbf{Accuracy} \\
 & \textbf{(sites)} & \textbf{(\si{\hour})} & \textbf{(\si{\giga\byte})} & \textbf{(vs HEOM)} \\
\midrule
HEOM (reference) & 7 & 38.2 & 12.4 & Exact \\
PT-HOPS & 7 & 3.8 & 2.1 & ${<}\,\SI{2}{\percent}$ deviation \\
SBD & 7 & 3.5 & 1.9 & ${<}\,\SI{3}{\percent}$ deviation \\
Redfield (Markov) & 7 & 0.3 & 0.5 & \SI{18}{\percent} deviation$^*$ \\
\midrule
PT-HOPS & 24 & 12.5 & 6.3 & N/A$^\dagger$ \\
PT-HOPS & 100 & 48.7 & 22.1 & N/A$^\dagger$ \\
SBD & 100 & 38.2 & 18.5 & N/A$^\dagger$ \\
SBD & 500 & 156.4 & 89.2 & N/A$^\dagger$ \\
SBD & 1000 & 320.1 & 185.7 & N/A$^\dagger$ \\
\bottomrule
\multicolumn{5}{l}{\scriptsize $^*$Markovian methods fail to capture non-Markovian coherence effects.} \\
\multicolumn{5}{l}{\scriptsize $^\dagger$HEOM computationally intractable for $N > 10$ sites.} \\
\end{tabular}
\end{table}

As shown in \Cref{tab:computational_benchmarks}, PT-HOPS achieves a \num{10}-fold speedup over HEOM with ${<}\,\SI{2}{\percent}$ accuracy and exhibits near-linear scaling with system size for localised excitons, enabling simulations of complete photosynthetic antenna complexes (100+ chromophores) with non-Markovian accuracy. The SBD approach provides additional computational advantages for very large systems ($N > 500$ chromophores), maintaining ${<}\,\SI{5}{\percent}$ accuracy relative to PT-HOPS while enabling simulation of systems approaching the size of complete chloroplasts.

\subsection{Padé decomposition of the bath correlation function}

The bath correlation function $C(t)$ is decomposed via Pad\'e approximation into exponentially decaying terms plus a residual non-exponential component:
\begin{equation}
K_{\mathrm{PT}}(t,s) = \sum_{k} g_k(t)\, f_k(s)\, \mathrm{e}^{-\lambda_k |t-s|} + K_{\mathrm{non\text{-}exp}}(t,s),
\end{equation}
where $g_k(t)$ and $f_k(s)$ are effective coupling functions, $\lambda_k$ are decay rates, and $K_{\mathrm{non\text{-}exp}}(t,s)$ captures residual memory effects beyond the exponential approximation.

\subsection{Spectrally bundled dissipators (SBD) formalism}

For large chromophore systems, the SBD approach provides additional computational efficiency by bundling dissipative processes according to their spectral characteristics:
\begin{equation}
\mathcal{L}_{\mathrm{SBD}}[\rho] = \sum_{\alpha} p_{\alpha}(t) \mathcal{D}_{\alpha}[\rho],
\end{equation}
where $\mathcal{D}_{\alpha}[\rho] = L_{\alpha} \rho L_{\alpha}^{\dagger} - \frac{1}{2}\{L_{\alpha}^{\dagger}L_{\alpha}, \rho\}$ represents the dissipator for bundle $\alpha$ with time-dependent probability $p_{\alpha}(t)$. The bundling strategy groups similar dissipative processes, reducing the number of terms in the quantum master equation while preserving the essential physics of non-Markovian dynamics.

\subsection{Thermal regime validity}

For simulations at physiological temperatures ($T = \SI{295}{\kelvin}$), the high-temperature approximation is valid ($k_B T \gg \hbar\gamma$), and explicit Matsubara reservoir terms are negligible. The simulation uses the standard Drude-Lorentz spectral density, maintaining computational efficiency while capturing thermal effects accurately. This efficiency enables high-throughput screening of OPV transmission functions and disorder ensembles essential for realistic agrivoltaic design optimisation.

%=============================================================================
\section{Full chloroplast modeling and hierarchical coarse-graining}\label{si:chloroplast_modeling}
%=============================================================================

The ultimate goal of quantum-enhanced agrivoltaics research is to understand and optimize quantum effects in complete biological systems. This section outlines the hierarchical coarse-graining approach necessary to model complete chloroplasts while preserving essential quantum dynamics.

\subsection{Multi-scale modeling roadmap}

Complete chloroplast modeling requires bridging multiple scales of biological organization:
\begin{enumerate}
    \item \textbf{Molecular scale} (\SI{1}{\nano\metre}): Individual chromophore dynamics with full quantum mechanical treatment (10-100 chromophores)
    \item \textbf{Supramolecular scale} (\SIrange{1}{10}{\nano\metre}): Antenna complexes with reduced quantum models (100-1000 chromophores)
    \item \textbf{Organelle scale} (\SIrange{1}{10}{\micro\metre}): Complete chloroplast with coarse-grained models (1000+ chromophores)
    \item \textbf{Cellular scale} (\SIrange{10}{100}{\micro\metre}): Integration with cellular metabolism and physiology
\end{enumerate}
Each scale requires different modeling approaches and computational methods, with the quantum description becoming increasingly approximate as system size increases.

\subsection{Hierarchical coarse-graining methodology}

The hierarchical approach preserves quantum effects of interest while maintaining computational tractability:
\begin{equation}
\mathcal{H}_{\mathrm{eff}}^{(n)} = \mathcal{P}_n \mathcal{H}_{\mathrm{full}} \mathcal{P}_n^{\dagger},
\end{equation}
where $\mathcal{P}_n$ is the projection operator from the full Hilbert space to the effective space at scale $n$, and $\mathcal{H}_{\mathrm{eff}}^{(n)}$ is the effective Hamiltonian at that scale.

This approach enables investigation of how quantum coherence effects observed at the molecular scale propagate to larger biological structures, potentially persisting in modified form at the supramolecular and organelle levels.

%=============================================================================
\section{Additional figures}\label{si:figures}
%=============================================================================

\subsection{Figure S1: Spectral density components}

\begin{figure}[ht]
\centering
\includegraphics[width=0.85\textwidth]{Graphics/Spectral_Density_Components_for_FMO_Environment.pdf}
\caption{\textbf{Spectral density of the FMO environmental bath.} The environment is modeled with an overdamped Drude--Lorentz component (blue) and four underdamped vibronic modes (orange peaks). The \SI{575}{\per\cm} mode is the primary target for quantum-engineered spectral filtering.}
\label{fig:SI_spectral_density}
\end{figure}

\subsection{Figure S2: Quantum metrics evolution}

\begin{figure}[ht]
\centering
\includegraphics[width=0.95\textwidth]{Graphics/Quantum_Metrics_Evolution.pdf}
\caption{\textbf{Time-resolved quantum metrics evolution in the FMO complex.} (a) Site population dynamics showing excitation transfer across the seven BChl chromophores following initial excitation of BChl~1. (b) $l_1$-norm coherence evolution. (c) State purity $\Tr[\bm{\rho}^2]$ and von Neumann entropy $S$ illustrating the coherent-to-incoherent transition at \SI{295}{\kelvin}. (d) Normalised Quantum Fisher Information tracking the metrological advantage during the coherent transport window.}
\label{fig:SI_quantum_metrics}
\end{figure}

\subsection{Figure S3: Global reactivity indices}

\begin{figure}[ht]
\centering
\includegraphics[width=0.85\textwidth]{Graphics/Global_Reactivity_Indices.pdf}
\caption{\textbf{Molecular reactivity and biodegradability indices.} Fukui functions $f^+$ and $f^0$ identifying reactive sites on OPV donor--acceptor candidates, dual descriptor for selectivity analysis, and frontier molecular orbital energies.}
\label{fig:SI_biodegradability}
\end{figure}

\subsection{Figure S4: PAR transmission (clean vs dusty)}

\begin{figure}[ht]
\centering
\includegraphics[width=0.85\textwidth]{Graphics/PAR_Transmission__Clean_vs_Dusty_Conditions.pdf}
\caption{\textbf{Spectral transmission under surface soiling.} Effective PAR transmission through OPV panels under clean (solid) and dusty (dashed) conditions. The dual-band resonance windows at \SIlist{750;820}{\nano\meter} maintain their spectral selectivity despite intensity reduction from particle scattering.}
\label{fig:SI_par_transmission}
\end{figure}

\subsection{Figure S5: Response functions}

\begin{figure}[ht]
\centering
\includegraphics[width=0.85\textwidth]{Graphics/Response_Functions__OPV_vs_PSU.pdf}
\caption{\textbf{Spectral response partitioning.} Comparison of OPV electrical absorption and photosynthetic spectral unit (PSU) biological response. The optimized dual-band window (\SIlist{750;820}{\nano\meter}) targets excitonic transitions to maximise quantum transport benefits with minimal impact on electrical harvesting.}
\label{fig:SI_response_functions}
\end{figure}

\subsection{Figure S6: Geographic climate maps}

\begin{figure}[ht]
\centering
\includegraphics[width=0.9\textwidth]{Graphics/fLatitude__lat__u00b0__Month__month.pdf}
\caption{\textbf{Annual variation of ETR enhancement by latitude.} Heatmap showing the geographic and seasonal distribution of the quantum advantage. Strategic spectral engineering provides persistent ETR enhancements of \SIrange{18}{28}{\percent} globally, with optimal performance when ambient temperatures align with the \SI{295}{\kelvin} vibronic resonance peak.}
\label{fig:SI_climate_map}
\end{figure}

\subsection{Figure S7: ETR uncertainty distributions}

\begin{figure}[ht]
\centering
\includegraphics[width=0.85\textwidth]{Graphics/ETR_Uncertainty_Distribution.pdf}
\caption{\textbf{Statistical distribution of quantum transport enhancement.} Distribution of ETR enhancement across 100 independent realizations of static energetic disorder ($\sigma = \SI{50}{\per\cm}$). The narrow distribution (mean: \SI{20}{\percent}, standard deviation: \SI{4}{\percent}) confirms that the quantum advantage is robust and independent of specific site-energy configurations.}
\label{fig:SI_etr_uncertainty}
\end{figure}

\subsection{Figure S8: Sub-Saharan Africa ETR enhancement}

\begin{figure}[ht]
\centering
\includegraphics[width=0.9\textwidth]{Graphics/ETR_Under_Environmental_Effects.pdf}
\caption{\textbf{Agrivoltaic quantum advantage in sub-Saharan Africa.} Environmental robustness data (reproduced from Fig.~4 of the main text for reader convenience) highlighting the sub-Saharan perspective: monthly ETR enhancement heatmap (left) and annual mean metrics (right) for five sites spanning three climate zones---Yaound\'e and Abidjan (equatorial), Abuja (tropical savanna), and Dakar and N'Djamena (Sahel). Persistent enhancements of \SIrange{18}{24}{\percent} demonstrate the regional potential for quantum-enhanced agrivoltaic production.}
\label{fig:SI_subsaharan}
\end{figure}

%=============================================================================
\bibliographystyle{unsrt}
\bibliography{references}
%=============================================================================

\end{document}
