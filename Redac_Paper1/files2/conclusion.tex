% Conclusion Section - EES Version
% Quantum Spectral Engineering for Enhanced Agrivoltaic Efficiency

\section{Conclusion}\label{sec:Conclusion}

We have demonstrated that spectral bath engineering---strategic spectral filtering through semi-transparent OPV panels targeting vibronic resonances---enhances the electron transport rate by up to \num{25}\% relative to Markovian models under matched photon flux. This enhancement, validated by a 12-test numerical suite including HEOM benchmarking ($<\num{2}\%$ deviation), originates from non-Markovian quantum coherence effects that extend coherence lifetimes and increase exciton delocalization at physiological temperature.

Pareto frontier analysis identifies practical OPV configurations achieving \numrange{16}{18}\% power conversion efficiency with \numrange{15}{20}\% ETR enhancement through dual-band transmission at \numlist{750;820}\,nm. Economic modelling estimates USD~\numrange{470}{3000}\,ha$^{-1}$\,yr$^{-1}$ additional revenue depending on crop value, with positive returns across all climate zones studied. These results are robust under realistic conditions: physiological temperature (\num{295}\,K), static disorder ($\sigma = \num{50}$\,\si{\per\cm}), and seasonal/geographic variation.

These predictions are experimentally testable: ultrafast spectroscopy should detect coherence lifetime extensions under filtered illumination, while field trials should demonstrate \numrange{10}{18}\% crop productivity improvements at equivalent PV coverage. We have provided quantitative materials specifications to guide OPV development.

Three research directions are most pressing: (1)~expansion to complete photosynthetic network modelling incorporating both light-dependent and carbon fixation reactions; (2)~experimental validation across diverse crop species and climatic zones; and (3)~development of adaptive filtering technologies that respond to environmental conditions in real time. The spectral bath engineering principle---engineering the spectral environment to maximise quantum resource utilisation---extends beyond agrivoltaics to artificial photosynthesis, quantum-enhanced solar cells, and bio-inspired molecular electronics.
