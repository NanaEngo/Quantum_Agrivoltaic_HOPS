% Theory and Methods Section - EES Version
% Quantum Spectral Engineering for Enhanced Agrivoltaic Efficiency

\section{Theory and Methods}\label{sec:Theory}

\subsection{Open Quantum System Framework}

We treat the photosynthetic unit as an open quantum system coupled to both a structured vibrational environment (protein-solvent modes and intramolecular vibrations) and a spectrally filtered photon bath determined by the overlying OPV transmission function. The dynamics of the reduced density matrix $\bm{\rho}(t)$ for the excitonic system is governed by the quantum master equation:
\begin{equation}\label{eq:master_eq}
\frac{d\bm{\rho}(t)}{dt} = \mathcal{L}(t)\bm{\rho}(t) = -\frac{i}{\hbar}[\hat{H}_S, \bm{\rho}(t)] + \mathcal{D}[\bm{\rho}(t)]
\end{equation}
where $\hat{H}_S$ is the system Hamiltonian and $\mathcal{D}[\bm{\rho}(t)]$ represents the dissipative terms due to system-bath interactions. For agrivoltaic applications, the key innovation is recognizing that $\mathcal{D}[\bm{\rho}(t)]$ can be engineered through strategic control of the incident spectral density via $T(\omega)$, transforming passive light harvesting into active quantum state engineering.

The electronic Hamiltonian for the excitonic system is expressed as:
\begin{equation}\label{eq:excitonic_hamiltonian}
\hat{H}_{\rm el} = \sum_n \varepsilon_n \dyad{n} + \sum_{n \neq m} J_{nm} \dyad{n}{m}
\end{equation}
where $\varepsilon_n$ represents the site energy of chromophore $n$, and $J_{nm}$ is the electronic coupling between chromophores $n$ and $m$. The interplay between site energy distribution and electronic couplings determines the exciton delocalization landscape, which is sensitively modulated by the spectral properties of the driving light field.

\subsection{System-Bath Interaction and Spectral Density Engineering}

The interaction with the protein-solvent environment and photon bath is modeled using a system-bath Hamiltonian:
\begin{equation}\label{eq:system_bath_hamiltonian}
\hat{H} = \hat{H}_S + \hat{H}_B + \hat{H}_{SB}
\end{equation}
where $\hat{H}_B$ describes the bath degrees of freedom, and $\hat{H}_{SB}$ represents the system-bath interaction.

The spectral density function $J(\omega)$ characterizes the coupling between the system and bath modes. For the protein-solvent environment, we employ a composite spectral density:
\begin{equation}\label{eq:spectral_density}
J_{\rm bath}(\omega) = \frac{2\lambda\gamma\omega}{\omega^2 + \gamma^2} + \sum_k \frac{2\lambda_k\omega_k^2\gamma_k}{(\omega-\omega_k)^2 + \gamma_k^2}
\end{equation}
where the first term represents overdamped protein-solvent modes (reorganization energy $\lambda$, cutoff frequency $\gamma$), and the second term represents underdamped intramolecular vibrational modes (reorganization energies $\lambda_k$, frequencies $\omega_k$, damping rates $\gamma_k$).

The critical innovation for agrivoltaic optimization is spectral density engineering of the photon bath. The effective incident spectral density becomes:
\begin{equation}\label{eq:filtered_spectral_density}
J_{\rm plant}(\omega) = T(\omega) \times J_{\rm solar}(\omega)
\end{equation}
where $T(\omega)$ is the OPV transmission function and $J_{\rm solar}(\omega)$ is the solar spectral irradiance (AM1.5G standard, \SI{1000}{W/m^2} integrated). By engineering $T(\omega)$ to align with vibronic resonances, we can enhance quantum coherence lifetimes and create efficient energy transfer pathways that would be suppressed under broadband illumination.

\subsection{Adaptive Hierarchy of Pure States (adHOPS)}

Simulations are performed using the adaptive Hierarchy of Pure States method, implemented in the open-source MesoHOPS library \cite{Citty2024, Varvelo2021}. This numerically exact technique bypasses the exponential scaling limitations of traditional Hierarchical Equations of Motion (HEOM) by exploiting the dynamic localization of excitons, achieving size-invariant scaling $\mathcal{O}(1)$ for large molecular aggregates ($N>100$) \cite{Varvelo2021, Suess2014}. This represents a critical computational advancement: whereas HEOM scales as $\mathcal{O}(N_{\rm basis}^{N_{\rm sites}})$ rendering calculations intractable for $N_{\rm sites} > 10$, adHOPS maintains constant computational cost independent of system size for spatially localized excitonic dynamics.

The key advantages of adHOPS for agrivoltaic applications include:
\begin{itemize}
\item \textbf{Non-Markovian accuracy}: Full environmental memory effects captured without weak-coupling approximations
\item \textbf{Computational efficiency}: $10\times$ speedup via Process Tensor formalism with Low-Temperature Correction
\item \textbf{Scalability}: Enables simulation of realistic multi-chromophore systems ($>$100 sites)
\item \textbf{Numerical stability}: Avoids recursive error accumulation inherent in density matrix propagation schemes
\end{itemize}

Unlike Markovian approximations (e.g., Lindblad, Redfield) that assume instantaneous environmental relaxation and can miss coherence-assisted transport effects, the non-Markovian treatment preserves structured bath fluctuations that enhance energy transfer efficiency under appropriately engineered spectral conditions.

\subsection{FMO Complex Model System}

We employ the well-characterized Fenna-Matthews-Olsen (FMO) complex as a benchmark system. The FMO monomer consists of 7 bacteriochlorophyll-a molecules with site energies $\varepsilon_n$ ranging from \SIrange{12000}{13000}{\per\cm} and electronic couplings $J_{nm}$ from \SIrange{5}{300}{\per\cm} based on Adolphs \& Renger parameters \cite{Renger2004}. This system is ideal for validation because:
\begin{itemize}
\item Extensive experimental characterization (structure, spectroscopy, dynamics)
\item Strong quantum coherence effects experimentally observed \cite{Engel2007}
\item Intermediate coupling regime where non-Markovian effects are significant
\item Sufficient complexity to exhibit realistic transport phenomena while remaining computationally tractable for benchmarking
\end{itemize}

The composite spectral density for FMO includes:
\begin{itemize}
\item \textbf{Drude-Lorentz contribution}: $\lambda = \SI{35}{\per\cm}$, $\gamma = \SI{50}{\per\cm}$ (protein-solvent modes)
\item \textbf{Vibronic modes}: $\omega_k = \SIlist{150;200;575;1185}{\per\cm}$ with Huang-Rhys factors $S_k = \{0.05, 0.02, 0.01, 0.005\}$ (intramolecular vibrations)
\end{itemize}

These parameters have been validated against experimental absorption spectra and ultrafast spectroscopy data \cite{Adolphs2006, Moix2011}, ensuring quantitative accuracy of our simulations.

\subsection{Multi-Objective Optimization Framework}

For agrivoltaic applications, we must simultaneously optimize two competing objectives:

\textbf{Objective 1 - Electrical Energy Harvesting}: Maximize OPV power conversion efficiency (PCE)
\begin{equation}\label{eq:PCE}
\mathrm{PCE} = \frac{\int_0^\infty [1-T(\omega)] J_{\rm solar}(\omega) \eta_{\rm PV}(\omega) d\omega}{\int_0^\infty J_{\rm solar}(\omega) d\omega}
\end{equation}
where $\eta_{\rm PV}(\omega)$ is the wavelength-dependent photovoltaic conversion efficiency.

\textbf{Objective 2 - Biological Energy Transfer}: Maximize photosynthetic electron transport rate (ETR)
\begin{equation}\label{eq:ETR}
\mathrm{ETR} = k_{\rm RC} \int_0^{t_{\rm max}} \Tr[\bm{\rho}_{\rm RC}(t)] \dd{t}
\end{equation}
where $\bm{\rho}_{\rm RC}(t)$ is the reduced density matrix projected onto the reaction center site and $k_{\rm RC}$ is the charge separation rate constant.

These objectives are inherently conflicting: increasing $T(\omega)$ enhances ETR by delivering more photons to crops but reduces PCE by decreasing photon capture for electricity generation. The optimization problem becomes:

\begin{equation}\label{eq:pareto_optimization}
\max_{\{T(\omega)\}} \left\{ \mathrm{PCE}[T(\omega)], \mathrm{ETR}[T(\omega)] \right\}
\end{equation}
subject to constraints:
\begin{align}
0 &\leq T(\omega) \leq 1 \quad \forall \omega \label{eq:constraint1}\\
\mathrm{PCE} &\geq \mathrm{PCE}_{\rm min} = \SI{15}{\percent} \label{eq:constraint2}\\
\mathrm{FWHM} &\in \SIrange{50}{200}{nm} \label{eq:constraint3}
\end{align}

The constraint in \cref{eq:constraint2} ensures the OPV layer maintains commercially viable efficiency, while \cref{eq:constraint3} restricts spectral windows to physically realizable filter bandwidths.

We parameterize the transmission function as a combination of Gaussian filters:
\begin{equation}\label{eq:transmission_function}
T(\omega) = T_{\rm peak} \sum_i w_i \exp\left[-\frac{(\omega - \omega_{c,i})^2}{2\sigma_i^2}\right]
\end{equation}
where $T_{\rm peak}$ is peak transmission (optimization parameter), $\omega_{c,i}$ are center frequencies (chosen to target vibronic resonances), $\sigma_i$ are bandwidths (FWHM$\approx 2.355\sigma_i$), and $w_i$ are relative weights normalized such that $\sum_i w_i = 1$.

We employ Pareto frontier analysis to identify optimal trade-offs between PCE and ETR. Points on the Pareto frontier represent transmission profiles where neither objective can be improved without degrading the other. This provides designers with a menu of options spanning the trade-off space, enabling selection based on specific application priorities (energy-focused, agriculture-focused, or balanced).

\subsection{Quantum Metrics}

We quantify quantum coherence and transport using established metrics:

\textbf{$l_1$-norm of coherence:}
\begin{equation}\label{eq:l1_coherence}
C_{l_1}(\rho) = \sum_{i \neq j} |\rho_{ij}|
\end{equation}
This measure quantifies total coherence across all excitonic pairs, with larger values indicating more extensive quantum superposition.

\textbf{Coherence lifetime $\tau_c$:} Characteristic decay time for off-diagonal density matrix elements to $1/e$ of initial values, extracted by fitting $|\rho_{ij}(t)| \approx |\rho_{ij}(0)| \exp(-t/\tau_c)$.

\textbf{Inverse participation ratio (delocalization):}
\begin{equation}\label{eq:IPR}
\xi_{\rm deloc} = \left( \sum_n |\psi_n|^4 \right)^{-1}
\end{equation}
Quantifies spatial extent of excitonic wavefunctions. Values approaching the number of chromophores indicate strong delocalization.

\textbf{Quantum advantage:}
\begin{equation}\label{eq:quantum_advantage}
\eta_{\rm quantum} = \frac{\mathrm{ETR}_{\rm HOPS}}{\mathrm{ETR}_{\rm Markovian}} - 1
\end{equation}
Quantifies ETR enhancement relative to Markovian (Redfield) models under identical photon flux conditions. Positive values indicate genuine quantum advantages arising from non-Markovian environmental dynamics.

\textbf{Quantum Fisher Information (QFI):}
\begin{equation}\label{eq:QFI}
F_Q[\rho, \hat{O}] = \Tr[\rho L_{\hat{O}}^2]
\end{equation}
where $L_{\hat{O}}$ is the symmetric logarithmic derivative. QFI quantifies parameter estimation sensitivity, with higher values indicating better quantum resource utilization.

\subsection{Comprehensive Validation Framework}

To ensure reliability of our predictions, we implement a rigorous 12-test validation suite organized in three categories:

\textbf{Convergence Tests (4 tests):}
\begin{itemize}
\item \textbf{HEOM benchmark}: Agreement $< \SI{2}{\percent}$ deviation for 3-site system against numerically exact HEOM propagation
\item \textbf{Matsubara cutoff}: Observable convergence to $< \SI{0.5}{\percent}$ for $N_{\rm Mat} \geq 10$ temperature modes
\item \textbf{Time step}: Invariance to factor-of-2 time step changes (\SI{0.5}{fs} vs \SI{1.0}{fs})
\item \textbf{Hierarchy truncation}: $< \SI{1}{\percent}$ variation for thresholds spanning $10^{-7}$ to $10^{-9}$
\end{itemize}

\textbf{Physical Consistency Tests (4 tests):}
\begin{itemize}
\item \textbf{Trace preservation}: $|\Tr(\rho(t)) - \num{1}| < \num{e-12}$ at all times
\item \textbf{Positivity}: All density matrix eigenvalues $> -\num{e-10}$ (within numerical precision)
\item \textbf{Energy conservation}: $< \SI{0.1}{\percent}$ drift in closed-system limit
\item \textbf{Detailed balance}: Equilibrium populations match Boltzmann distribution within \SI{1}{\percent}
\end{itemize}

\textbf{Environmental Robustness Tests (4 tests):}
\begin{itemize}
\item \textbf{Temperature variations}: Quantum advantage maintained within \SI{15}{\percent} for $\pm \SI{10}{K}$ around \SI{295}{K}
\item \textbf{Static disorder}: Enhancement reduced by $< \SI{20}{\percent}$ for Gaussian site energy disorder $\sigma = \SI{50}{\per\cm}$
\item \textbf{Bath parameter variations}: Qualitative features preserved for $\pm \SI{20}{\percent}$ variations in $\lambda$, $\gamma$, $\omega_k$
\item \textbf{Markovian limit recovery}: Agreement with Redfield theory in high-temperature ($T > \SI{500}{K}$) limit
\end{itemize}

This comprehensive validation establishes confidence that observed quantum advantages are genuine physical phenomena rather than numerical artifacts, and that predictions remain robust under realistic environmental perturbations encountered in practical agrivoltaic implementations.

All simulations employ double-precision floating-point arithmetic and were performed using MesoHOPS v1.2.0 on a computational cluster with 32-core AMD EPYC processors. Typical runtime for a single FMO complex simulation (7 chromophores, \SI{100}{ps} dynamics) is approximately 4 hours on a single node, enabling extensive parameter sweeps for Pareto optimization.
