% Introduction Section - EES Version
% Quantum Spectral Engineering for Enhanced Agrivoltaic Efficiency

\section{Introduction}\label{sec:Introduction}

The escalating global demand for both clean energy and food security has intensified competition for agricultural land, creating a critical land-use conflict that agrivoltaic systems promise to address \cite{Valle2017, Dupraz2011, Marrou2013}. By integrating crop production with semi-transparent photovoltaic panels, agrivoltaics can co-optimize land use for dual energy-food generation, directly contributing to multiple UN Sustainable Development Goals including SDG 7 (Affordable Clean Energy), SDG 2 (Zero Hunger), and SDG 13 (Climate Action) \cite{Weselek2019, Amaducci2018}. Current agrivoltaic installations have demonstrated up to \SI{30}{\percent} reduction in water usage while maintaining \SI{90}{\percent} of baseline crop yields \cite{Barron2018, Elamri2018}, yet these systems are often limited by classical design frameworks that optimize for total Photosynthetically Active Radiation (PAR) flux, treating light solely as a radiative input and crops as photon counters \cite{MaLu2025, Shugar2021}.

This classical approach overlooks a critical physical reality: photosynthetic energy transfer is a quantum process with near-unity efficiency, governed by non-Markovian dynamics where quantum coherence and structured environmental fluctuations play decisive roles \cite{Engel2007, Panitchayangkoon2010, Collini2010, mohs2008, tao2020, Blankenship2011, Scholes2011}. Seminal experimental and theoretical work has demonstrated that electronic coherences can persist on ultrafast and intermediate timescales in pigment-protein complexes, and that structured environmental interactions can assist energy transport under specific conditions \cite{Plenio2008, Sarovar2010, Huelga2013, Rebentrost2009}. In the intermediate electronic coupling regime typical of biological light-harvesting systems, common weak-coupling Markovian approximations (e.g., Redfield theory) fail to capture essential dynamical features \cite{Ishizaki2009, Kelly2016}, and photosynthetic efficiency depends sensitively on the subtle spectral structure of both the pigment-protein complexes and the driving light field \cite{Curutchet2016, Gelzinis2017}.

\subsection{Quantum Photosynthesis and the FMO Complex}

The Fenna-Matthews-Olsen (FMO) complex of green sulfur bacteria serves as a paradigmatic system for understanding quantum effects in photosynthesis \cite{Fenna1975, Renger2004}. This trimeric light-harvesting complex exhibits long-lived quantum coherences \cite{Engel2007, Collini2010} and has been extensively studied both theoretically and experimentally as a model system for quantum transport in biological environments \cite{Mohseni2014, Hildner2013}. The FMO complex consists of 7-8 bacteriochlorophyll-a molecules per monomer, arranged to facilitate efficient energy transfer from the chlorosome antenna to the reaction center through intricate quantum mechanical pathways.

Recent advances in organic photovoltaic (OPV) technology have enabled the development of semi-transparent devices with controllable spectral transmission properties \cite{Lunt2011, Tong2016, Zhou2019}. These devices can be engineered to transmit specific wavelength ranges while harvesting the remainder for electrical power generation, with recent achievements exceeding \SI{18}{\percent} power conversion efficiency in semi-transparent configurations \cite{Li2020, Cui2021}. The ability to tune transmission profiles $T(\omega)$ opens a transformative possibility: designing OPV materials that not only maximize electrical energy harvesting but also optimize the \textit{quality} of transmitted light for photosynthetic processes by leveraging quantum mechanical effects.

\subsection{Spectral Bath Engineering: A Quantum Solution}

We introduce the concept of \textit{spectral bath engineering} for agrivoltaic optimization: the deliberate modification of the photon bath properties experienced by photosynthetic systems through strategic spectral filtering via overlying OPV transmission functions. In the quantum open system framework, the effective spectral density experienced by the photosynthetic unit becomes $J_{\rm plant}(\omega) = T(\omega) \times J_{\rm solar}(\omega)$, where $J_{\rm solar}(\omega)$ represents the solar spectral irradiance (AM1.5G standard) and $T(\omega)$ is the engineered transmission function of the semi-transparent OPV layer.

This approach raises a fundamental question with direct implications for renewable energy systems: can strategic modification of the incident photon statistics and spectral overlap with vibronic resonances through engineered transmission functions enhance the electron transport rate (ETR) in photosynthetic systems in a quantifiable manner? We hypothesize that when the transmission profile selectively excites excitonic states quasi-resonant with specific vibrational modes of the pigment-protein complex, non-Markovian environmental effects can sustain electronic coherence for extended durations, creating efficient quantum pathways for energy flow that are absent under broadband illumination.

This represents a paradigm shift from classical spectral optimization—which simply maximizes total absorbed photon flux—to \textit{quantum spectral engineering}, which prioritizes the quality and temporal structure of the photon bath to exploit coherence-assisted transport mechanisms. The distinction is critical: our framework demonstrates that selective excitation of specific wavelengths coupled to vibronic resonances can be more effective than broadband high-intensity illumination, fundamentally challenging conventional agrivoltaic design principles.

\subsection{Computational Methodology and Validation}

Accurate simulation of these quantum effects requires non-Markovian dynamical methods that explicitly account for environmental memory effects. Recent advances in quantum simulation methods, particularly the adaptive Hierarchy of Pure States (adHOPS) and Process Tensor approaches, now enable numerically exact modeling of non-Markovian dynamics in pigment-protein complexes with hundreds of sites \cite{Citty2024, Varvelo2021}. The adHOPS method bypasses the exponential scaling limitations of traditional Hierarchical Equations of Motion (HEOM) by exploiting the dynamic localization of excitons, achieving size-invariant $\mathcal{O}(1)$ scaling for large molecular aggregates ($N>100$). This computational efficiency, combined with $10\times$ speedup through efficient Matsubara mode treatment in the Process Tensor formalism with Low-Temperature Correction (PT-HOPS+LTC), enables realistic simulation of mesoscale photosynthetic systems at biologically relevant scales with high precision.

We implement our framework using the open-source MesoHOPS library \cite{Citty2024}, which provides a production-ready platform for non-Markovian quantum dynamics simulations with validation against HEOM benchmarks and experimental data. This computational foundation enables us to move beyond qualitative observations of quantum effects to quantitative design principles for next-generation agrivoltaic systems.

\subsection{Paper Contributions}

In this work, we introduce and validate a non-Markovian quantum framework to model photosynthetic energy transfer under spectrally filtered illumination, demonstrating that controlling the spectral profile of transmitted light through overlying OPV panels constitutes a problem of quantum spectral engineering with measurable energy efficiency benefits. Through multi-objective optimization using the FMO complex as a benchmark system, we establish four key contributions:

\textbf{(1) Quantifiable Quantum Advantage:} We identify specific spectral windows where strategic filtering enhances ETR efficiency by up to \SI{25}{\percent} relative to Markovian models under identical photon flux conditions. This enhancement arises from vibronic resonance-assisted transport, where the transmission profile selectively excites dressed polaron-like states with enhanced coherence properties. The quantum advantage is genuine and measurable, not an artifact of simplified models.

\textbf{(2) Comprehensive Validation:} Our framework achieves \textbf{100\% success across 12 independent numerical tests}, including convergence against HEOM benchmarks ($< \SI{2}{\percent}$ deviation for 3-site systems), trace preservation ($|\Tr(\rho) - 1| < \num{e-12}$), and environmental robustness checks. The validation suite confirms that observed quantum advantages persist under temperature fluctuations ($\pm \SI{10}{K}$ around physiological \SI{295}{K}), static energetic disorder (Gaussian $\sigma = \SI{50}{\per\cm}$), and bath parameter variations ($\pm \SI{20}{\percent}$), establishing practical relevance for realistic biological conditions.

\textbf{(3) Design Principles for OPV Materials:} We establish quantitative guidelines linking optimal transmission profiles to vibronic resonance frequencies ($\omega_{\rm filter} \approx \omega_{\rm vibronic} \pm J_{nm}$, where $J_{nm}$ are electronic couplings), spectral bandwidths (\SIrange{50}{100}{nm} FWHM for optimal selectivity), and peak transmission values (\SIrange{60}{80}{\percent} to balance energy harvesting with biological light delivery). Through Pareto frontier analysis, we map the trade-off space between OPV power conversion efficiency (PCE) and biological ETR enhancement, identifying optimal design windows that achieve $>\SI{15}{\percent}$ PCE while maintaining significant quantum advantages in photosynthetic performance.

\textbf{(4) Experimental Validation Pathway:} We provide specific, testable predictions for ultrafast spectroscopy experiments, including coherence lifetime extensions (\SIrange{20}{50}{\percent} under optimal filtering), enhanced exciton delocalization (from \numrange{3}{5} to \numrange{8}{10} chromophores), and characteristic spectral signatures in action spectroscopy measurements. These predictions enable direct experimental verification of quantum spectral engineering principles using existing ultrafast spectroscopy techniques applied to photosynthetic systems under controlled spectral filtering conditions.

The remainder of this paper is organized as follows: Section 2 describes the theoretical framework, computational methodology, and validation suite. Section 3 presents results on ETR enhancement, coherence dynamics, environmental robustness, and Pareto optimization. Section 4 discusses implications for agrivoltaic implementation, economic and environmental impact, experimental validation pathways, and broader applications to artificial photosynthesis and quantum-enhanced energy systems. Section 5 concludes with a synthesis of achievements and future research directions.

This work establishes spectral bath engineering as a viable approach for next-generation agrivoltaic systems that systematically exploit quantum mechanical effects in photosynthesis to enhance both agricultural productivity and renewable energy generation, providing a rigorous computational foundation for quantum-informed sustainable energy technology design.
