% Discussion Section - EES Version
% Quantum Spectral Engineering for Enhanced Agrivoltaic Efficiency

\section{Discussion}\label{sec:Discussion}

\subsection{Quantum advantage in a renewable energy context}

The \SI{25}{\percent} ETR enhancement from spectral bath engineering has direct consequences for agrivoltaic system design. Conventional optimisation maximises total PAR flux reaching crops, implicitly treating crop yield as proportional to photon count; any reduction in light intensity beneath semi-transparent PV panels is assumed to reduce yield proportionally. Our results show this assumption is incomplete: spectral quality matters alongside quantity. By filtering to enhance quantum coherence, higher biological efficiency per absorbed photon partially compensates for reduced total flux, enabling greater PV coverage fractions than classical models predict.

For a 1-hectare agrivoltaic installation with \num{40}\% PV coverage, classical analysis predicts \num{40}\% reduction in crop yield. Spectral bath engineering reduces this penalty to \numrange{25}{28}\% (at \num{15}\% quantum ETR enhancement), representing a \numrange{30}{40}\% improvement in agricultural productivity relative to classical designs. For high-value crops (USD~\numrange{5000}{10000}\,\si{\per\hectare} annual revenue), this translates to USD~\numrange{1500}{3000}\,\si{ha^{-1}\,yr^{-1}} additional income.

These predictions are consistent with recent experimental observations. Adeyemi et al.\ \cite{adeyemi2025spectral} reported that conventional spectral filtering modulates crop microclimates but often incurs yield penalties when PAR is significantly reduced; our framework mitigates these penalties through improved biological efficiency per photon. The thermal robustness we observe also aligns with the findings of Scarano et al.\ \cite{scarano2024thermal}, who emphasised the importance of agrivoltaic shading for mitigating heat stress. Our work adds a quantum dimension: stable temperatures near \SI{295}{K} are also optimal for coherence-assisted transport.

Beyond direct agricultural benefits, enhanced photosynthetic efficiency reduces water requirements (fewer photons needed for equivalent biomass), mitigates thermal dissipation, and improves nutrient use efficiency through better CO$_2$ fixation per unit resource input.

\subsection{Agrivoltaic implementation strategy}

\subsubsection{OPV material design guidelines}

\cref{tab:opv_specs} consolidates the OPV design specifications derived from Pareto optimisation.

% OPV Design Specifications Table
\begin{table}[ht]
\centering
\caption{\textbf{OPV design specifications for quantum-enhanced agrivoltaics.} Derived from Pareto optimisation over 10,000+ configurations. Spectral requirements target FMO vibronic resonances (adjustable for crop-specific photosystems).}
\label{tab:opv_specs}
\begin{tabular}{lll}
\toprule
\textbf{Parameter} & \textbf{Specification} & \textbf{Rationale} \\
\midrule
\multicolumn{3}{l}{\textit{Spectral Requirements}} \\
\quad Target wavelengths & \SIlist{750;820}{nm} & FMO vibronic resonances \\
\quad Bandwidth (FWHM) & \SIrange{70}{90}{nm} & Selective excitation \\
\quad Peak transmission & \SIrange{65}{75}{\percent} & PAR/energy balance \\
\quad Out-of-band absorption & $>\SI{85}{\percent}$ & OPV efficiency \\[0.5em]
\multicolumn{3}{l}{\textit{Performance Targets}} \\
\quad PCE (minimum) & $\geq\SI{15}{\percent}$ & Commercial viability \\
\quad ETR enhancement & $\geq\SI{15}{\percent}$ & Quantum advantage \\
\quad Operating range & \SIrange{270}{320}{K} & All-climate \\
\quad Lifetime & > \SI{10000}{hours} & > \SI{1}{year} \\[0.5em]
\multicolumn{3}{l}{\textit{Sustainability Requirements}} \\
\quad Biodegradability & > \SI{80}{\percent} (\SI{180}{days}) & OECD 301 \\
\quad Material limits & No Pb, Cd, halogens & Safety \\
\bottomrule
\end{tabular}
\end{table}

These targets are achievable with current-generation OPV materials \cite{Li2020, Cui2021} incorporating bio-derived polymers (e.g., cellulose derivatives, lignin-based side chains) and ester linkages for enzymatic degradation. Molecular design prioritising enhanced $\pi$-conjugation, optimal HOMO-LUMO gaps ($\sim\SIrange{1.6}{1.8}{\electronvolt}$) for dual-band absorption, and non-aromatic biodegradable side chains can simultaneously meet performance and sustainability requirements. Recent tandem OPV architectures with tunable transmission windows \cite{Li2020, Cui2021} provide a technological foundation for realising these specifications.

\subsubsection{Geographic optimisation}

Optimal transmission profiles vary by latitude and climate. Temperate zones (\SIrange{40}{60}{\degree} latitude) benefit from dual-band filtering at \SIlist{750;820}{nm} with seasonal adjustment potential. Tropical zones (\SIrange{0}{25}{\degree} latitude) favour broader single-band transmission at \SI{780}{nm}, leveraging year-round temperature stability near the quantum optimum. Desert regions benefit from narrower-band filtering at \SI{750}{nm} to maximise selectivity under intense direct sunlight, with additional infrared reflection for heat stress mitigation. Site-specific optimisation can yield an additional \SIrange{5}{10}{\percent} improvement relative to universal designs.

\subsubsection{Operational considerations}

Practical deployment must account for angle-dependent transmission (quantum advantages remain substantial at \SIrange{18}{22}{\percent} for tilt angles up to \SI{30}{\degree}), dust and soiling effects on spectral profiles, OPV degradation over time, and crop-specific photosystem compositions. Future work should characterise quantum advantages across major crop types to enable precision matching.

\subsection{Economic and environmental impact}

\subsubsection{Economic analysis}

We compare a classical agrivoltaic installation (\SI{35}{\percent} PV coverage, \SI{15}{\%} PCE, \SI{70}{\%} crop yield) against a quantum-optimised design (\SI{40}{\percent} PV coverage, \SI{16}{\%} PCE, \SI{75}{\%} crop yield). The classical configuration yields USD~\num{6000}\,\si{ha^{-1}\,yr^{-1}} total revenue (USD~\num{2500} electrical + USD~\num{3500} agricultural). The quantum-optimised system yields USD~\num{6470}\,\si{ha^{-1}\,yr^{-1}} (USD~\num{2720} electrical + USD~\num{3750} agricultural), a net improvement of USD~\num{470}\,\si{ha^{-1}\,yr^{-1}} (\num{+7.8}\%). Over a 20-year system lifetime, this represents USD~\num{9400}\,\si{\per\hectare} additional value.

\cref{tab:economic_analysis} extends this analysis across climate zones.

% Economic Analysis Table
\begin{table}[ht]
\centering
\caption{\textbf{Economic benefit of quantum-enhanced agrivoltaics by climate zone.} Assumptions: wheat crop, OPV cost \num{150}\,USD/m$^2$, quantum OPV premium +15\%, crop value \num{250}\,USD/t. ROI over 10-year horizon with 2\% annual degradation and \num{0.15}\,USD/kWh electricity price.}
\label{tab:economic_analysis}
\begin{tabular}{lcccc}
\toprule
\textbf{Climate Zone} & \textbf{Baseline} & \textbf{ETR} & \textbf{Value/ha/yr} & \textbf{10yr ROI} \\
 & \textbf{(t/ha)} & \textbf{(\%)} & \textbf{(USD)} & \textbf{(\%)} \\
\midrule
Temperate & 8.2 & 22 & 1,850 & 185 \\
Mediterranean & 7.5 & 25 & 2,100 & 210 \\
Tropical & 9.8 & 18 & 2,450 & 245 \\
Subtropical & 8.9 & 20 & 2,180 & 218 \\
Semi-arid & 6.1 & 28 & 1,920 & 192 \\
Continental & 7.3 & 19 & 1,520 & 152 \\
\midrule
\textbf{Average} & \textbf{7.9} & \textbf{22} & \textbf{2,000} & \textbf{200} \\
\bottomrule
\end{tabular}
\end{table}

For high-value specialty crops (\numrange{15000}{25000}\,USD/ha baseline), quantum advantages yield \numrange{1500}{3000}\,USD additional annual revenue, enabling agrivoltaics in premium agricultural markets.

\subsubsection{Environmental benefits}

Quantum spectral engineering provides compounding environmental benefits: \numrange{10}{12}\% reduction in irrigation requirements for equivalent biomass production; estimated additional carbon sequestration of \numrange{0.5}{1.0}\,tCO$_2$\,ha$^{-1}$\,yr$^{-1}$ from enhanced photosynthesis; improved land-use efficiency reducing pressure on natural habitats (SDG~15); and strengthened food-energy co-production for regions with limited land availability. Life cycle assessment indicates an overall \numrange{15}{20}\% reduction in environmental footprint relative to classical designs.

\subsection{Experimental validation pathway}

Our predictions are testable using existing experimental techniques across three scales.

\textbf{Ultrafast spectroscopy.} Two-dimensional electronic spectroscopy (2DES) under filtered vs.\ broadband illumination should reveal \SIrange{20}{50}{\percent} extension of quantum beating lifetimes at vibronic resonances. Specific predictions include beating frequency enhancement at $\sim \SI{180}{\per\cm}$ with \SIrange{25}{40}{\percent} amplitude increase, cross-peak lifetime extension from \SI{300}{fs} to \SIrange{400}{500}{fs}, and spectral signatures at \numlist{750;820}\,nm. Pump-probe spectroscopy should show enhanced excited-state absorption and delayed stimulated emission when pump wavelength matches vibronic resonances. Transient absorption measurements should reveal enhanced P680$^+$ signal and \SIrange{50}{100}{fs} delayed stimulated emission under filtered illumination.

\textbf{Controlled environment experiments.} Intact photosynthetic systems (isolated chloroplasts, algae cultures) under LED arrays with programmable spectral profiles should show \SIrange{8}{15}{\percent} quantum yield enhancement at equal total photon flux. Pulse-amplitude-modulated (PAM) fluorometry should detect \SIrange{15}{25}{\percent} enhancement in $\Phi_{\rm PSII}$ and \SIrange{12}{18}{\percent} increase in photochemical quenching under filtered illumination.

\textbf{Field trials.} Multi-season trials comparing quantum-optimised OPV panels against classical semi-transparent PV and unshaded controls, across multiple climatic zones, should demonstrate \SIrange{10}{18}{\percent} higher crop productivity at equivalent PV coverage fractions.

\subsection{Limitations and future work}

Several limitations should be noted. The FMO complex represents only one component of the photosynthetic apparatus; full chloroplast modelling incorporating PSI, PSII, cytochrome b$_6$f, and ATP synthase is needed for quantitative crop-level predictions. Our calculations assume fixed transmission profiles; adaptive filtering responsive to environmental conditions could yield further benefits. Integration with Calvin cycle kinetics and crop-specific photosystem compositions is also necessary for biomass-level predictions.

Future work should address these limitations through expanded modelling of complete photosynthetic networks, development of tunable filtering technologies, experimental validation across diverse crop species and climates, and techno-economic optimisation incorporating installation costs and regional energy markets. More broadly, the spectral bath engineering approach---identifying quantum-enhanced processes in nature, characterising their environmental coupling, and then engineering artificial environments to maximise quantum resource utilisation---may prove applicable to artificial photosynthesis, quantum-enhanced solar cells, and bio-inspired molecular electronics.
