% Results Section - EES Version
% Quantum Spectral Engineering for Enhanced Agrivoltaic Efficiency

\section{Results}\label{sec:Results}

\subsection{Quantum Enhancement of Electron Transport Rate}

Through systematic optimization of the OPV transmission function $T(\omega)$, we demonstrate that strategic spectral filtering can enhance photosynthetic electron transport rate (ETR) by up to \SI{25}{\percent} relative to Markovian models under identical photon flux conditions (\cref{fig:quantum_advantage}). This enhancement represents a genuine quantum advantage arising from vibronic resonance-assisted transport, not achievable through classical spectral optimization approaches that simply maximize total absorbed photon flux.

\begin{figure}[ht]
\centering
\includegraphics[width=0.85\textwidth]{Graphics/Quantum_Advantage_in_Energy_Transfer.pdf}
\caption{Quantum enhancement of electron transport rate (ETR) through strategic spectral filtering, demonstrating \SI{25}{\percent} improvement over classical approaches. Optimized transmission windows (\SIlist{750;820}{nm} dual-band) align with vibronic resonances in photosynthetic complex, enabling coherence-assisted energy transfer at physiological temperature (\SI{295}{K}).}
\label{fig:quantum_advantage}
\end{figure}

\begin{equation}\label{eq:resonance_condition}
\omega_{\rm filter} \approx \omega_{\rm vibronic} \pm J_{nm}
\end{equation}
where $\omega_{\rm filter}$ characterizes the filtered spectrum peak, $\omega_{\rm vibronic}$ are the relevant vibronic mode frequencies (\SIlist{150;200;575;1185}{\per\cm} for FMO), and $J_{nm}$ are electronic coupling strengths (\SIrange{5}{300}{\per\cm}). When this condition is satisfied, the transmission profile selectively excites excitonic states that couple strongly to vibrational modes, creating dressed polaron-like states with enhanced coherence properties.

For the FMO complex, maximum quantum advantage occurs when filtering targets the \SI{575}{\per\cm} vibronic mode with transmission windows centered at $\lambda_c \approx \SI{750}{nm}$ (\SI{13333}{\per\cm}) and $\lambda_c \approx \SI{820}{nm}$ (\SI{12195}{\per\cm}), corresponding to transitions that are quasi-resonant with this key vibrational mode. Under these conditions, the non-Markovian environment sustains electronic coherence for extended durations, enabling efficient quantum pathways for energy flow to the reaction center site through constructive interference effects that are absent under broadband illumination or predicted by Markovian models.

The quantum advantage metric $\eta_{\rm quantum}$ (\cref{eq:quantum_advantage}) quantifies the ETR enhancement relative to Redfield (Markovian) theory under identical conditions. Values of $\eta_{\rm quantum} = 0.25$ indicate that the non-Markovian HOPS simulations predict \SI{25}{\percent} higher ETR than Markovian models for the same photon flux, demonstrating that environmental memory effects---ignored in Markovian treatments---play a crucial role in determining photosynthetic efficiency under spectrally engineered illumination.

\subsection{Coherence Dynamics and Quantum Mechanism}

Analysis of the $l_1$-norm of coherence (\cref{eq:l1_coherence}) reveals that optimal spectral filtering extends coherence lifetimes by \SIrange{20}{50}{\percent} compared to unfiltered broadband illumination (\cref{fig:coherence}). The coherence lifetime $\tau_c$ under optimal filtering reaches values exceeding \SI{500}{fs} at physiological temperature (\SI{295}{K}), significantly longer than the $\tau_c \approx \SI{300}{fs}$ observed under broadband solar illumination. This extension is not merely a consequence of reduced light intensity---when normalized to equal absorbed photon flux, filtered conditions still exhibit \SIrange{20}{50}{\percent} longer coherence lifetimes, confirming that spectral quality, not just quantity, determines quantum transport efficiency.

The exciton delocalization length, quantified by the inverse participation ratio $\xi_{\rm deloc}$ (\cref{eq:IPR}), increases from \numrange{3}{5} chromophores under broadband illumination to \numrange{8}{10} chromophores under optimized spectral filtering. This enhanced delocalization indicates that quantum superposition states extend over larger numbers of chromophores, enabling more parallel pathways for efficient energy transfer. Crucially, this delocalization is maintained at physiological temperatures, not just at cryogenic conditions where quantum effects are typically most pronounced.

The underlying physical mechanism relies on vibronic resonance matching. When the spectral filter selectively excites states quasi-resonant with vibrational modes, effective polaron formation occurs with modified energy transfer dynamics. The dressed states exhibit reduced dephasing rates because the filtering suppresses frequencies that induce decoherence while preserving those that support coherent pathways. Quantum Fisher Information analysis confirms that filtered conditions maximize parameter estimation sensitivity, indicating optimal quantum resource utilization for energy transfer.

Time-resolved analysis shows that under optimal filtering, the exciton population exhibits oscillatory dynamics with frequencies matching vibronic mode energies, signature of coherent vibronic coupling. These oscillations persist for hundreds of femtoseconds—timescales comparable to energy transfer times—enabling vibronic modes to actively participate in transport rather than merely acting as dissipative channels as in purely classical models.

\begin{figure}[ht]
\centering
\includegraphics[width=0.85\textwidth]{Graphics/Figure_3.png}
\caption{\textbf{Coherence dynamics under optimal spectral filtering.} (a) Temporal evolution of $l_1$-norm coherence showing \SIrange{20}{50}{\percent} lifetime extension under filtered illumination (\SIlist{750;820}{nm} dual-band) compared to broadband. (b) Exciton delocalization measured by inverse participation ratio (IPR), demonstrating spatial spread from 3--5 to 8--10 chromophores, enabling efficient long-range energy transfer. (c) Spectral density components showing alignment between filter-modified bath (Drude + vibronic) and FMO excitonic transitions. (d) System-bath correlation function illustrating non-Markovian memory effects that sustain coherence. All data at \SI{295}{K} with realistic disorder ($\sigma = \SI{50}{\per\cm}$).}
\label{fig:coherence}
\end{figure}

These extended coherence lifetimes enable exciton delocalization across \numrange{8}{10} chromophores under optimal filtering, compared to \numrange{3}{5} chromophores under broadband illumination (\cref{fig:coherence}b). The inverse participation ratio (IPR), which quantifies spatial exciton delocalization, increases from $N_{\rm eff} \approx 4$ to $N_{\rm eff} \approx 9$ under optimal filtering. This enhanced delocalization allows excitations to sample a larger conformational space, increasing the probability of finding efficient pathways to the reaction center through quantum interference effects.

\cref{tab:quantum_metrics} provides a comprehensive comparison of quantum metrics between filtered and broadband illumination, quantifying the enhancement across all key observables. These improvements are mutually reinforcing: extended coherence enables greater delocalization, which in turn facilitates faster energy transfer as measured by the \SI{25}{\percent} ETR enhancement.

% Quantum Metrics Comparison Table
\begin{table}[ht]
\centering
\caption{\textbf{Quantum metrics comparison: Spectral filtering vs broadband illumination.} All measurements at physiological temperature (\SI{295}{K}) with realistic static disorder ($\sigma = \SI{50}{\per\cm}$). Filtered condition uses optimized dual-band transmission (\SI{750}{nm}/\SI{820}{nm}, FWHM \SI{70}{nm}, \SIrange{65}{75}{\percent} peak transmission). Errors represent \SI{95}{\percent} confidence intervals from ensemble averaging over 500 disorder realizations. Lower entropy indicates more coherent quantum state.}
\label{tab:quantum_metrics}
\begin{tabular}{lccc}
\toprule
\textbf{Metric} & \textbf{Filtered (750/820 nm)} & \textbf{Broadband} & \textbf{Enhancement} \\
\midrule
ETR (relative) & \num{1.25 \pm 0.03} & \num{1.00 \pm 0.02} & \SI{+25}{\percent} \\
Coherence lifetime (fs) & \num{420 \pm 35} & \num{280 \pm 25} & \SI{+50}{\percent} \\
Delocalization (sites) & \num{8.2 \pm 0.7} & \num{4.1 \pm 0.5} & \SI{+100}{\percent} \\
QFI (max) & \num{12.4 \pm 1.1} & \num{7.8 \pm 0.8} & \SI{+59}{\percent} \\
Purity ($t = \SI{500}{fs}$) & \num{0.82 \pm 0.04} & \num{0.71 \pm 0.05} & \SI{+15}{\percent} \\
Von Neumann entropy & \num{0.51 \pm 0.06} & \num{0.73 \pm 0.07} & \SI{-30}{\percent}$^*$ \\
\bottomrule
\multicolumn{4}{l}{\scriptsize $^*$Lower entropy indicates more ordered quantum state (beneficial).} \\
\end{tabular}
\end{table}

The quantum Fisher information (QFI) enhancement of \SI{59}{\percent} indicates improved sensitivity to small perturbations in system parameters, which correlates with enhanced energy transport efficiency. The purity increase (\SI{+15}{\percent}) and entropy decrease (\SI{-30}{\percent}) confirm that filtered illumination maintains more coherent quantum states throughout the energy transfer process.

\subsection{Pareto Optimization: Balancing Energy Generation and Agriculture}

Multi-objective optimization reveals a well-defined Pareto frontier mapping the trade-off between OPV power conversion efficiency (PCE) and biological ETR enhancement (\cref{fig:pareto}). This frontier provides agrivoltaic system designers with a menu of optimal configurations spanning the full range from electricity-focused to agriculture-focused implementations.

Key findings from Pareto analysis:

\textbf{Balanced Configuration} (recommended for most applications):
\begin{itemize}
\item PCE = \SIrange{16}{18}{\percent} (commercially viable)
\item ETR enhancement = \SIrange{15}{20}{\percent} relative to unfiltered baseline
\item Transmission peak $T_{\rm peak} = \SIrange{65}{75}{\percent} $
\item Spectral windows: dual-band at \SI{750}{nm} and \SI{820}{nm}, FWHM \SI{70}{nm}
\end{itemize}

\textbf{Energy-Focused Configuration}:
\begin{itemize}
\item PCE = \SIrange{19}{21}{\percent} (maximized electrical generation)
\item ETR enhancement = \SIrange{8}{12}{\percent} (moderate agricultural benefit)
\item Transmission peak $T_{\rm peak} = \SIrange{55}{60}{\percent} $
\item Spectral windows: single narrow band, FWHM \SI{50}{nm}
\end{itemize}

\textbf{Agriculture-Focused Configuration}:
\begin{itemize}
\item PCE = \SIrange{13}{15}{\percent} (minimum viable electrical generation)
\item ETR enhancement = \SIrange{22}{25}{\percent} (maximized crop productivity)
\item Transmission peak $T_{\rm peak} = \SIrange{75}{80}{\percent} $
\item Spectral windows: dual broad bands, FWHM \SIrange{90}{100}{nm}
\end{itemize}

\begin{figure}[ht]
\centering
\includegraphics[width=0.75\textwidth]{Graphics/Pareto_Front__PCE_vs_ETR_Trade_off.pdf}
\caption{Pareto frontier from multi-objective optimization showing trade-off between OPV power conversion efficiency (PCE) and photosynthetic ETR enhancement. Three optimal configurations identified: Balanced (\SIrange{16}{18}{\percent} PCE, \SIrange{15}{20}{\percent} ETR), Energy-focused (\SIrange{19}{21}{\percent} PCE, \SIrange{8}{12}{\percent} ETR), Agriculture-focused (\SIrange{13}{15}{\percent} PCE, \SIrange{22}{25}{\percent} ETR). All configurations maintain commercially viable performance while providing significant quantum advantages.}
\label{fig:pareto}
\end{figure}

The Pareto frontier demonstrates that significant quantum advantages (\SIrange{15}{25}{\percent} ETR enhancement) are achievable while maintaining commercially viable PCE ($>\SI{15}{\percent}$), addressing a key concern for practical implementation. Systems operating at the Pareto extreme (\SI{25}{\percent} ETR enhancement, \SI{13}{\percent} PCE) may be suitable for high-value crop applications where agricultural productivity outweighs electrical revenue, while balanced configurations offer compelling value propositions for general deployment.

Economic analysis (detailed in Discussion) indicates that even moderate quantum enhancements (\SI{15}{\percent}) can yield substantial benefits: for a 1-hectare agrivoltaic installation with high-value crops, \SI{15}{\percent} ETR improvement translates to \$\numrange{3000}{5000} additional annual agricultural revenue, partially offsetting the $\sim\SI{10}{\percent}$ reduction in electrical revenue from lowering PCE from \SI{20}{\percent} to \SI{16}{\percent}.

\subsection{Temperature Robustness and Physiological Relevance}

The quantum advantage persists across physiologically relevant temperature ranges (\cref{fig:temperature}). While coherence effects are strongest at low temperatures (\SI{280}{K}: $\eta_{\rm quantum} = 0.30$), significant enhancement remains at standard field conditions (\SI{295}{K}: $\eta_{\rm quantum} = 0.25$) and even under moderate heat stress (\SI{310}{K}: $\eta_{\rm quantum} = 0.18$).

Temperature dependence exhibits non-monotonic behavior with maximum coherence preservation at intermediate temperatures (\SIrange{285}{300}{K}). This reflects a balance: thermal energy must be sufficient to populate vibronic modes that mediate coherent transport, but not so high as to induce excessive dephasing. The peak at \SI{295}{K}—precisely the typical temperature for temperate climate agriculture—suggests evolutionary optimization may have tuned photosynthetic systems to exploit quantum effects under native conditions.

Seasonal variations in ambient temperature (\SIrange{270}{320}{K} range) maintain quantum advantages within \SIrange{20}{30}{\percent} range, indicating year-round benefits for agrivoltaic implementations across diverse climatic zones. Even under extreme heat conditions (\SI{330}{K}, representing mid-summer desert conditions), residual quantum enhancement of \SIrange{10}{12}{\percent} persists, demonstrating robustness beyond laboratory settings.

The temperature robustness has critical implications for field deployment: quantum spectral engineering is not limited to controlled laboratory environments but provides tangible benefits under real-world agricultural conditions with typical diurnal and seasonal temperature variations.

\subsection{Disorder Tolerance and Biological Realism}

Static energetic disorder with Gaussian distribution ($\sigma = \SI{50}{\per\cm}$, typical of biological systems due to structural heterogeneity) reduces quantum advantage by approximately \SI{20}{\percent}, but significant enhancement (\SIrange{18}{20}{\percent}) persists (\cref{fig:disorder}). This confirms viability in realistic biological environments with native structural fluctuations.

Ensemble averaging over 100 disorder realizations shows that the quantum advantage distribution remains statistically significant: mean $\langle\eta_{\rm quantum}\rangle = \num{0.20 \pm 0.04}$ under $\sigma = \SI{50}{\per\cm}$ disorder versus $\eta_{\rm quantum} = 0.25$ for the disorder-free case. The narrowness of the distribution (coefficient of variation $<\SI{20}{\percent}$) indicates that quantum enhancement is a robust feature, not sensitive to specific molecular configurations.

Increasing disorder to $\sigma = \SI{100}{\per\cm}$ (extreme biological variability) reduces mean quantum advantage to \SIrange{12}{15}{\percent}, still providing measurable benefits. This robustness arises because vibronic resonance conditions depend primarily on mode frequencies---properties determined by intramolecular bonds that are relatively insensitive to environmental fluctuations---rather than precise site energies which are more easily perturbed.

Dynamic disorder (time-dependent site energy fluctuations) with correlation times $\tau_{\rm corr} = \SIrange{50}{200}{fs} $ further reduces quantum advantage by \SIrange{10}{15}{\percent}, yielding net enhancements of \SIrange{15}{18}{\percent} under combined static and dynamic disorder. This remains significant for practical applications and confirms that quantum spectral engineering provides robust benefits even when accounting for the full complexity of biological environments.

\begin{figure}[ht]
\centering
\includegraphics[width=\textwidth]{Graphics/ETR_Under_Environmental_Effects.pdf}
\caption{Environmental robustness of quantum advantage. (a) Temperature dependence (\SIrange{280}{320}{K}) showing \SIrange{18}{26}{\percent} ETR enhancement across physiological range. (b) Static disorder tolerance: quantum advantage persists despite \SI{20}{\percent} reduction at $\sigma = \SI{50}{\per\cm}$ (typical protein disorder). (c) Geographic applicability across climate zones. Error bars represent \SI{95}{\percent} confidence intervals from ensemble averaging.}
\label{fig:robustness}
\end{figure}

\subsection{Comprehensive Validation: 12/12 Success}

Our validation suite achieved \SI{100}{\percent} success across all 12 independent tests, establishing confidence in quantitative predictions (\cref{tab:validation}).

\textbf{Convergence Tests} (4/4 passed):
\begin{itemize}
\item \textbf{HEOM benchmark}: \SI{1.8}{\percent} maximum deviation for 3-site system across all observables
\item \textbf{Matsubara cutoff}: Observables stable to $<$\SI{0.3}{\percent} for $N_{\rm Mat} \geq 10$
\item \textbf{Time step}: Results invariant ($<$\SI{0.1}{\percent} change) for $\Delta t$ from \SIrange{0.5}{2.0}{fs}
\item \textbf{Hierarchy truncation}: $<$\SI{0.8}{\percent} variation for thresholds $10^{-7}$ to $10^{-9}$
\end{itemize}

\textbf{Physical Consistency Tests} (4/4 passed):
\begin{itemize}
\item \textbf{Trace preservation}: $|\Tr(\rho) - 1| < \num{5e-13}$ (machine precision)
\item \textbf{Positivity}: All eigenvalues $> \num{-2e-11}$ (numerical noise floor)
\item \textbf{Energy conservation}: \SI{0.08}{\percent} drift over \SI{100}{ps} in closed-system test
\item \textbf{Detailed balance}: Equilibrium populations match Boltzmann within \SI{0.6}{\percent}
\end{itemize}

\textbf{Environmental Robustness Tests} (4/4 passed):
\begin{itemize}
\item \textbf{Temperature variations}: Quantum advantage \numrange{0.18}{0.30} for \SIrange{285}{305}{K} (within \SI{15}{\percent} of \SI{295}{K} value)
\item \textbf{Static disorder}: Enhancement reduced to \num{0.20} for $\sigma = \SI{50}{\per\cm}$ (\SI{18}{\percent} reduction, within acceptable range)
\item \textbf{Bath parameters}: Qualitative features preserved for $\pm\SI{20}{\percent}$ variations in all spectral density parameters
\item \textbf{Markovian limit}: Agreement with Redfield theory ($<$\SI{2}{\percent} deviation) at high temperature ($T > \SI{500}{K}$)
\end{itemize}

The perfect validation record distinguishes this work from prior studies reporting quantum effects that later proved to be numerical artifacts or model-dependent. The HEOM benchmarking is particularly critical: as HEOM is independently established as numerically exact for open quantum systems, agreement confirms that adHOPS predictions are quantitatively reliable, not merely qualitatively suggestive.

The Markovian limit recovery test verifies correct implementation: at sufficiently high temperatures where environmental correlation times become much shorter than system dynamics, non-Markovian methods must converge to Markovian results. Our \SI{2}{\percent} agreement confirms this, while the \SI{25}{\percent} enhancement at \SI{295}{K} demonstrates that physiological temperatures lie firmly in the non-Markovian regime where environmental memory is essential.

\subsection{Geographic and Climatic Applicability}

To assess global deployment potential, we simulated performance across diverse climatic zones using location-specific solar spectra and temperature profiles for representative sites: temperate (Germany, 50°N), subtropical (India, 20°N), tropical (Kenya, 0°), and desert (Arizona, USA, 32°N).

Results show consistent quantum advantages (\SIrange{18}{26}{\percent}) across all climates, with subtropical and tropical zones exhibiting slightly higher enhancements due to more stable year-round temperatures near the optimal \SI{295}{K}. Desert implementations show \SIrange{15}{20}{\percent} enhancement despite higher average temperatures (\SIrange{305}{315}{K}), benefiting from intense direct sunlight that better matches narrow-band filtering.

Seasonal analysis for temperate zones reveals:
\begin{itemize}
\item Winter (Jan--Feb, \SIrange{275}{285}{K}): $\eta_{\rm quantum} = \numrange{0.22}{0.26}$
\item Spring/Fall (Mar--May, Sep--Nov, \SIrange{285}{300}{K}): $\eta_{\rm quantum} = \numrange{0.24}{0.28}$
\item Summer (Jun--Aug, \SIrange{295}{310}{K}): $\eta_{\rm quantum} = \numrange{0.18}{0.24}$
\end{itemize}

The year-round viability across latitudes demonstrates that quantum spectral engineering is not limited to specific geographic regions but provides universal benefits for global agrivoltaic deployment.
