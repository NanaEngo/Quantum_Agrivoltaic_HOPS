% Results Section - EES Version
% Quantum Spectral Engineering for Enhanced Agrivoltaic Efficiency

\section{Results}\label{sec:Results}

\subsection{Quantum enhancement of electron transport rate}

Systematic optimisation of the OPV transmission function $T(\omega)$ shows that strategic spectral filtering enhances the photosynthetic electron transport rate (ETR) by up to \num{25}\% relative to Markovian models under identical photon flux conditions (\cref{fig:quantum_advantage}). This enhancement arises from vibronic resonance-assisted transport---a genuinely non-Markovian effect absent from classical spectral optimisation.

\begin{figure}[ht]
\centering
\includegraphics[width=0.85\textwidth]{Graphics/Quantum_Advantage_in_Energy_Transfer.pdf}
\caption{Quantum enhancement of electron transport rate through strategic spectral filtering. Optimised transmission windows (\numlist{750;820}\,nm dual-band) align with vibronic resonances in the photosynthetic complex, enabling coherence-assisted energy transfer at \SI{295}{K}. The \num{25}\% improvement over classical models reflects genuine non-Markovian quantum effects.}
\label{fig:quantum_advantage}
\end{figure}

Maximum quantum advantage occurs when the transmitted light targets the \SI{575}{\per\cm} vibronic mode, corresponding to transmission windows centred at $\lambda_c \approx \SI{750}{nm}$ (\SI{13333}{\per\cm}) and $\lambda_c \approx \SI{820}{nm}$ (\SI{12195}{\per\cm}). Under these conditions, the resonance matching criterion,
\begin{equation}\label{eq:resonance_condition}
\omega_{\rm filter} \approx \omega_{\rm vibronic} \pm J_{nm}
\end{equation}
is satisfied, and the transmission profile selectively excites excitonic states that couple to vibrational modes, creating dressed polaron-like states with reduced dephasing rates. The non-Markovian environment then sustains electronic coherence over timescales comparable to energy transfer times, enabling constructive interference effects that accelerate transport to the reaction centre.

\subsection{Coherence dynamics under spectral filtering}

The $l_1$-norm of coherence (\cref{eq:l1_coherence}) reveals that optimal spectral filtering extends coherence lifetimes by \numrange{20}{50}\% compared to broadband illumination (\cref{fig:coherence}). Under optimal filtering, $\tau_c$ exceeds \SI{500}{fs} at \SI{295}{K}, compared to $\sim$\SI{300}{fs} under broadband conditions. This extension persists when normalised to equal absorbed photon flux, confirming that spectral quality---not merely reduced intensity---determines quantum transport efficiency.

The exciton delocalization length, quantified by the inverse participation ratio $\xi_{\rm deloc}$ (\cref{eq:IPR}), increases from $N_{\rm eff} \approx 4$ under broadband illumination to $N_{\rm eff} \approx 9$ under optimised filtering. This enhanced delocalization allows excitations to access a larger set of pathways to the reaction centre through quantum interference. Importantly, this delocalization is maintained at physiological temperature, not only under cryogenic conditions.

The underlying mechanism is vibronic resonance matching: selective excitation of states quasi-resonant with vibrational modes promotes effective polaron formation with modified energy transfer dynamics. The resulting dressed states exhibit reduced dephasing because the filter suppresses decoherence-inducing frequencies while preserving coherent pathways. Time-resolved analysis shows oscillatory exciton population dynamics at vibronic mode energies---a signature of coherent vibronic coupling---persisting for hundreds of femtoseconds.

\begin{figure}[ht]
\centering
\includegraphics[width=0.85\textwidth]{Graphics/Figure_3.png}
\caption{\textbf{Coherence dynamics under optimal spectral filtering.} (a) Temporal evolution of $l_1$-norm coherence showing \numrange{20}{50}\% lifetime extension under filtered illumination (\numlist{750;820}\,nm dual-band) relative to broadband. (b) Inverse participation ratio demonstrating spatial spread from 3--5 to 8--10 chromophores. (c) Spectral density components showing alignment between filter-modified bath and FMO excitonic transitions. (d) System-bath correlation function illustrating non-Markovian memory effects. All data at \SI{295}{K} with realistic disorder ($\sigma = \SI{50}{\per\cm}$).}
\label{fig:coherence}
\end{figure}

\cref{tab:quantum_metrics} summarises the quantitative comparison of quantum metrics between filtered and broadband illumination.

% Quantum Metrics Comparison Table
\begin{table}[ht]
\centering
\caption{\textbf{Quantum metrics comparison: spectral filtering vs.\ broadband illumination.} All measurements at \SI{295}{K} with realistic static disorder ($\sigma = \SI{50}{\per\cm}$). Filtered condition: optimised dual-band transmission (\SI{750}{nm}/\SI{820}{nm}, FWHM \num{70}\,nm, \SIrange{65}{75}{\%} peak transmission). Errors are \SI{95}{\%} confidence intervals from 500 disorder realisations.}
\label{tab:quantum_metrics}
\begin{tabular}{lccc}
\toprule
\textbf{Metric} & \textbf{Filtered (750/820 nm)} & \textbf{Broadband} & \textbf{Enhancement} \\
\midrule
ETR (relative) & \num{1.25 \pm 0.03} & \num{1.00 \pm 0.02} & \SI{+25}{\%} \\
Coherence lifetime (fs) & \num{420 \pm 35} & \num{280 \pm 25} & \SI{+50}{\%} \\
Delocalization (sites) & \num{8.2 \pm 0.7} & \num{4.1 \pm 0.5} & \SI{+100}{\%} \\
QFI (max) & \num{12.4 \pm 1.1} & \num{7.8 \pm 0.8} & \SI{+59}{\%} \\
Purity ($t = \SI{500}{fs}$) & \num{0.82 \pm 0.04} & \num{0.71 \pm 0.05} & \SI{+15}{\%} \\
Von Neumann entropy & \num{0.51 \pm 0.06} & \num{0.73 \pm 0.07} & \SI{-30}{\%}$^*$ \\
\bottomrule
\multicolumn{4}{l}{\scriptsize $^*$Lower entropy indicates more ordered quantum state (beneficial).} \\
\end{tabular}
\end{table}

The improvements are mutually reinforcing: extended coherence enables greater delocalization, which facilitates the \num{25}\% ETR enhancement. The QFI increase (\SI{59}{\%}) indicates improved sensitivity to system parameters, correlating with enhanced transport efficiency. The purity increase and entropy decrease confirm that filtered illumination maintains more coherent quantum states throughout the energy transfer process.

\subsection{Pareto optimisation: Balancing energy and agriculture}

Multi-objective optimisation reveals a well-defined Pareto frontier for the PCE--ETR trade-off (\cref{fig:pareto}). Three configurations span the design space:

The \textbf{balanced configuration} (recommended for most deployments) achieves \SIrange{16}{18}{\%} PCE with \SIrange{15}{20}{\%} ETR enhancement, using dual-band transmission at \SI{750}{nm} and \SI{820}{nm} (FWHM \num{70}\,nm, peak transmission \SIrange{65}{75}{\%}). The \textbf{energy-focused configuration} maximises PCE (\SIrange{19}{21}{\%}) at the cost of reduced ETR enhancement (\SIrange{8}{12}{\%}), using a single narrower band (FWHM \SI{50}{nm}). The \textbf{agriculture-focused configuration} maximises ETR enhancement (\SIrange{22}{25}{\%}) with minimum viable PCE (\SIrange{13}{15}{\%}), using dual broad bands (FWHM \SIrange{90}{100}{nm}).

\begin{figure}[ht]
\centering
\includegraphics[width=0.75\textwidth]{Graphics/Pareto_Front__PCE_vs_ETR_Trade_off.pdf}
\caption{Pareto frontier from multi-objective optimisation showing the PCE--ETR trade-off. Three optimal configurations span the design space: Balanced (\SIrange{16}{18}{\%} PCE, \SIrange{15}{20}{\%} ETR), Energy-focused (\SIrange{19}{21}{\%} PCE, \SIrange{8}{12}{\%} ETR), and Agriculture-focused (\SIrange{13}{15}{\%} PCE, \SIrange{22}{25}{\%} ETR). All maintain commercially viable performance.}
\label{fig:pareto}
\end{figure}

The frontier shows that significant quantum advantages (\SIrange{15}{25}{\%} ETR enhancement) are achievable while maintaining PCE above \SI{15}{\%}. For a representative 1-hectare installation with high-value crops, even \SI{15}{\%} ETR improvement translates to USD~\numrange{3000}{5000} additional annual agricultural revenue, partially offsetting the $\sim$\SI{10}{\%} reduction in electrical revenue from operating at \SI{16}{\%} rather than \SI{20}{\%} PCE. A detailed economic analysis is presented in Section~4.

\subsection{Environmental robustness}

The quantum advantage persists across physiologically relevant conditions (\cref{fig:robustness}). Temperature dependence is non-monotonic, with maximum coherence preservation at \SIrange{285}{300}{K}---a range that coincidentally encompasses typical temperate-climate agricultural conditions. At \SI{295}{K}, $\eta_{\rm quantum} = 0.25$; even under moderate heat stress (\SI{310}{K}), $\eta_{\rm quantum} = 0.18$. The non-monotonic behaviour reflects a balance: thermal energy must populate vibronic modes that mediate coherent transport without inducing excessive dephasing.

Static energetic disorder ($\sigma = \SI{50}{\per\cm}$, typical of biological systems) reduces the quantum advantage by approximately \SI{20}{\%}, but significant enhancement (\SIrange{18}{20}{\%}) persists. Ensemble averaging over 100 disorder realisations yields $\langle\eta_{\rm quantum}\rangle = \num{0.20 \pm 0.04}$, with a coefficient of variation below \SI{20}{\%}, indicating that quantum enhancement is statistically robust. Even at extreme disorder ($\sigma = \SI{100}{\per\cm}$), \SIrange{12}{15}{\%} enhancement remains. This robustness arises because vibronic resonance conditions depend primarily on intramolecular mode frequencies---determined by bond properties largely insensitive to environmental fluctuations---rather than precise site energies.

Combined static and dynamic disorder (correlation times $\tau_{\rm corr} = \SIrange{50}{200}{fs}$) yields net enhancements of \SIrange{15}{18}{\%}, still meaningful for practical applications.

\begin{figure}[ht]
\centering
\includegraphics[width=\textwidth]{Graphics/ETR_Under_Environmental_Effects.pdf}
\caption{Environmental robustness of quantum advantage. (a) Temperature dependence (\SIrange{280}{320}{K}) showing \numrange{18}{26}\% ETR enhancement across the physiological range. (b) Static disorder tolerance: quantum advantage persists despite \SI{20}{\%} reduction at $\sigma = \SI{50}{\per\cm}$ (typical protein disorder). (c) Geographic applicability across climate zones. Error bars: \SI{95}{\%} confidence intervals.}
\label{fig:robustness}
\end{figure}

\subsection{Validation results}

The 12-test validation suite achieved \SI{100}{\%} success across all categories (see Table~S3 in the Supporting Information). Key results include: HEOM benchmark agreement to \SI{1.8}{\%} for 3-site systems; trace preservation to $|{\rm Tr}(\rho) - 1| < \num{5e-13}$; and recovery of the Markovian limit (Redfield theory) to within \SI{2}{\%} at high temperature ($T > \SI{500}{K}$). Full details are provided in Section~S3 of the Supporting Information.

This Markovian limit recovery is informative: at high temperatures, environmental correlation times become much shorter than system dynamics, so non-Markovian methods must converge to Markovian results. The \num{2}\% agreement confirms correct implementation, while the \num{25}\% enhancement at \num{295}\,K shows that physiological temperatures lie firmly in the non-Markovian regime where environmental memory matters.

\subsection{Geographic and climatic applicability}

Simulations across diverse climatic zones---temperate (Germany, \num{50}\,\si{\degree}N), subtropical (India, \num{20}\,\si{\degree}N), tropical (Kenya, \num{0}\,\si{\degree}), and desert (Arizona, USA, \num{32}\,\si{\degree}N)---using location-specific solar spectra and temperature profiles show consistent quantum advantages of \numrange{18}{26}\% across all climates. Subtropical and tropical zones exhibit slightly higher enhancements due to stable year-round temperatures near the \num{295}\,K optimum. Even desert implementations show \numrange{15}{20}\% enhancement despite elevated temperatures (\numrange{305}{315}\,K).

Seasonal analysis for temperate zones shows $\eta_{\rm quantum}$ ranges of \numrange{0.22}{0.26} in winter, \numrange{0.24}{0.28} in spring/autumn, and \numrange{0.18}{0.24} in summer. This year-round viability across latitudes indicates that spectral bath engineering provides benefits for global agrivoltaic deployment, with direct relevance to food security and clean energy targets in both developed and developing regions.
