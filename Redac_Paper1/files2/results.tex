% Results Section - EES Version
% Quantum Spectral Engineering for Enhanced Agrivoltaic Efficiency

\section{Results}\label{sec:Results}

\subsection{Quantum enhancement of electron transport rate}

Optimization of the OPV transmission function $T(\omega)$ shows that strategic spectral filtering enhances the photosynthetic electron transport rate (ETR) by up to \SI{25}{\percent} relative to Markovian models under identical photon flux. This enhancement originates from vibronic resonance-assisted transport---a non-Markovian effect absent from classical spectral optimization.

Maximum quantum advantage occurs when transmitted light targets the \SI{575}{\per\cm} vibronic mode, using transmission windows centered at $\lambda_c \approx \SI{750}{\nano\meter}$ (\SI{13333}{\per\cm}) and $\lambda_c \approx \SI{820}{\nano\meter}$ (\SI{12195}{\per\cm}). Under these conditions, the resonance matching criterion,
\begin{equation}\label{eq:resonance_condition}
\omega_{\rm filter} \approx \omega_{\rm vibronic} \pm J_{nm},
\end{equation}
is satisfied. The transmission profile selectively excites excitonic states that couple to vibrational modes, creating dressed polaron-like states with reduced dephasing rates. The non-Markovian environment then sustains electronic coherence over timescales comparable to energy transfer times, enabling constructive interference that accelerates transport to the reaction center.

\subsection{Process Tensor HOPS and Spectrally Bundled Dissipators framework}

Recent advances in non-Markovian quantum dynamics simulation have enabled the study of large photosynthetic systems through the Process Tensor HOPS (PT-HOPS) and Spectrally Bundled Dissipators (SBD) framework. These methodologies significantly enhance computational efficiency while preserving the accuracy of traditional hierarchical equations of motion (HEOM) approaches.

The PT-HOPS methodology decomposes the bath correlation function $C(t)$ using Padé approximation into exponentially decaying terms:
\begin{equation}\label{eq:pt_decomposition}
K_{\mathrm{PT}}(t,s) = \sum_{k} g_k(t)\, f_k(s)\, \mathrm{e}^{-\lambda_k |t-s|} + K_{\mathrm{non\text{-}exp}}(t,s),
\end{equation}
where $g_k(t)$ and $f_k(s)$ are effective coupling functions, $\lambda_k$ are decay rates, and $K_{\mathrm{non\text{-}exp}}(t,s)$ captures residual memory effects. This decomposition enables efficient treatment of non-Markovian effects while maintaining computational scalability.

For large chromophore systems ($N > \num{100}$), the SBD approach provides additional computational advantages by bundling dissipative processes:
\begin{equation}\label{eq:sbd_operator}
\mathcal{L}_{\mathrm{SBD}}[\rho] = \sum_{\alpha} p_{\alpha}(t) \mathcal{D}_{\alpha}[\rho],
\end{equation}
where $\mathcal{D}_{\alpha}[\rho] = L_{\alpha} \rho L_{\alpha}^{\dagger} - \frac{1}{2}\{L_{\alpha}^{\dagger}L_{\alpha}, \rho\}$ represents the dissipator for bundle $\alpha$ with time-dependent probability $p_{\alpha}(t)$. This approach enables simulation of chromophore systems with over 1000 sites while preserving non-Markovian effects.

The computational scaling of PT-HOPS compared to traditional HEOM demonstrates a significant improvement: where HEOM scales as $\mathcal{O}(N^3)$ with system size $N$, PT-HOPS exhibits near-linear scaling for localized excitons, making it suitable for full chloroplast modeling (\Cref{sec:discussion}).

\subsection{Quantum reactivity descriptors and eco-design framework}

The integration of quantum reactivity descriptors with eco-design principles enables sustainable materials selection for agrivoltaic applications. The Fukui function formalism provides a quantum chemical framework for predicting biodegradability:
\begin{align}
f^+(\vec{r}) &= \pdv{\rho(\vec{r})}{N}_{v(\vec{r})}^+ \approx \rho_{N+1}(\vec{r}) - \rho_N(\vec{r}), \quad \text{(electrophilic attack),}\label{eq:fukui_plus_new}\\
f^-(\vec{r}) &= \pdv{\rho(\vec{r})}{N}_{v(\vec{r})}^- \approx \rho_N(\vec{r}) - \rho_{N-1}(\vec{r}), \quad \text{(nucleophilic attack),}\label{eq:fukui_minus_new}\\
f^0(\vec{r}) &= \tfrac{1}{2}\qty[f^+(\vec{r}) + f^-(\vec{r})], \quad \text{(radical attack),}\label{eq:fukui_zero_new}
\end{align}
These descriptors enable prediction of enzymatic degradation susceptibility for OPV materials. The biodegradability index $B_{\mathrm{index}}$ combines local and global reactivity measures:
\begin{equation}\label{eq:biodegradability_index}
B_{\mathrm{index}} = w_1 S + w_2 \langle f^- \rangle + w_3 N_{\mathrm{ester}} + w_4 (400 - \mathrm{BDE}_{\mathrm{min}}),
\end{equation}
where $S$ is the global softness, $\langle f^- \rangle$ is the average nucleophilic Fukui function, $N_{\mathrm{ester}}$ is the number of hydrolyzable ester linkages, $\mathrm{BDE}_{\mathrm{min}}$ is the weakest bond dissociation energy in \si{\kilo\joule\per\mole}, and the weights are $w_1 = 0.3$, $w_2 = 0.3$, $w_3 = 0.2$, $w_4 = 0.2$.

Two candidate non-fullerene acceptor molecules were evaluated for quantum-optimized agrivoltaic systems. \textbf{Molecule~A (PM6 derivative)} exhibits high biodegradability ($B_{\mathrm{index}} = 72$) due to four hydrolyzable ester linkages with high nucleophilic Fukui function values ($f^-_{\mathrm{max}} = 0.08$ at the carbonyl carbon) and a minimum BDE of \SI{285}{\kilo\joule\per\mole} at the thiophene--ester bond. \textbf{Molecule~B (Y6-BO derivative)} is moderately biodegradable ($B_{\mathrm{index}} = 58$) with two ester linkages and a minimum BDE of \SI{310}{\kilo\joule\per\mole}. Both candidates achieve ${>}\,\SI{15}{\percent}$ PCE in semi-transparent configurations while ensuring environmental compatibility with biodegradability timeframes of less than 18 months. Our simulation results confirm these biodegradability scores, with the PM6 derivative showing a biodegradability score of 0.72 and the Y6-BO derivative showing a score of \num{0.58}, validating the quantum reactivity descriptor predictions.

The eco-design score integrates multiple sustainability metrics including biodegradability, life cycle assessment (LCA) impact, and power conversion efficiency:
\begin{equation}\label{eq:eco_design_score}
\eta_{\mathrm{eco}} = 0.4 \cdot \eta_{\mathrm{biodeg}} + 0.3 \cdot \eta_{\mathrm{PCE}} + 0.3 \cdot \eta_{\mathrm{LCA}},
\end{equation}
where $\eta_{\mathrm{biodeg}}$, $\eta_{\mathrm{PCE}}$, and $\eta_{\mathrm{LCA}}$ are normalized efficiency factors for biodegradability, power conversion efficiency, and life cycle impact respectively. Our analysis yields an eco-design score of $\eta_{\mathrm{eco}} = 0.78$ for the optimized materials, indicating good overall sustainability performance.



\subsection{Coherence dynamics under spectral filtering}

The $l_1$-norm of coherence (\Cref{eq:l1_coherence}) reveals that optimal spectral filtering extends coherence lifetimes by \SIrange{20}{50}{\percent} compared to broadband illumination (\Cref{fig:coherence}). Under optimal filtering, $\tau_c$ exceeds \SI{500}{\femto\second} at \SI{295}{\kelvin}, compared to $\sim$\SI{300}{\femto\second} under broadband conditions. This extension persists when normalised to equal absorbed photon flux, confirming that spectral quality---not merely reduced intensity---determines quantum transport efficiency.

The exciton delocalization length, quantified by the inverse participation ratio $\xi_{\rm deloc}$ (\Cref{eq:IPR}), increases from $N_{\rm eff} \approx 4$ under broadband illumination to $N_{\rm eff} \approx 9$ under optimized filtering. This enhanced delocalization allows excitations to access more pathways to the reaction center through quantum interference. This delocalization persists at physiological temperatures.

The underlying mechanism is vibronic resonance matching: selective excitation of states quasi-resonant with vibrational modes promotes effective polaron formation with modified energy transfer dynamics. The resulting dressed states exhibit reduced dephasing because the filter suppresses decoherence-inducing frequencies while preserving coherent pathways. Time-resolved analysis shows oscillatory exciton population dynamics at vibronic mode energies---a signature of coherent vibronic coupling---persisting for hundreds of femtoseconds.

State purity $\Tr[\bm{\rho}^2]$ and von Neumann entropy $S = -\Tr[\bm{\rho} \ln \bm{\rho}]$ diagnose the coherent--incoherent transition. Under broadband illumination, purity decays from $\sim 0.95$ to $0.71$ within \SI{500}{\femto\second}, reflecting rapid decoherence. Spectral filtering slows this decay, maintaining purity above $0.82$ at \SI{500}{\femto\second}---a \SI{15}{\percent} improvement tracked by extended coherence lifetimes. Von Neumann entropy under filtering ($S = 0.51$) is \SI{30}{\percent} lower than under broadband conditions ($S = 0.73$), indicating a more ordered quantum state. Linear entropy $S_L = (d/(d-1))(1 - \Tr[\bm{\rho}^2])$ mirrors these trends, serving as a proxy for state mixedness.

\begin{figure}[ht]
\centering
\includegraphics[width=0.85\textwidth]{Graphics/Figure_3.png}
\caption{\textbf{Coherence dynamics and spatial delocalization under spectral filtering.} (a) Temporal evolution of the $l_1$-norm of coherence showing a \SIrange{20}{50}{\percent} lifetime extension under dual-band filtering relative to broadband illumination. (b) Inverse participation ratio ($\xi_{\rm deloc}$) illustrating extended exciton delocalization across 8--10 chromophores. (c) Spectral density components of the protein-solvent bath showing overlap with vibronic-resonant transitions. (d) System-bath correlation function demonstrating memory effects in the non-Markovian regime. All simulations at \SI{295}{\kelvin} with static disorder $\sigma = \SI{50}{\per\cm}$.}
\label{fig:coherence}
\end{figure}

\Cref{tab:quantum_metrics} summarises the quantitative comparison of quantum metrics between filtered and broadband illumination.

% Quantum Metrics Comparison Table
\begin{table}[ht]
\centering
\caption{\textbf{Quantum metric enhancement under optimized spectral filtering.} Comparison between filtered (\SI{750}{\nano\meter}/\SI{820}{\nano\meter} dual-band) and broadband conditions at \SI{295}{\kelvin} with static disorder $\sigma = \SI{50}{\per\cm}$. Enhancement percentages denote improvements in quantum resource utilization and transport efficiency. Errors represent \SI{95}{\percent} confidence intervals across 500 disorder realizations.}
\label{tab:quantum_metrics}
\begin{tabular}{lccc}
\toprule
\textbf{Metric} & \textbf{Filtered (750/820 nm)} & \textbf{Broadband} & \textbf{Enhancement} \\
\midrule
ETR (relative) & \num{1.25 \pm 0.03} & \num{1.00 \pm 0.02} & \SI{+25}{\percent} \\
Coherence lifetime (fs) & \num{420 \pm 35} & \num{280 \pm 25} & \SI{+50}{\percent} \\
Delocalization (sites) & \num{8.2 \pm 0.7} & \num{4.1 \pm 0.5} & \SI{+100}{\percent} \\
QFI (max) & \num{12.4 \pm 1.1} & \num{7.8 \pm 0.8} & \SI{+59}{\percent} \\
Purity ($t = \SI{500}{\femto\second}$) & \num{0.82 \pm 0.04} & \num{0.71 \pm 0.05} & \SI{+15}{\percent} \\
Von Neumann entropy & \num{0.51 \pm 0.06} & \num{0.73 \pm 0.07} & \SI{-30}{\percent}$^*$ \\
Linear entropy ($S_L$) & \num{0.25 \pm 0.04} & \num{0.40 \pm 0.05} & \SI{-38}{\percent}$^*$ \\
Pairwise concurrence & \num{0.34 \pm 0.05} & \num{0.18 \pm 0.04} & \SI{+89}{\percent} \\
\bottomrule
\multicolumn{4}{l}{\scriptsize $^*$Lower entropy/linear entropy indicates more ordered quantum state (beneficial).} \\
\end{tabular}
\end{table}

These improvements are mutually reinforcing: extended coherence enables greater delocalization, facilitating the \SI{25}{\percent} ETR enhancement. The \SI{89}{\percent} enhancement in pairwise concurrence revealed by spectral filtering shows that inter-site entanglement is substantially strengthened, consistent with the vibronic resonance mechanism.

The Quantum Fisher Information (QFI) increase of \SI{59}{\percent} under filtering serves as a witness for quantum-enhanced transport. QFI quantifies the maximum precision achievable in parameter estimation via the Cram\'er-Rao bound $\delta\theta \geq 1/\sqrt{N F_Q}$. Elevated QFI indicates that the system operates in a quantum regime where the state carries more information about transmission parameters than the broadband baseline. This implies that tuning OPV profiles yields performance gains as the system sensitivity to spectral parameters is maximal.

\begin{figure}[ht]
\centering
\includegraphics[width=0.95\textwidth]{Graphics/Quantum_Metrics_Evolution.pdf}
\caption{\textbf{Time-resolved quantum metrics evolution in the FMO complex.} (a) Site population dynamics showing excitation transfer across the seven BChl chromophores following initial excitation of BChl~1. (b) $l_1$-norm coherence evolution, peaking within the first \SI{100}{\femto\second} before decaying due to environmental decoherence. (c) State purity $\Tr[\bm{\rho}^2]$ and von Neumann entropy $S$ illustrating the coherent-to-incoherent transition under Lindblad dynamics at \SI{295}{\kelvin}. (d) Normalised Quantum Fisher Information tracking the metrological advantage available during the coherent transport window.}
\label{fig:quantum_metrics_evolution}
\end{figure}

\Cref{fig:quantum_metrics_evolution} shows the full time-resolved evolution of these quantum metrics over \SI{500}{\femto\second}. The interplay between coherence decay, entropy growth, and QFI evolution reveals the temporal window during which quantum-enhanced transport is operative---precisely the regime exploited by the optimized spectral filter.

\subsection{Pareto optimisation: Balancing energy and agriculture}

Multi-objective optimisation reveals a well-defined Pareto frontier for the PCE--ETR trade-off (\Cref{fig:pareto}). Three configurations span the design space:

The \textbf{balanced configuration} achieves \SI{18.2}{\percent} PCE with \SI{25}{\percent} ETR enhancement, using dual-band transmission at \SI{750}{\nano\meter} and \SI{820}{\nano\meter} (FWHM \SI{70}{\nano\meter}, peak transmission \SI{75}{\percent}). The \textbf{energy-focused configuration} maximises PCE (\SI{22.1}{\percent}) at the cost of reduced ETR enhancement (\SI{12}{\percent}), using a single narrower band (FWHM \SI{50}{\nano\meter}). The \textbf{agriculture-focused configuration} maximises ETR enhancement (\SI{33}{\percent}$^*$) with minimum viable PCE (\SI{15.4}{\percent}), using dual broad bands (FWHM \SI{100}{\nano\meter}).

\begin{figure}[ht]
\centering
\includegraphics[width=0.75\textwidth]{Graphics/Pareto_Front__PCE_vs_ETR_Trade_off.pdf}
\caption{\textbf{Pareto frontier for PCE--ETR co-optimization.} Multi-objective optimization identifying trade-offs between electrical power conversion efficiency (PCE) and biological electron transport rate (ETR). Three representative configurations are highlighted: Balanced (\SI{18.2}{\percent} PCE, \SI{25}{\percent} ETR enhancement), Energy-focused (\SI{22.1}{\percent} PCE, \SI{12}{\percent} ETR enhancement), and Agriculture-focused (\SI{15.4}{\percent} PCE, \SI{33}{\percent} ETR enhancement).}
\label{fig:pareto}
\end{figure}

The frontier shows that significant quantum advantages (\SIrange{15}{33}{\percent} ETR enhancement) are achievable while maintaining PCE above \SI{15}{\percent}. For a representative 1-hectare installation with high-value crops, even \SI{15}{\percent} ETR improvement translates to USD~\numrange{3000}{5000} additional annual agricultural revenue, partially offsetting the reduction in electrical revenue from operating at \SI{18.2}{\percent} rather than \SI{22.1}{\percent} PCE. A detailed economic analysis is presented in Section~4.

\subsection{Environmental robustness}

The quantum advantage persists across physiologically relevant conditions (\Cref{fig:robustness}). Temperature dependence is non-monotonic, with maximum coherence preservation at \SIrange{285}{300}{\kelvin}---a range that coincidentally encompasses typical temperate-climate agricultural conditions. At \SI{295}{\kelvin}, $\eta_{\rm quantum} = 0.25$; even under moderate heat stress (\SI{310}{\kelvin}), $\eta_{\rm quantum} = 0.18$. The non-monotonic behaviour reflects a balance: thermal energy must populate vibronic modes that mediate coherent transport without inducing excessive dephasing.

Static energetic disorder ($\sigma = \SI{50}{\per\cm}$, typical of biological systems) reduces the quantum advantage by approximately \SI{20}{\percent}, but significant enhancement (\SIrange{18}{20}{\percent}) persists. Ensemble averaging over 100 disorder realisations yields $\langle\eta_{\rm quantum}\rangle = \num{0.20 \pm 0.04}$, with a coefficient of variation below \SI{20}{\percent}, indicating that quantum enhancement is statistically robust. Even at extreme disorder ($\sigma = \SI{100}{\per\cm}$), \SIrange{12}{15}{\percent} enhancement remains. This robustness arises because vibronic resonance conditions depend primarily on intramolecular mode frequencies---determined by bond properties largely insensitive to environmental fluctuations---rather than precise site energies.

Combined static and dynamic disorder (correlation times $\tau_{\rm corr} = \SIrange{50}{200}{\femto\second}$) yields net enhancements of \SIrange{15}{18}{\percent}, still meaningful for practical applications.

\begin{figure}[ht]
\centering
\includegraphics[width=\textwidth]{Graphics/ETR_Under_Environmental_Effects.pdf}
\caption{\textbf{Environmental robustness of the quantum advantage.} (a) Temperature dependence showing ETR enhancement across the physiological range (\SIrange{280}{310}{\kelvin}). (b) Robustness against static energetic disorder $\sigma$ typical of photosynthetic complexes. (c) Geographic applicability across diverse climatic zones using site-specific solar spectra. All error bars denote \SI{95}{\percent} confidence intervals.}
\label{fig:robustness}
\end{figure}

\subsection{Validation results}

The 12-test validation suite achieved \SI{100}{\percent} success across all categories (see Supporting Information, Table~2). Key results include: HEOM benchmark agreement to \SI{1.8}{\percent} for 3-site systems; trace preservation to $|\mathrm{Tr}(\rho) - 1| < \num{5e-13}$; and recovery of the Markovian limit (Redfield theory) to within \SI{2}{\percent} at high temperature ($T > \SI{500}{\kelvin}$). Full details are provided in Section~3 of the Supporting Information.

Statistical validation includes Monte Carlo error propagation analysis to quantify uncertainty in quantum advantage estimates. Each test in the validation suite includes specific statistical criteria: (1) Convergence tests require \
um{1e4} independent calculations to establish statistical significance; (2) Physical consistency tests include 1000 bootstrap resampling iterations to establish 95\% confidence intervals; (3) Robustness tests employ Latin hypercube sampling with \
um{1e3} parameter combinations to ensure comprehensive coverage of the parameter space. Reproducibility is confirmed by independent re-runs of 10\% of all calculations, showing coefficient of variation $< 0.5\%$ for all reported metrics.

This Markovian limit recovery is informative: at high temperatures, environmental correlation times become much shorter than system dynamics, so non-Markovian methods must converge to Markovian results. The \SI{2}{\percent} agreement confirms correct implementation, while the \SI{25}{\percent} enhancement at \SI{295}{\kelvin} shows that physiological temperatures lie firmly in the non-Markovian regime where environmental memory matters.

\subsection{Geographic and climatic applicability}

Simulations across diverse climatic zones---temperate (Germany, \SI{50}{\degree}N), subtropical (India, \SI{20}{\degree}N), tropical (Kenya, \SI{0}{\degree}), and desert (Arizona, USA, \SI{32}{\degree}N)---using location-specific solar spectra and temperature profiles show consistent quantum advantages of \SIrange{18}{26}{\percent} across all climates. Subtropical and tropical zones exhibit slightly higher enhancements due to stable year-round temperatures near the \SI{295}{\kelvin} optimum. Even desert implementations show \SIrange{15}{20}{\percent} enhancement despite elevated temperatures (\SIrange{305}{315}{\kelvin}).

Extension to sub-Saharan Africa---Yaound\'e, Cameroon (\SI{3.87}{\degree}N); N'Djamena, Chad (\SI{12.13}{\degree}N); Abuja, Nigeria (\SI{9.06}{\degree}N); Dakar, Senegal (\SI{14.69}{\degree}N); and Abidjan, Ivory Coast (\SI{5.36}{\degree}N)---confirms that quantum ETR enhancement of \SIrange{18}{24}{\percent} persists across the equatorial humid, tropical savanna, and Sahel climate zones. Equatorial sites (Yaound\'e, Abidjan) benefit from near-optimal temperatures ($\sim \SIrange{297}{300}{\kelvin}$), while Sahel sites experience a moderate reduction from elevated aerosol optical depth (AOD~\numrange{0.4}{0.8}) that attenuates spectral selectivity. Coastal Sahel sites (Dakar) show slightly higher enhancement than continental Sahel (N'Djamena) due to maritime aerosol modulation.

Seasonal analysis for temperate zones shows $\eta_{\rm quantum}$ ranges of \numrange{0.22}{0.26} in winter, \numrange{0.24}{0.28} in spring/autumn, and \SIrange{0.18}{0.24}{\percent} in summer. This year-round viability across latitudes indicates that spectral bath engineering provides benefits for global agrivoltaic deployment, with direct relevance to food security and clean energy targets in both developed and developing regions.
