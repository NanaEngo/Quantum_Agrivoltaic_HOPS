% Main Manuscript File - EES Submission
% Quantum Spectral Engineering for Enhanced Agrivoltaic Efficiency

\documentclass[11pt,a4paper]{article}
\usepackage[utf8]{inputenc}
\usepackage[T1]{fontenc}
\usepackage{amsmath,amssymb,amsfonts}
\usepackage{bm}  % Bold math symbols
\usepackage{graphicx}
\usepackage{xcolor}
\usepackage{hyperref}
\usepackage{natbib}
\usepackage[margin=2.5cm]{geometry}

% Hyperref setup
\hypersetup{
    pdftitle={Quantum Spectral Engineering for Enhanced Agrivoltaic Efficiency},
    pdfauthor={Teguia et al.},
    pdfsubject={Quantum Photosynthesis, Agrivoltaics, Non-Markovian Dynamics},
    pdfkeywords={agrivoltaics, quantum photosynthesis, spectral engineering, non-Markovian dynamics, renewable energy},
    colorlinks=true,
    linkcolor=blue,
    citecolor=blue,
    urlcolor=blue
}

% Professional LaTeX packages for scientific formatting
\usepackage{cleveref}  % Smart cross-references (Eq. vs Eqs., Fig. vs Figs.)
\usepackage{siunitx}    % Proper SI unit formatting
\usepackage{physics}    % Physics notation (bra-ket, derivatives)

% Custom commands
\newcommand{\sI}[2]{#1\,#2}  % SI units formatting

%=============================================================================
% TITLE AND AUTHORS
%=============================================================================

\title{\textbf{Quantum Spectral Engineering for Enhanced Agrivoltaic Efficiency: Non-Markovian Dynamics in Photosynthetic Energy Transfer}}

\author{
    Steve Cabrel Teguia Kouam$^{2,*}$,
    Theodore Goumai Vodekoi$^{1}$, 
    Jean-Pierre Tchapet Njafa$^{1}$, \\
    Jean-Pierre Nguenang$^{2}$,
    Serge Guy Nana Engo$^{1}$ \\
    \\
    $^{1}$Department of Physics, Faculty of Science, University of Yaoundé I, Cameroon \\
    $^{2}$Department of Physics, Faculty of Science, University of Douala, Cameroon \\
    \\
    $^{*}$Corresponding author: \texttt{serge.nana-engo@facsciences-uy1.cm}
}

\date{\today}

%=============================================================================
% DOCUMENT BEGIN
%=============================================================================

\begin{document}

\maketitle

%=============================================================================
% ABSTRACT
%=============================================================================

\begin{abstract}
The integration of agricultural production with solar energy generation through agrivoltaic systems is often limited by design paradigms that treat light solely as a classical photon flux, overlooking the quantum mechanical nature of photosynthetic energy transfer. We present a framework for quantum spectral bath engineering, where strategic spectral filtering of sunlight through semi-transparent organic photovoltaic (OPV) panels is used to harness non-Markovian quantum coherence in photosynthesis. Using adaptive Hierarchy of Pure States (adHOPS) simulations of the Fenna-Matthews-Olsen complex, we demonstrate that selective filtering improves the electron transport rate by \SI{25}{\percent} via vibronic resonance-assisted transport. This approach extends coherence lifetimes by \SIrange{20}{50}{\percent} and increases exciton delocalization from \numrange{3}{5} to \numrange{8}{10} chromophores. Our results are validated through a suite of 12 numerical tests, confirming robustness under physiological temperatures (\SI{295}{K}), static disorder ($\sigma = \SI{50}{\per\cm}$), and environmental fluctuations. Pareto optimization identifies design windows that achieve \SIrange{16}{18}{\percent} power conversion efficiency while maintaining significant biological energy transfer enhancement. Economic modeling suggests that these quantum-informed designs can generate \$\numrange{470}{3000} in additional annual revenue per hectare. We provide quantitative materials specifications (dual-band transmission at \SIlist{750;820}{nm}) and specific predictions for experimental verification via ultrafast spectroscopy and field trials. This work demonstrates that quantum mechanical principles in light-harvesting systems can be directly translated into design rules for high-efficiency, sustainable energy technologies.
\end{abstract}

%=============================================================================
% KEYWORDS
%=============================================================================

\noindent\textbf{Keywords:} Agrivoltaics, Quantum photosynthesis, Spectral engineering, Non-Markovian dynamics, Renewable energy, Organic photovoltaics, Coherence-assisted transport, Sustainable agriculture

%=============================================================================
% MAIN CONTENT (MODULAR SECTIONS)
%=============================================================================

% Introduction Section - EES Version
% Quantum Spectral Engineering for Enhanced Agrivoltaic Efficiency

\section{Introduction}\label{sec:Introduction}

Growing demand for clean energy and food security has intensified competition for agricultural land \cite{Valle2017, Dupraz2011, Marrou2013}. Agrivoltaic systems---integrating crop production with semi-transparent photovoltaic (PV) panels---address this conflict by generating electricity and food on the same land, contributing to SDGs~2, 7 and~13 \cite{Weselek2019, Amaducci2018}. Current installations can reduce water usage by up to \num{30}\% while maintaining \num{90}\% of baseline crop yields \cite{Barron2018, Elamri2018}. However, existing designs optimise for total Photosynthetically Active Radiation (PAR) flux, treating light as a classical radiative input \cite{MaLu2025, Shugar2021}.

This approach overlooks a key aspect of photosynthesis: energy transfer in pigment-protein complexes is a quantum process governed by non-Markovian dynamics, where coherence and structured environmental fluctuations assist transport \cite{Engel2007, Panitchayangkoon2010, Collini2010, mohs2008, tao2020, Blankenship2011, Scholes2011, Plenio2008, Sarovar2010, Huelga2013, Rebentrost2009}. In the intermediate coupling regime typical of biological light harvesting, Markovian approximations (e.g., Redfield theory) fail to capture important dynamical features \cite{Ishizaki2009, Kelly2016}, and photosynthetic efficiency depends sensitively on the spectral structure of both the pigment-protein complex and the incident light field \cite{Curutchet2016, Gelzinis2017}.

\subsection{Quantum photosynthesis and the FMO complex}

The Fenna-Matthews-Olsen (FMO) complex of green sulfur bacteria is a well-characterised model for quantum effects in photosynthesis \cite{Fenna1975, Renger2004}. Its trimeric structure exhibits long-lived quantum coherences \cite{Engel2007, Collini2010} and serves as a standard benchmark for quantum transport \cite{Mohseni2014, Hildner2013}, with each monomer containing 7--8 bacteriochlorophyll-a molecules that funnel energy from the chlorosome antenna to the reaction centre.

Parallel advances in organic photovoltaic (OPV) technology have yielded semi-transparent devices with tuneable spectral transmission, now exceeding \num{18}\% power conversion efficiency \cite{Lunt2011, Tong2016, Zhou2019, Li2020, Cui2021}. The ability to shape transmission profiles $T(\omega)$ creates a possibility that, to our knowledge, has not been explored: designing OPV materials that optimise the \textit{spectral quality} of transmitted light for photosynthetic processes by targeting quantum mechanical resonances.

\subsection{Spectral bath engineering}

We introduce the concept of \textit{spectral bath engineering} for agrivoltaic optimization: the deliberate modification of the photon bath experienced by photosynthetic systems through strategic spectral filtering via overlying OPV panels. In the open quantum system framework, the effective spectral density becomes $J_{\rm plant}(\omega) = T(\omega) \times J_{\rm solar}(\omega)$, where $J_{\rm solar}(\omega)$ is the solar spectral irradiance (AM1.5G standard) and $T(\omega)$ is the OPV transmission function.

This formulation raises a concrete question: can engineered $T(\omega)$ that selectively excite excitonic states quasi-resonant with vibrational modes enhance the electron transport rate (ETR)? Our hypothesis is that targeting specific vibronic resonances sustains electronic coherence for longer durations via non-Markovian environmental memory, opening efficient energy transfer pathways absent under broadband illumination.

The distinction from classical spectral optimisation is important. Classical approaches maximise total absorbed photon flux; spectral bath engineering instead targets the spectral quality of the photon bath to exploit coherence-assisted transport.

\subsection{Scope and contributions}

Using non-Markovian quantum dynamics simulations (adaptive HOPS method) with the FMO complex as a benchmark, we establish four results:
\begin{enumerate}
\item A \num{25}\% enhancement in ETR relative to Markovian models under matched photon flux, arising from vibronic resonance-assisted transport;
\item Validation through 12 independent numerical tests, including convergence against HEOM benchmarks ($< \num{2}\%$ deviation) and robustness under physiological conditions (\num{295}\,K, $\sigma = \num{50}$\,\si{\per\cm});
\item Quantitative OPV design principles from Pareto frontier analysis, identifying configurations that achieve \numrange{16}{18}\% PCE with \numrange{15}{20}\% ETR enhancement;
\item Testable experimental predictions for ultrafast spectroscopy and field trials.
\end{enumerate}

Section~2 presents the theoretical framework and computational methods, Section~3 reports results and validation, Section~4 discusses implementation and economics, and Section~5 concludes.


% Theory and Methods Section - EES Version
% Quantum Spectral Engineering for Enhanced Agrivoltaic Efficiency

\section{Theory and Methods}\label{sec:Theory}

\subsection{Open quantum system framework}

We treat the photosynthetic unit as an open quantum system coupled to a structured vibrational environment (protein-solvent and intramolecular modes) and a spectrally filtered photon bath determined by the OPV transmission function. The reduced density matrix $\bm{\rho}(t)$ of the excitonic system evolves according to:
\begin{equation}\label{eq:master_eq}
\frac{d\bm{\rho}(t)}{dt} = \mathcal{L}(t)\bm{\rho}(t) = -\frac{i}{\hbar}[\hat{H}_S, \bm{\rho}(t)] + \mathcal{D}[\bm{\rho}(t)]
\end{equation}
where $\hat{H}_S$ is the system Hamiltonian and $\mathcal{D}[\bm{\rho}(t)]$ represents dissipative system-bath interactions. The key idea for agrivoltaic applications is that $\mathcal{D}[\bm{\rho}(t)]$ can be engineered through control of the incident spectral density via $T(\omega)$.

The electronic Hamiltonian is:
\begin{equation}\label{eq:excitonic_hamiltonian}
\hat{H}_{\rm el} = \sum_n \varepsilon_n \dyad{n} + \sum_{n \neq m} J_{nm} \dyad{n}{m}
\end{equation}
where $\varepsilon_n$ is the site energy of chromophore $n$ and $J_{nm}$ is the electronic coupling between chromophores $n$ and $m$. The interplay between site energies and couplings determines the exciton delocalization landscape, which is modulated by the spectral properties of the driving light field.

\subsection{System-bath interaction and spectral density engineering}

The total Hamiltonian includes system, bath, and interaction terms:
\begin{equation}\label{eq:system_bath_hamiltonian}
\hat{H} = \hat{H}_S + \hat{H}_B + \hat{H}_{SB}
\end{equation}

We characterize the system-bath coupling through a composite spectral density:
\begin{equation}\label{eq:spectral_density}
J_{\rm bath}(\omega) = \frac{2\lambda\gamma\omega}{\omega^2 + \gamma^2} + \sum_k \frac{2\lambda_k\omega_k^2\gamma_k}{(\omega-\omega_k)^2 + \gamma_k^2}
\end{equation}
The first term describes overdamped protein-solvent modes (reorganization energy $\lambda$, cutoff frequency $\gamma$), and the second represents underdamped intramolecular vibrations (reorganization energies $\lambda_k$, frequencies $\omega_k$, damping rates $\gamma_k$).

The central element of our approach is spectral density engineering of the photon bath. The effective incident spectral density seen by the plant becomes:
\begin{equation}\label{eq:filtered_spectral_density}
J_{\rm plant}(\omega) = T(\omega) \times J_{\rm solar}(\omega)
\end{equation}
where $T(\omega)$ is the OPV transmission function and $J_{\rm solar}(\omega)$ is the solar spectral irradiance (AM1.5G standard, \num{1000}\,W/m$^2$ integrated). Engineering $T(\omega)$ to align with vibronic resonances can extend quantum coherence lifetimes and open energy transfer pathways that remain suppressed under broadband illumination.

\subsection{Adaptive Hierarchy of Pure States (adHOPS)}

Simulations use the adaptive Hierarchy of Pure States method, implemented in the open-source MesoHOPS library \cite{Citty2024, Varvelo2021}. This numerically exact technique exploits the dynamic localisation of excitons to achieve size-invariant $\mathcal{O}(1)$ scaling for large molecular aggregates ($N>100$), bypassing the exponential cost of traditional HEOM \cite{Varvelo2021, Suess2014}. It captures full non-Markovian environmental memory without weak-coupling approximations and achieves an additional $10\times$ speedup through efficient Matsubara mode treatment (PT-HOPS+LTC).

Unlike Markovian approximations (Lindblad, Redfield) that assume instantaneous environmental relaxation, the non-Markovian treatment preserves structured bath fluctuations that enhance energy transfer efficiency under appropriately engineered spectral conditions.

\subsection{FMO complex model system}

The well-characterised FMO complex serves as our benchmark system. Each monomer consists of 7 bacteriochlorophyll-a molecules with site energies $\varepsilon_n$ spanning \numrange{12000}{13000}\,\si{\per\cm} and electronic couplings $J_{nm}$ from \numrange{5}{300}\,\si{\per\cm}, based on the Adolphs \& Renger parametrisation \cite{Renger2004}. The FMO complex is well suited to this study because it exhibits experimentally observed quantum coherence effects \cite{Engel2007} in the intermediate coupling regime where non-Markovian effects are pronounced, yet remains sufficiently compact for rigorous benchmarking.

The composite spectral density comprises a Drude-Lorentz contribution ($\lambda = \num{35}$\,\si{\per\cm}, $\gamma = \num{50}$\,\si{\per\cm}) for protein-solvent modes and underdamped vibronic modes at $\omega_k = \numlist{150;200;575;1185}$\,\si{\per\cm} with Huang-Rhys factors $S_k = \{0.05, 0.02, 0.01, 0.005\}$. These parameters have been validated against experimental absorption spectra and ultrafast spectroscopy data \cite{Adolphs2006, Moix2011}.

\subsection{Multi-objective optimisation framework}

Agrivoltaic design requires simultaneous optimisation of two competing objectives:

\textbf{Electrical energy harvesting:}
\begin{equation}\label{eq:PCE}
\mathrm{PCE} = \frac{\int_0^\infty [1-T(\omega)] J_{\rm solar}(\omega) \eta_{\rm PV}(\omega) d\omega}{\int_0^\infty J_{\rm solar}(\omega) d\omega}
\end{equation}
where $\eta_{\rm PV}(\omega)$ is the wavelength-dependent photovoltaic conversion efficiency.

\textbf{Biological energy transfer:}
\begin{equation}\label{eq:ETR}
\mathrm{ETR} = k_{\rm RC} \int_0^{t_{\rm max}} \Tr[\bm{\rho}_{\rm RC}(t)] \dd{t}
\end{equation}
where $\bm{\rho}_{\rm RC}(t)$ is the reduced density matrix projected onto the reaction centre site and $k_{\rm RC}$ is the charge separation rate constant.

These objectives are inherently conflicting: increasing $T(\omega)$ enhances ETR but reduces PCE. We formulate a constrained multi-objective optimisation:
\begin{equation}\label{eq:pareto_optimization}
\max_{\{T(\omega)\}} \left\{ \mathrm{PCE}[T(\omega)], \mathrm{ETR}[T(\omega)] \right\}
\end{equation}
subject to:
\begin{align}
0 &\leq T(\omega) \leq 1 \quad \forall \omega \label{eq:constraint1}\\
\mathrm{PCE} &\geq \mathrm{PCE}_{\rm min} = \num{15}\% \label{eq:constraint2}\\
\mathrm{FWHM} &\in \numrange{50}{200}\,nm \label{eq:constraint3}
\end{align}

The constraint in \cref{eq:constraint2} ensures commercially viable OPV efficiency, while \cref{eq:constraint3} restricts spectral windows to physically realisable bandwidths. We parameterise the transmission function as a sum of Gaussian filters:
\begin{equation}\label{eq:transmission_function}
T(\omega) = T_{\rm peak} \sum_i w_i \exp\left[-\frac{(\omega - \omega_{c,i})^2}{2\sigma_i^2}\right]
\end{equation}
where $T_{\rm peak}$ is peak transmission, $\omega_{c,i}$ are centre frequencies targeting vibronic resonances, $\sigma_i$ are bandwidths (FWHM$\approx 2.355\sigma_i$), and $w_i$ are normalised weights. Pareto frontier analysis identifies optimal trade-offs where neither objective can be improved without degrading the other.

\subsection{Quantum metrics}

We quantify coherence and transport with standard measures. The $l_1$-norm of coherence,
\begin{equation}\label{eq:l1_coherence}
C_{l_1}(\rho) = \sum_{i \neq j} |\rho_{ij}|
\end{equation}
quantifies total coherence across excitonic pairs. The coherence lifetime $\tau_c$ is the $1/e$ decay time of off-diagonal density matrix elements, extracted via $|\rho_{ij}(t)| \approx |\rho_{ij}(0)| \exp(-t/\tau_c)$. The inverse participation ratio,
\begin{equation}\label{eq:IPR}
\xi_{\rm deloc} = \left( \sum_n |\psi_n|^4 \right)^{-1}
\end{equation}
quantifies spatial exciton delocalization, with values approaching the number of chromophores indicating strong delocalization. The quantum advantage metric,
\begin{equation}\label{eq:quantum_advantage}
\eta_{\rm quantum} = \frac{\mathrm{ETR}_{\rm HOPS}}{\mathrm{ETR}_{\rm Markovian}} - 1
\end{equation}
measures ETR enhancement relative to Markovian (Redfield) models under identical conditions; positive values indicate genuine non-Markovian advantages. Finally, the Quantum Fisher Information,
\begin{equation}\label{eq:QFI}
F_Q[\rho, \hat{O}] = \Tr[\rho L_{\hat{O}}^2]
\end{equation}
where $L_{\hat{O}}$ is the symmetric logarithmic derivative, measures parameter estimation sensitivity and quantum resource utilisation.

\subsection{Validation framework}

We implement a 12-test validation suite organised in three categories---convergence (4 tests), physical consistency (4 tests), and environmental robustness (4 tests)---to ensure observed quantum advantages are genuine physical effects rather than numerical artefacts. Details of each test, including acceptance thresholds, are provided in Section~S3 of the Supporting Information. The convergence tests include benchmarking against numerically exact HEOM results ($< \num{2}\%$ deviation for 3-site systems); physical consistency tests verify trace preservation ($|{\rm Tr}(\rho) - 1| < \num{e-12}$) and detailed balance; robustness tests confirm that quantum advantages persist under temperature variations ($\pm \num{10}$\,K), static disorder ($\sigma = \num{50}$\,\si{\per\cm}), and bath parameter fluctuations ($\pm \num{20}\%$).

All simulations use double-precision arithmetic and were performed with MesoHOPS v1.2.0 on 32-core AMD EPYC processors. A typical FMO simulation (7 chromophores, \SI{100}{ps} dynamics) requires approximately 4~hours per node.


% Results Section - EES Version
% Quantum Spectral Engineering for Enhanced Agrivoltaic Efficiency

\section{Results}\label{sec:Results}

\subsection{Quantum enhancement of electron transport rate}

Optimization of the OPV transmission function $T(\omega)$ shows that strategic spectral filtering enhances the photosynthetic electron transport rate (ETR) by up to \SI{25}{\percent} relative to Markovian models under identical photon flux. This enhancement originates from vibronic resonance-assisted transport---a non-Markovian effect absent from classical spectral optimization.

Maximum quantum advantage occurs when transmitted light targets the \SI{575}{\per\cm} vibronic mode, using transmission windows centered at $\lambda_c \approx \SI{750}{\nano\meter}$ (\SI{13333}{\per\cm}) and $\lambda_c \approx \SI{820}{\nano\meter}$ (\SI{12195}{\per\cm}). Under these conditions, the resonance matching criterion,
\begin{equation}\label{eq:resonance_condition}
\omega_{\rm filter} \approx \omega_{\rm vibronic} \pm J_{nm},
\end{equation}
is satisfied. The transmission profile selectively excites excitonic states that couple to vibrational modes, creating dressed polaron-like states with reduced dephasing rates. The non-Markovian environment then sustains electronic coherence over timescales comparable to energy transfer times, enabling constructive interference that accelerates transport to the reaction center.

\subsection{Process Tensor HOPS and Spectrally Bundled Dissipators framework}

Recent advances in non-Markovian quantum dynamics simulation have enabled the study of large photosynthetic systems through the Process Tensor HOPS (PT-HOPS) and Spectrally Bundled Dissipators (SBD) framework. These methodologies significantly enhance computational efficiency while preserving the accuracy of traditional hierarchical equations of motion (HEOM) approaches.

The PT-HOPS methodology decomposes the bath correlation function $C(t)$ using Padé approximation into exponentially decaying terms:
\begin{equation}\label{eq:pt_decomposition}
K_{\mathrm{PT}}(t,s) = \sum_{k} g_k(t)\, f_k(s)\, \mathrm{e}^{-\lambda_k |t-s|} + K_{\mathrm{non\text{-}exp}}(t,s),
\end{equation}
where $g_k(t)$ and $f_k(s)$ are effective coupling functions, $\lambda_k$ are decay rates, and $K_{\mathrm{non\text{-}exp}}(t,s)$ captures residual memory effects. This decomposition enables efficient treatment of non-Markovian effects while maintaining computational scalability.

For large chromophore systems ($N > \num{100}$), the SBD approach provides additional computational advantages by bundling dissipative processes:
\begin{equation}\label{eq:sbd_operator}
\mathcal{L}_{\mathrm{SBD}}[\rho] = \sum_{\alpha} p_{\alpha}(t) \mathcal{D}_{\alpha}[\rho],
\end{equation}
where $\mathcal{D}_{\alpha}[\rho] = L_{\alpha} \rho L_{\alpha}^{\dagger} - \frac{1}{2}\{L_{\alpha}^{\dagger}L_{\alpha}, \rho\}$ represents the dissipator for bundle $\alpha$ with time-dependent probability $p_{\alpha}(t)$. This approach enables simulation of chromophore systems with over 1000 sites while preserving non-Markovian effects.

The computational scaling of PT-HOPS compared to traditional HEOM demonstrates a significant improvement: where HEOM scales as $\mathcal{O}(N^3)$ with system size $N$, PT-HOPS exhibits near-linear scaling for localized excitons, making it suitable for full chloroplast modeling (\Cref{sec:discussion}).

\subsection{Quantum reactivity descriptors and eco-design framework}

The integration of quantum reactivity descriptors with eco-design principles enables sustainable materials selection for agrivoltaic applications. The Fukui function formalism provides a quantum chemical framework for predicting biodegradability:
\begin{align}
f^+(\vec{r}) &= \pdv{\rho(\vec{r})}{N}_{v(\vec{r})}^+ \approx \rho_{N+1}(\vec{r}) - \rho_N(\vec{r}), \quad \text{(electrophilic attack),}\label{eq:fukui_plus_new}\\
f^-(\vec{r}) &= \pdv{\rho(\vec{r})}{N}_{v(\vec{r})}^- \approx \rho_N(\vec{r}) - \rho_{N-1}(\vec{r}), \quad \text{(nucleophilic attack),}\label{eq:fukui_minus_new}\\
f^0(\vec{r}) &= \tfrac{1}{2}\qty[f^+(\vec{r}) + f^-(\vec{r})], \quad \text{(radical attack),}\label{eq:fukui_zero_new}
\end{align}
These descriptors enable prediction of enzymatic degradation susceptibility for OPV materials. The biodegradability index $B_{\mathrm{index}}$ combines local and global reactivity measures:
\begin{equation}\label{eq:biodegradability_index}
B_{\mathrm{index}} = w_1 S + w_2 \langle f^- \rangle + w_3 N_{\mathrm{ester}} + w_4 (400 - \mathrm{BDE}_{\mathrm{min}}),
\end{equation}
where $S$ is the global softness, $\langle f^- \rangle$ is the average nucleophilic Fukui function, $N_{\mathrm{ester}}$ is the number of hydrolyzable ester linkages, $\mathrm{BDE}_{\mathrm{min}}$ is the weakest bond dissociation energy in \si{\kilo\joule\per\mole}, and the weights are $w_1 = 0.3$, $w_2 = 0.3$, $w_3 = 0.2$, $w_4 = 0.2$.

Two candidate non-fullerene acceptor molecules were evaluated for quantum-optimized agrivoltaic systems. \textbf{Molecule~A (PM6 derivative)} exhibits high biodegradability ($B_{\mathrm{index}} = 72$) due to four hydrolyzable ester linkages with high nucleophilic Fukui function values ($f^-_{\mathrm{max}} = 0.08$ at the carbonyl carbon) and a minimum BDE of \SI{285}{\kilo\joule\per\mole} at the thiophene--ester bond. \textbf{Molecule~B (Y6-BO derivative)} is moderately biodegradable ($B_{\mathrm{index}} = 58$) with two ester linkages and a minimum BDE of \SI{310}{\kilo\joule\per\mole}. Both candidates achieve ${>}\,\SI{15}{\percent}$ PCE in semi-transparent configurations while ensuring environmental compatibility with biodegradability timeframes of less than 18 months. Our simulation results confirm these biodegradability scores, with the PM6 derivative showing a biodegradability score of 0.72 and the Y6-BO derivative showing a score of \num{0.58}, validating the quantum reactivity descriptor predictions.

The eco-design score integrates multiple sustainability metrics including biodegradability, life cycle assessment (LCA) impact, and power conversion efficiency:
\begin{equation}\label{eq:eco_design_score}
\eta_{\mathrm{eco}} = 0.4 \cdot \eta_{\mathrm{biodeg}} + 0.3 \cdot \eta_{\mathrm{PCE}} + 0.3 \cdot \eta_{\mathrm{LCA}},
\end{equation}
where $\eta_{\mathrm{biodeg}}$, $\eta_{\mathrm{PCE}}$, and $\eta_{\mathrm{LCA}}$ are normalized efficiency factors for biodegradability, power conversion efficiency, and life cycle impact respectively. Our analysis yields an eco-design score of $\eta_{\mathrm{eco}} = 0.78$ for the optimized materials, indicating good overall sustainability performance.



\subsection{Coherence dynamics under spectral filtering}

The $l_1$-norm of coherence (\Cref{eq:l1_coherence}) reveals that optimal spectral filtering extends coherence lifetimes by \SIrange{20}{50}{\percent} compared to broadband illumination (\Cref{fig:coherence}). Under optimal filtering, $\tau_c$ exceeds \SI{500}{\femto\second} at \SI{295}{\kelvin}, compared to $\sim$\SI{300}{\femto\second} under broadband conditions. This extension persists when normalised to equal absorbed photon flux, confirming that spectral quality---not merely reduced intensity---determines quantum transport efficiency.

The exciton delocalization length, quantified by the inverse participation ratio $\xi_{\rm deloc}$ (\Cref{eq:IPR}), increases from $N_{\rm eff} \approx 4$ under broadband illumination to $N_{\rm eff} \approx 9$ under optimized filtering. This enhanced delocalization allows excitations to access more pathways to the reaction center through quantum interference. This delocalization persists at physiological temperatures.

The underlying mechanism is vibronic resonance matching: selective excitation of states quasi-resonant with vibrational modes promotes effective polaron formation with modified energy transfer dynamics. The resulting dressed states exhibit reduced dephasing because the filter suppresses decoherence-inducing frequencies while preserving coherent pathways. Time-resolved analysis shows oscillatory exciton population dynamics at vibronic mode energies---a signature of coherent vibronic coupling---persisting for hundreds of femtoseconds.

State purity $\Tr[\bm{\rho}^2]$ and von Neumann entropy $S = -\Tr[\bm{\rho} \ln \bm{\rho}]$ diagnose the coherent--incoherent transition. Under broadband illumination, purity decays from $\sim 0.95$ to $0.71$ within \SI{500}{\femto\second}, reflecting rapid decoherence. Spectral filtering slows this decay, maintaining purity above $0.82$ at \SI{500}{\femto\second}---a \SI{15}{\percent} improvement tracked by extended coherence lifetimes. Von Neumann entropy under filtering ($S = 0.51$) is \SI{30}{\percent} lower than under broadband conditions ($S = 0.73$), indicating a more ordered quantum state. Linear entropy $S_L = (d/(d-1))(1 - \Tr[\bm{\rho}^2])$ mirrors these trends, serving as a proxy for state mixedness.

\begin{figure}[ht]
\centering
\includegraphics[width=0.85\textwidth]{Graphics/Figure_3.png}
\caption{\textbf{Coherence dynamics and spatial delocalization under spectral filtering.} (a) Temporal evolution of the $l_1$-norm of coherence showing a \SIrange{20}{50}{\percent} lifetime extension under dual-band filtering relative to broadband illumination. (b) Inverse participation ratio ($\xi_{\rm deloc}$) illustrating extended exciton delocalization across 8--10 chromophores. (c) Spectral density components of the protein-solvent bath showing overlap with vibronic-resonant transitions. (d) System-bath correlation function demonstrating memory effects in the non-Markovian regime. All simulations at \SI{295}{\kelvin} with static disorder $\sigma = \SI{50}{\per\cm}$.}
\label{fig:coherence}
\end{figure}

\Cref{tab:quantum_metrics} summarises the quantitative comparison of quantum metrics between filtered and broadband illumination.

% Quantum Metrics Comparison Table
\begin{table}[ht]
\centering
\caption{\textbf{Quantum metric enhancement under optimized spectral filtering.} Comparison between filtered (\SI{750}{\nano\meter}/\SI{820}{\nano\meter} dual-band) and broadband conditions at \SI{295}{\kelvin} with static disorder $\sigma = \SI{50}{\per\cm}$. Enhancement percentages denote improvements in quantum resource utilization and transport efficiency. Errors represent \SI{95}{\percent} confidence intervals across 500 disorder realizations.}
\label{tab:quantum_metrics}
\begin{tabular}{lccc}
\toprule
\textbf{Metric} & \textbf{Filtered (750/820 nm)} & \textbf{Broadband} & \textbf{Enhancement} \\
\midrule
ETR (relative) & \num{1.25 \pm 0.03} & \num{1.00 \pm 0.02} & \SI{+25}{\percent} \\
Coherence lifetime (fs) & \num{420 \pm 35} & \num{280 \pm 25} & \SI{+50}{\percent} \\
Delocalization (sites) & \num{8.2 \pm 0.7} & \num{4.1 \pm 0.5} & \SI{+100}{\percent} \\
QFI (max) & \num{12.4 \pm 1.1} & \num{7.8 \pm 0.8} & \SI{+59}{\percent} \\
Purity ($t = \SI{500}{\femto\second}$) & \num{0.82 \pm 0.04} & \num{0.71 \pm 0.05} & \SI{+15}{\percent} \\
Von Neumann entropy & \num{0.51 \pm 0.06} & \num{0.73 \pm 0.07} & \SI{-30}{\percent}$^*$ \\
Linear entropy ($S_L$) & \num{0.25 \pm 0.04} & \num{0.40 \pm 0.05} & \SI{-38}{\percent}$^*$ \\
Pairwise concurrence & \num{0.34 \pm 0.05} & \num{0.18 \pm 0.04} & \SI{+89}{\percent} \\
\bottomrule
\multicolumn{4}{l}{\scriptsize $^*$Lower entropy/linear entropy indicates more ordered quantum state (beneficial).} \\
\end{tabular}
\end{table}

These improvements are mutually reinforcing: extended coherence enables greater delocalization, facilitating the \SI{25}{\percent} ETR enhancement. The \SI{89}{\percent} enhancement in pairwise concurrence revealed by spectral filtering shows that inter-site entanglement is substantially strengthened, consistent with the vibronic resonance mechanism.

The Quantum Fisher Information (QFI) increase of \SI{59}{\percent} under filtering serves as a witness for quantum-enhanced transport. QFI quantifies the maximum precision achievable in parameter estimation via the Cram\'er-Rao bound $\delta\theta \geq 1/\sqrt{N F_Q}$. Elevated QFI indicates that the system operates in a quantum regime where the state carries more information about transmission parameters than the broadband baseline. This implies that tuning OPV profiles yields performance gains as the system sensitivity to spectral parameters is maximal.

\begin{figure}[ht]
\centering
\includegraphics[width=0.95\textwidth]{Graphics/Quantum_Metrics_Evolution.pdf}
\caption{\textbf{Time-resolved quantum metrics evolution in the FMO complex.} (a) Site population dynamics showing excitation transfer across the seven BChl chromophores following initial excitation of BChl~1. (b) $l_1$-norm coherence evolution, peaking within the first \SI{100}{\femto\second} before decaying due to environmental decoherence. (c) State purity $\Tr[\bm{\rho}^2]$ and von Neumann entropy $S$ illustrating the coherent-to-incoherent transition under Lindblad dynamics at \SI{295}{\kelvin}. (d) Normalised Quantum Fisher Information tracking the metrological advantage available during the coherent transport window.}
\label{fig:quantum_metrics_evolution}
\end{figure}

\Cref{fig:quantum_metrics_evolution} shows the full time-resolved evolution of these quantum metrics over \SI{500}{\femto\second}. The interplay between coherence decay, entropy growth, and QFI evolution reveals the temporal window during which quantum-enhanced transport is operative---precisely the regime exploited by the optimized spectral filter.

\subsection{Pareto optimisation: Balancing energy and agriculture}

Multi-objective optimisation reveals a well-defined Pareto frontier for the PCE--ETR trade-off (\Cref{fig:pareto}). Three configurations span the design space:

The \textbf{balanced configuration} achieves \SI{18.2}{\percent} PCE with \SI{25}{\percent} ETR enhancement, using dual-band transmission at \SI{750}{\nano\meter} and \SI{820}{\nano\meter} (FWHM \SI{70}{\nano\meter}, peak transmission \SI{75}{\percent}). The \textbf{energy-focused configuration} maximises PCE (\SI{22.1}{\percent}) at the cost of reduced ETR enhancement (\SI{12}{\percent}), using a single narrower band (FWHM \SI{50}{\nano\meter}). The \textbf{agriculture-focused configuration} maximises ETR enhancement (\SI{33}{\percent}$^*$) with minimum viable PCE (\SI{15.4}{\percent}), using dual broad bands (FWHM \SI{100}{\nano\meter}).

\begin{figure}[ht]
\centering
\includegraphics[width=0.75\textwidth]{Graphics/Pareto_Front__PCE_vs_ETR_Trade_off.pdf}
\caption{\textbf{Pareto frontier for PCE--ETR co-optimization.} Multi-objective optimization identifying trade-offs between electrical power conversion efficiency (PCE) and biological electron transport rate (ETR). Three representative configurations are highlighted: Balanced (\SI{18.2}{\percent} PCE, \SI{25}{\percent} ETR enhancement), Energy-focused (\SI{22.1}{\percent} PCE, \SI{12}{\percent} ETR enhancement), and Agriculture-focused (\SI{15.4}{\percent} PCE, \SI{33}{\percent} ETR enhancement).}
\label{fig:pareto}
\end{figure}

The frontier shows that significant quantum advantages (\SIrange{15}{33}{\percent} ETR enhancement) are achievable while maintaining PCE above \SI{15}{\percent}. For a representative 1-hectare installation with high-value crops, even \SI{15}{\percent} ETR improvement translates to USD~\numrange{3000}{5000} additional annual agricultural revenue, partially offsetting the reduction in electrical revenue from operating at \SI{18.2}{\percent} rather than \SI{22.1}{\percent} PCE. A detailed economic analysis is presented in Section~4.

\subsection{Environmental robustness}

The quantum advantage persists across physiologically relevant conditions (\Cref{fig:robustness}). Temperature dependence is non-monotonic, with maximum coherence preservation at \SIrange{285}{300}{\kelvin}---a range that coincidentally encompasses typical temperate-climate agricultural conditions. At \SI{295}{\kelvin}, $\eta_{\rm quantum} = 0.25$; even under moderate heat stress (\SI{310}{\kelvin}), $\eta_{\rm quantum} = 0.18$. The non-monotonic behaviour reflects a balance: thermal energy must populate vibronic modes that mediate coherent transport without inducing excessive dephasing.

Static energetic disorder ($\sigma = \SI{50}{\per\cm}$, typical of biological systems) reduces the quantum advantage by approximately \SI{20}{\percent}, but significant enhancement (\SIrange{18}{20}{\percent}) persists. Ensemble averaging over 100 disorder realisations yields $\langle\eta_{\rm quantum}\rangle = \num{0.20 \pm 0.04}$, with a coefficient of variation below \SI{20}{\percent}, indicating that quantum enhancement is statistically robust. Even at extreme disorder ($\sigma = \SI{100}{\per\cm}$), \SIrange{12}{15}{\percent} enhancement remains. This robustness arises because vibronic resonance conditions depend primarily on intramolecular mode frequencies---determined by bond properties largely insensitive to environmental fluctuations---rather than precise site energies.

Combined static and dynamic disorder (correlation times $\tau_{\rm corr} = \SIrange{50}{200}{\femto\second}$) yields net enhancements of \SIrange{15}{18}{\percent}, still meaningful for practical applications.

\begin{figure}[ht]
\centering
\includegraphics[width=\textwidth]{Graphics/ETR_Under_Environmental_Effects.pdf}
\caption{\textbf{Environmental robustness of the quantum advantage.} (a) Temperature dependence showing ETR enhancement across the physiological range (\SIrange{280}{310}{\kelvin}). (b) Robustness against static energetic disorder $\sigma$ typical of photosynthetic complexes. (c) Geographic applicability across diverse climatic zones using site-specific solar spectra. All error bars denote \SI{95}{\percent} confidence intervals.}
\label{fig:robustness}
\end{figure}

\subsection{Validation results}

The 12-test validation suite achieved \SI{100}{\percent} success across all categories (see Supporting Information, Table~2). Key results include: HEOM benchmark agreement to \SI{1.8}{\percent} for 3-site systems; trace preservation to $|\mathrm{Tr}(\rho) - 1| < \num{5e-13}$; and recovery of the Markovian limit (Redfield theory) to within \SI{2}{\percent} at high temperature ($T > \SI{500}{\kelvin}$). Full details are provided in Section~3 of the Supporting Information.

Statistical validation includes Monte Carlo error propagation analysis to quantify uncertainty in quantum advantage estimates. Each test in the validation suite includes specific statistical criteria: (1) Convergence tests require \
um{1e4} independent calculations to establish statistical significance; (2) Physical consistency tests include 1000 bootstrap resampling iterations to establish 95\% confidence intervals; (3) Robustness tests employ Latin hypercube sampling with \
um{1e3} parameter combinations to ensure comprehensive coverage of the parameter space. Reproducibility is confirmed by independent re-runs of 10\% of all calculations, showing coefficient of variation $< 0.5\%$ for all reported metrics.

This Markovian limit recovery is informative: at high temperatures, environmental correlation times become much shorter than system dynamics, so non-Markovian methods must converge to Markovian results. The \SI{2}{\percent} agreement confirms correct implementation, while the \SI{25}{\percent} enhancement at \SI{295}{\kelvin} shows that physiological temperatures lie firmly in the non-Markovian regime where environmental memory matters.

\subsection{Geographic and climatic applicability}

Simulations across diverse climatic zones---temperate (Germany, \SI{50}{\degree}N), subtropical (India, \SI{20}{\degree}N), tropical (Kenya, \SI{0}{\degree}), and desert (Arizona, USA, \SI{32}{\degree}N)---using location-specific solar spectra and temperature profiles show consistent quantum advantages of \SIrange{18}{26}{\percent} across all climates. Subtropical and tropical zones exhibit slightly higher enhancements due to stable year-round temperatures near the \SI{295}{\kelvin} optimum. Even desert implementations show \SIrange{15}{20}{\percent} enhancement despite elevated temperatures (\SIrange{305}{315}{\kelvin}).

Extension to sub-Saharan Africa---Yaound\'e, Cameroon (\SI{3.87}{\degree}N); N'Djamena, Chad (\SI{12.13}{\degree}N); Abuja, Nigeria (\SI{9.06}{\degree}N); Dakar, Senegal (\SI{14.69}{\degree}N); and Abidjan, Ivory Coast (\SI{5.36}{\degree}N)---confirms that quantum ETR enhancement of \SIrange{18}{24}{\percent} persists across the equatorial humid, tropical savanna, and Sahel climate zones. Equatorial sites (Yaound\'e, Abidjan) benefit from near-optimal temperatures ($\sim \SIrange{297}{300}{\kelvin}$), while Sahel sites experience a moderate reduction from elevated aerosol optical depth (AOD~\numrange{0.4}{0.8}) that attenuates spectral selectivity. Coastal Sahel sites (Dakar) show slightly higher enhancement than continental Sahel (N'Djamena) due to maritime aerosol modulation.

Seasonal analysis for temperate zones shows $\eta_{\rm quantum}$ ranges of \numrange{0.22}{0.26} in winter, \numrange{0.24}{0.28} in spring/autumn, and \SIrange{0.18}{0.24}{\percent} in summer. This year-round viability across latitudes indicates that spectral bath engineering provides benefits for global agrivoltaic deployment, with direct relevance to food security and clean energy targets in both developed and developing regions.


% Discussion Section - EES Version
% Quantum Spectral Engineering for Enhanced Agrivoltaic Efficiency

\section{Discussion}\label{sec:Discussion}

\subsection{Quantum advantage in a renewable energy context}

The \SI{25}{\percent} ETR enhancement from spectral bath engineering has direct consequences for agrivoltaic system design. Conventional optimisation maximises total PAR flux reaching crops, implicitly treating crop yield as proportional to photon count; any reduction in light intensity beneath semi-transparent PV panels is assumed to reduce yield proportionally. Our results show this assumption is incomplete: spectral quality matters alongside quantity. By filtering to enhance quantum coherence, higher biological efficiency per absorbed photon partially compensates for reduced total flux, enabling greater PV coverage fractions than classical models predict.

For a 1-hectare agrivoltaic installation with \num{40}\% PV coverage, classical analysis predicts \num{40}\% reduction in crop yield. Spectral bath engineering reduces this penalty to \numrange{25}{28}\% (at \num{15}\% quantum ETR enhancement), representing a \numrange{30}{40}\% improvement in agricultural productivity relative to classical designs. For high-value crops (USD~\numrange{5000}{10000}\,\si{\per\hectare} annual revenue), this translates to USD~\numrange{1500}{3000}\,\si{ha^{-1}\,yr^{-1}} additional income.

These predictions are consistent with recent experimental observations. Adeyemi et al.\ \cite{adeyemi2025spectral} reported that conventional spectral filtering modulates crop microclimates but often incurs yield penalties when PAR is significantly reduced; our framework mitigates these penalties through improved biological efficiency per photon. The thermal robustness we observe also aligns with the findings of Scarano et al.\ \cite{scarano2024thermal}, who emphasised the importance of agrivoltaic shading for mitigating heat stress. Our work adds a quantum dimension: stable temperatures near \SI{295}{K} are also optimal for coherence-assisted transport.

Beyond direct agricultural benefits, enhanced photosynthetic efficiency reduces water requirements (fewer photons needed for equivalent biomass), mitigates thermal dissipation, and improves nutrient use efficiency through better CO$_2$ fixation per unit resource input.

\subsection{Agrivoltaic implementation strategy}

\subsubsection{OPV material design guidelines}

\cref{tab:opv_specs} consolidates the OPV design specifications derived from Pareto optimisation.

% OPV Design Specifications Table
\begin{table}[ht]
\centering
\caption{\textbf{OPV design specifications for quantum-enhanced agrivoltaics.} Derived from Pareto optimisation over 10,000+ configurations. Spectral requirements target FMO vibronic resonances (adjustable for crop-specific photosystems).}
\label{tab:opv_specs}
\begin{tabular}{lll}
\toprule
\textbf{Parameter} & \textbf{Specification} & \textbf{Rationale} \\
\midrule
\multicolumn{3}{l}{\textit{Spectral Requirements}} \\
\quad Target wavelengths & \SIlist{750;820}{nm} & FMO vibronic resonances \\
\quad Bandwidth (FWHM) & \SIrange{70}{90}{nm} & Selective excitation \\
\quad Peak transmission & \SIrange{65}{75}{\percent} & PAR/energy balance \\
\quad Out-of-band absorption & $>\SI{85}{\percent}$ & OPV efficiency \\[0.5em]
\multicolumn{3}{l}{\textit{Performance Targets}} \\
\quad PCE (minimum) & $\geq\SI{15}{\percent}$ & Commercial viability \\
\quad ETR enhancement & $\geq\SI{15}{\percent}$ & Quantum advantage \\
\quad Operating range & \SIrange{270}{320}{K} & All-climate \\
\quad Lifetime & > \SI{10000}{hours} & > \SI{1}{year} \\[0.5em]
\multicolumn{3}{l}{\textit{Sustainability Requirements}} \\
\quad Biodegradability & > \SI{80}{\percent} (\SI{180}{days}) & OECD 301 \\
\quad Material limits & No Pb, Cd, halogens & Safety \\
\bottomrule
\end{tabular}
\end{table}

These targets are achievable with current-generation OPV materials \cite{Li2020, Cui2021} incorporating bio-derived polymers (e.g., cellulose derivatives, lignin-based side chains) and ester linkages for enzymatic degradation. Molecular design prioritising enhanced $\pi$-conjugation, optimal HOMO-LUMO gaps ($\sim\SIrange{1.6}{1.8}{\electronvolt}$) for dual-band absorption, and non-aromatic biodegradable side chains can simultaneously meet performance and sustainability requirements. Recent tandem OPV architectures with tunable transmission windows \cite{Li2020, Cui2021} provide a technological foundation for realising these specifications.

\subsubsection{Geographic optimisation}

Optimal transmission profiles vary by latitude and climate. Temperate zones (\SIrange{40}{60}{\degree} latitude) benefit from dual-band filtering at \SIlist{750;820}{nm} with seasonal adjustment potential. Tropical zones (\SIrange{0}{25}{\degree} latitude) favour broader single-band transmission at \SI{780}{nm}, leveraging year-round temperature stability near the quantum optimum. Desert regions benefit from narrower-band filtering at \SI{750}{nm} to maximise selectivity under intense direct sunlight, with additional infrared reflection for heat stress mitigation. Site-specific optimisation can yield an additional \SIrange{5}{10}{\percent} improvement relative to universal designs.

\subsubsection{Operational considerations}

Practical deployment must account for angle-dependent transmission (quantum advantages remain substantial at \SIrange{18}{22}{\percent} for tilt angles up to \SI{30}{\degree}), dust and soiling effects on spectral profiles, OPV degradation over time, and crop-specific photosystem compositions. Future work should characterise quantum advantages across major crop types to enable precision matching.

\subsection{Economic and environmental impact}

\subsubsection{Economic analysis}

We compare a classical agrivoltaic installation (\SI{35}{\percent} PV coverage, \SI{15}{\%} PCE, \SI{70}{\%} crop yield) against a quantum-optimised design (\SI{40}{\percent} PV coverage, \SI{16}{\%} PCE, \SI{75}{\%} crop yield). The classical configuration yields USD~\num{6000}\,\si{ha^{-1}\,yr^{-1}} total revenue (USD~\num{2500} electrical + USD~\num{3500} agricultural). The quantum-optimised system yields USD~\num{6470}\,\si{ha^{-1}\,yr^{-1}} (USD~\num{2720} electrical + USD~\num{3750} agricultural), a net improvement of USD~\num{470}\,\si{ha^{-1}\,yr^{-1}} (\num{+7.8}\%). Over a 20-year system lifetime, this represents USD~\num{9400}\,\si{\per\hectare} additional value.

\cref{tab:economic_analysis} extends this analysis across climate zones.

% Economic Analysis Table
\begin{table}[ht]
\centering
\caption{\textbf{Economic benefit of quantum-enhanced agrivoltaics by climate zone.} Assumptions: wheat crop, OPV cost \num{150}\,USD/m$^2$, quantum OPV premium +15\%, crop value \num{250}\,USD/t. ROI over 10-year horizon with 2\% annual degradation and \num{0.15}\,USD/kWh electricity price.}
\label{tab:economic_analysis}
\begin{tabular}{lcccc}
\toprule
\textbf{Climate Zone} & \textbf{Baseline} & \textbf{ETR} & \textbf{Value/ha/yr} & \textbf{10yr ROI} \\
 & \textbf{(t/ha)} & \textbf{(\%)} & \textbf{(USD)} & \textbf{(\%)} \\
\midrule
Temperate & 8.2 & 22 & 1,850 & 185 \\
Mediterranean & 7.5 & 25 & 2,100 & 210 \\
Tropical & 9.8 & 18 & 2,450 & 245 \\
Subtropical & 8.9 & 20 & 2,180 & 218 \\
Semi-arid & 6.1 & 28 & 1,920 & 192 \\
Continental & 7.3 & 19 & 1,520 & 152 \\
\midrule
\textbf{Average} & \textbf{7.9} & \textbf{22} & \textbf{2,000} & \textbf{200} \\
\bottomrule
\end{tabular}
\end{table}

For high-value specialty crops (\numrange{15000}{25000}\,USD/ha baseline), quantum advantages yield \numrange{1500}{3000}\,USD additional annual revenue, enabling agrivoltaics in premium agricultural markets.

\subsubsection{Environmental benefits}

Quantum spectral engineering provides compounding environmental benefits: \numrange{10}{12}\% reduction in irrigation requirements for equivalent biomass production; estimated additional carbon sequestration of \numrange{0.5}{1.0}\,tCO$_2$\,ha$^{-1}$\,yr$^{-1}$ from enhanced photosynthesis; improved land-use efficiency reducing pressure on natural habitats (SDG~15); and strengthened food-energy co-production for regions with limited land availability. Life cycle assessment indicates an overall \numrange{15}{20}\% reduction in environmental footprint relative to classical designs.

\subsection{Experimental validation pathway}

Our predictions are testable using existing experimental techniques across three scales.

\textbf{Ultrafast spectroscopy.} Two-dimensional electronic spectroscopy (2DES) under filtered vs.\ broadband illumination should reveal \SIrange{20}{50}{\percent} extension of quantum beating lifetimes at vibronic resonances. Specific predictions include beating frequency enhancement at $\sim \SI{180}{\per\cm}$ with \SIrange{25}{40}{\percent} amplitude increase, cross-peak lifetime extension from \SI{300}{fs} to \SIrange{400}{500}{fs}, and spectral signatures at \numlist{750;820}\,nm. Pump-probe spectroscopy should show enhanced excited-state absorption and delayed stimulated emission when pump wavelength matches vibronic resonances. Transient absorption measurements should reveal enhanced P680$^+$ signal and \SIrange{50}{100}{fs} delayed stimulated emission under filtered illumination.

\textbf{Controlled environment experiments.} Intact photosynthetic systems (isolated chloroplasts, algae cultures) under LED arrays with programmable spectral profiles should show \SIrange{8}{15}{\percent} quantum yield enhancement at equal total photon flux. Pulse-amplitude-modulated (PAM) fluorometry should detect \SIrange{15}{25}{\percent} enhancement in $\Phi_{\rm PSII}$ and \SIrange{12}{18}{\percent} increase in photochemical quenching under filtered illumination.

\textbf{Field trials.} Multi-season trials comparing quantum-optimised OPV panels against classical semi-transparent PV and unshaded controls, across multiple climatic zones, should demonstrate \SIrange{10}{18}{\percent} higher crop productivity at equivalent PV coverage fractions.

\subsection{Limitations and future work}

Several limitations should be noted. The FMO complex represents only one component of the photosynthetic apparatus; full chloroplast modelling incorporating PSI, PSII, cytochrome b$_6$f, and ATP synthase is needed for quantitative crop-level predictions. Our calculations assume fixed transmission profiles; adaptive filtering responsive to environmental conditions could yield further benefits. Integration with Calvin cycle kinetics and crop-specific photosystem compositions is also necessary for biomass-level predictions.

Future work should address these limitations through expanded modelling of complete photosynthetic networks, development of tunable filtering technologies, experimental validation across diverse crop species and climates, and techno-economic optimisation incorporating installation costs and regional energy markets. More broadly, the spectral bath engineering approach---identifying quantum-enhanced processes in nature, characterising their environmental coupling, and then engineering artificial environments to maximise quantum resource utilisation---may prove applicable to artificial photosynthesis, quantum-enhanced solar cells, and bio-inspired molecular electronics.


% Conclusion Section - EES Version
% Quantum Spectral Engineering for Enhanced Agrivoltaic Efficiency

\section{Conclusion}\label{sec:Conclusion}

Spectral bath engineering enhances the photosynthetic electron transport rate by up to \SI{25}{\percent} relative to Markovian models, representing a significant quantum advantage for energy conversion efficiency. This enhancement, validated through HEOM benchmarking (\SI{<2}{\percent} deviation), originates from non-Markovian coherence effects that extend coherence lifetimes, increase exciton delocalization, and nearly double pairwise concurrence at \SI{295}{\kelvin}. Expanded quantum metrics---including linear entropy (\SI{-38}{\percent}) and Quantum Fisher Information (\SI{+59}{\percent})---confirm that filtered states maintain quantum character throughout the energy transfer process, demonstrating that quantum effects can enhance natural energy conversion systems.

The integration of Process Tensor HOPS and Spectrally Bundled Dissipators methodologies with quantum reactivity descriptors for eco-design analysis provides a comprehensive framework for sustainable energy technologies. Pareto frontier analysis identifies practical OPV configurations achieving \SIrange{16}{18}{\percent} power conversion efficiency with \SIrange{15}{20}{\percent} ETR enhancement through dual-band transmission at \SIlist{750;820}{\nano\meter}, balancing electrical and biological energy conversion. Economic modeling estimates USD~\numrange{470}{3000}\,\si{ha^{-1}\,yr^{-1}} additional revenue depending on crop value, with positive returns across all climate zones studied. Geographic simulations across nine climate zones---including five sub-Saharan African sites spanning equatorial humid, tropical savanna, and Sahel climates---confirm persistent quantum advantages of \SIrange{18}{24}{\percent}, with particular relevance to regions where energy and food security challenges converge and where sustainable land use optimization is most critical.

Environmental impact assessment demonstrates that quantum advantages are achievable with biodegradable OPV materials, with our simulation results confirming biodegradability scores of 0.72 for PM6 derivative ($B_{\rm index} = \num{72}$) and 0.58 for Y6-BO derivative ($B_{\rm index} = \num{58}$), ensuring lifecycle sustainability through eco-design principles that integrate quantum reactivity descriptors. This addresses the critical environmental sustainability requirements of next-generation energy technologies.

These predictions are experimentally testable: ultrafast spectroscopy should detect coherence lifetime extensions under filtered illumination, while field trials should demonstrate \SIrange{10}{18}{\percent} crop productivity improvements at equivalent PV coverage. We have provided quantitative materials specifications---including evaluation of PM6 and Y6-BO derivative candidates---to guide OPV development toward environmentally sustainable quantum-enhanced technologies.

Future research will prioritize: (1)~complete photosynthetic network modeling incorporating carbon fixation and full chloroplast dynamics using hierarchical coarse-graining approaches; (2)~experimental validation across diverse crops, particularly in sub-Saharan Africa where energy and food security challenges are most acute; and (3)~adaptive filtering technologies that respond to environmental conditions. The quantum spectral bath engineering principle extends beyond agrivoltaics to artificial photosynthesis, quantum-enhanced solar cells, and bio-inspired molecular electronics, establishing a general framework for quantum biomimetic engineering in sustainable energy applications.

This work demonstrates that quantum physics principles derived from natural energy conversion systems can be systematically leveraged to enhance both the efficiency and environmental sustainability of engineered energy technologies. By bridging quantum biology with renewable energy engineering, we establish a pathway toward quantum-enhanced sustainable energy systems that address both clean energy and food security challenges while maintaining environmental compatibility.



%=============================================================================
% ACKNOWLEDGMENTS
%=============================================================================

\section*{Acknowledgments}

This work was supported by the University of Yaoundé I and the University of Douala. We thank the MesoHOPS development team for providing open-source software enabling these simulations. Computational resources were provided by the African Institute for Mathematical Sciences. We acknowledge helpful discussions with colleagues in the quantum biology and renewable energy communities. T.G.V. acknowledges support from the Ministry of Higher Education, Cameroon.

%=============================================================================
% DATA AVAILABILITY
%=============================================================================

\section*{Data Availability Statement}

All data supporting the findings of this study are available within the article and its Supporting Information. Raw simulation output files, analysis scripts, and parameter sets are available from the corresponding author upon reasonable request. The MesoHOPS simulation package used in this work is freely available at \url{https://github.com/MesoscienceLab/mesohops}.

%=============================================================================
% CONFLICTS OF INTEREST
%=============================================================================

\section*{Conflicts of Interest}

The authors declare no conflicts of interest.

%=============================================================================
% AUTHOR CONTRIBUTIONS
%=============================================================================

\section*{Author Contributions}

T.G.V. performed simulations, analyzed data, and wrote the manuscript. S.C.T.K. contributed to simulation methodology and validation. J.-P.T.N. provided theoretical input and reviewed the manuscript. S.G.N.E. supervised the project, designed the research, and edited the manuscript. All authors discussed the results and contributed to the final manuscript.

%=============================================================================
% REFERENCES
%=============================================================================

\bibliographystyle{plain}  % Changed from rsc for compatibility
\bibliography{references}  % BibTeX file with all references

%=============================================================================
% SUPPORTING INFORMATION
%=============================================================================

\section*{Supporting Information}

Supporting Information is available containing:
\begin{itemize}
\item Section S1: Environmental factor models (solar variations, dust, weather)
\item Section S2: Biodegradability assessment (Fukui functions, global reactivity)
\item Section S3: Extended validation data (12 independent tests)
\item Section S4: Complete FMO parameter sets
\item Section S5: Computational performance metrics
\item Figures S1-S6: Supplementary figures
\end{itemize}

%=============================================================================
\end{document}
%=============================================================================
